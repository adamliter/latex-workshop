% !TEX encoding = UTF-8 Unicode
% !TEX TS-program = arara
% arara: xelatex: { shell: yes }
% arara: xelatex: { shell: yes, synctex: yes }

\documentclass{article}

\usepackage{fontspec}
\setmainfont{CMU Serif Roman}

\usepackage[
	margin=0.75in
]{geometry}
\setlength{\parindent}{0em}
\setlength{\parskip}{1ex}

\usepackage[
	%mark,% put \gitMark inside of \cfoot of fancyhdr instead
	missing={master}
]{gitinfo2}

\usepackage{lastpage}

\usepackage{fancyhdr}
\pagestyle{fancy}

\usepackage{etoolbox}
\makeatletter
\patchcmd{\@fancyhead}{\rlap}{\color{gray}\small\sffamily\rlap}{}{}
\patchcmd{\@fancyfoot}{\rlap}{\color{gray}\small\sffamily\rlap}{}{}
\makeatother

\renewcommand{\headrulewidth}{0pt}
\lhead{}
\chead{}
\rhead{}
\lfoot{}
\cfoot{\gitMark}
\rfoot{\thepage/\pageref*{LastPage}}

\usepackage{titlesec}
\titleclass{\subsubsubsection}{straight}[\subsection]

\newcounter{subsubsubsection}[subsubsection]
\renewcommand\thesubsubsubsection{\thesubsubsection.\arabic{subsubsubsection}}
\renewcommand\theparagraph{\thesubsubsubsection.\arabic{paragraph}} % optional; useful if paragraphs are to be numbered

\titleformat{\subsubsubsection}
  {\normalfont\normalsize\bfseries}{\thesubsubsubsection}{1em}{}
\titlespacing*{\subsubsubsection}
{0pt}{3.25ex plus 1ex minus .2ex}{1.5ex plus .2ex}

\makeatletter
\renewcommand\paragraph{\@startsection{paragraph}{5}{\z@}%
  {3.25ex \@plus1ex \@minus.2ex}%
  {-1em}%
  {\normalfont\normalsize\bfseries}}
\renewcommand\subparagraph{\@startsection{subparagraph}{6}{\parindent}%
  {3.25ex \@plus1ex \@minus .2ex}%
  {-1em}%
  {\normalfont\normalsize\bfseries}}
\def\toclevel@subsubsubsection{4}
\def\toclevel@paragraph{5}
\def\toclevel@paragraph{6}
\def\l@subsubsubsection{\@dottedtocline{4}{7em}{4em}}
\def\l@paragraph{\@dottedtocline{5}{10em}{5em}}
\def\l@subparagraph{\@dottedtocline{6}{14em}{6em}}
\makeatother

\setcounter{secnumdepth}{4}
\setcounter{tocdepth}{4}

\usepackage[
	justification=centering
]{caption}
\usepackage{booktabs}

\usepackage{minted}
\setminted{frame=single}

\usepackage{calc}

\usepackage{forest}
\forestset{
	qtree/.style={
		for tree={
			parent anchor=south,
			child anchor=north,
			align=center,
			inner sep=1pt
		}
	},
	.style={
		baseline
	}
}
\usetikzlibrary{matrix}
\usetikzlibrary{decorations.pathreplacing}

\definecolor{fblue}{RGB}{92,144,192}

\newcommand\myfolder[2][fblue]{%
\begin{tikzpicture}[overlay]
  \draw[fill=#1!82!black] 
    (-20pt,14pt) -- 
    (-17pt,17pt) --
    (-1pt,17pt) --
    (1pt,19pt) --
    (12pt,19pt) --
    (14pt,17pt) --
    (17pt,17pt) --
    (20pt,14pt) -- cycle;
  \draw[line width=0.75pt,white] 
    (-18.5pt,14pt) -- 
    (-15.5pt,16.5pt) --
    (0.5pt,16.5pt) --
    (2pt,18.3pt) --
    (10.5pt,18.3pt) --
    (12.5pt,16.5pt) --
    (15.5pt,16.5pt) --
    (18.5pt,14pt) -- cycle;
  \draw[rounded corners,top color=#1,bottom color=#1!30] 
    (-23pt,14pt) -- 
    (23pt,14pt) --
    (21pt,-14pt) --
    (-21pt,-14pt) -- cycle;
  \draw[rounded corners,line width=1pt,white] 
    (-22pt,13pt) -- 
    (22pt,13pt) --
    (20pt,-13pt) --
    (-20pt,-13pt) -- cycle;
\end{tikzpicture}%
\makebox[0pt]{\raisebox{-3pt}{{\ttfamily\small#2}}}%
}

\usepackage{xcolor}
\definecolor{darkred}{HTML}{B22613}

\usepackage[
	colorlinks
]{hyperref}
\hypersetup{
	linkcolor=darkred,
	citecolor=gray,
	urlcolor=cyan
}

\usepackage{hologo}

\usepackage{textglos}
\usepackage[
	nomain,
	nostyles
]{glossaries}
\usepackage{glossary-inline}
\usepackage{leipzig-mod}
\makeglossaries
\glsdisablehyper

\newleipzig{pol}{pol}{polite}

\usepackage[
	shadedcells,
	notipa
]{ot-tableau}

\usepackage{gb4e}
\noautomath

\title{\LaTeX{} workshop (for linguists)%
\thanks{%
This handout was originally used as the basis for an impromptu \LaTeX{} workshop that I gave at the 2015 Chicago Linguistic Institute.
Since then, I have updated the PDF, and I will continue to periodically keep it up to date.
If you use this handout, I encourage you to occasionally check for a more up-to-date version of it because things do change.
The most recent version of this PDF can be found at \url{https://bit.ly/latex-workshop}.
Furthermore, the \File{.tex} file that produced this PDF is available at \url{https://github.com/adamliter/latex-workshop}.
There are a few dependencies and oddities that might prevent you from actually being able to compile it yourself on your machine.
(It definitely won't compile on one of the online editors because you need to enable shell escape.)
Anyway, it's not worth explaining these things any further in a footnote.
This document already has enough footnotes as it is, but you're welcome to look at the source code of the document to get an idea of how to write something in \LaTeX.
Pull requests and/or suggested changes to the document are definitely welcome, too!%
}%
}
\author{\begin{tabular}{cc}Adam Liter\\\Email{adam.liter@gmail.com}\end{tabular}}
\date{Last updated: \today}

\newcommand*{\Email}[1]{\href{mailto:#1}{\nolinkurl{#1}}}

\newcommand*{\File}[1]{\texttt{#1}}
\newcommand*{\Package}[1]{\texttt{#1}}
\newcommand*{\Directory}[1]{\texttt{#1}}

\newcommand*{\KEY}[1]{\textbf{#1}}
\newcommand*{\Semantics}[1]{\textbf{#1}}

\newcommand*{\TitleText}[1]{\textit{#1}}

\newcommand*{\Constr}[1]{\textsc{#1}}

\newcommand*{\IE}{\textit{i.e.},}
\newcommand*{\ETC}{\textit{etc.}}

\begin{document}

\maketitle
\thispagestyle{fancy}

\tableofcontents

% !TEX encoding = UTF-8 Unicode
% !TEX root = ../latex-workshop-for-linguists.tex

\section{Setting up your machine}
\label{sec:setting-up-your-machine}

\subsection{Installing a \TeX{} distribution}
\label{subsec:installing-a-tex-distro}

There are two relatively new and popular web editors for \LaTeX---namely, \href{https://www.sharelatex.com/}{ShareLaTeX} and \href{https://www.overleaf.com/}{Overleaf}.
The web editors are useful tools for collaboratively working on a \LaTeX{} document.
They are also useful because you do not have to bother with installing your own \TeX{} distribution on your computer.

Nonetheless, there are several advantages to installing a \TeX{} distribution on your computer and being able to edit and compile \File{.tex} documents locally.
The biggest advantage is being able to maintain a single master \File{.bib} file and use it in all of your \File{.tex} documents for references.
For more on this, please read \S\ref{subsec:local-files} and \S\ref{subsec:bibliographies}.

\subsubsection{Mac}
\label{subsubsec:tex-distro:mac}

If you're on a Mac, you should install \href{https://tug.org/mactex/}{{Mac\TeX}}.%
\footnote{%
If you upgrade to the new Mac operating system in the fall of 2015, please read Herbert Schulz's document \href{https://tug.org/mactex/UpdatingForElCapitan.pdf}{\TitleText{{Mac\TeX}-2015
and El Capitan}}.
If you do not follow the instructions in that document after you upgrade to El Capitan, you will run into problems.%
}
{Mac\TeX} is all TeX Live underneath with just a thin wrapper that makes things work smoothly on a Mac.
{Mac\TeX} also installs two editors---TeXShop and TeXworks---and a program for managing a \File{.bib} file, called BibDesk.

\subsubsection{Linux}
\label{subsubsec:tex-distro:linux}

Do \emph{not} install TeX Live on Linux via your package manager!
The \TeX{} distribution that you will get from your package manager will most likely be out of date, which will preclude you from being able to update packages.

Instead, you should \href{http://tex.stackexchange.com/q/1092/32888}{install a ``vanilla'' version of TeX Live}.

\subsubsection{Windows}
\label{subsubsec:tex-distro:windows}

The easiest thing to install on Windows is \hologo{MiKTeX},%
\footnote{%
Disclaimer: I know very little about Windows and \hologo{MiKTeX}.%
} %
which is a different distribution than TeX Live.
\hologo{MiKTeX} doesn't install every package but instead installs a minimal distribution and allows you to install packages on the fly when compiling your document if the requisite package is not already installed.

At one point, there were security concerns about \hologo{MiKTeX} and thus it was preferable to install TeX Live.
However, these security concerns seem to have been mitigated, and it's not clear that there is a huge reason to prefer a TeX Live installation on Windows.%
\footnote{%
See \href{http://tex.stackexchange.com/q/20036/32888}{this question and its answers on TeX.SX} for discussion.%
}
Moreover, it is not as straightforward to install TeX Live as it is to install \hologo{MiKTeX}.
Nonetheless, if you wish to do so, see \href{http://www.tug.org/texlive/acquire-netinstall.html}{here}.

\subsection{Keeping your \TeX{} distribution up to date}
\label{subsec:keeping-your-tex-distro-up-to-date}

It is good practice to periodically update your \TeX{} distribution.
A \TeX{} distribution includes a bunch of packages, which are periodically edited by their maintainers.
These packages are hosted on the \href{http://ctan.org/}{Comprehensive \TeX{} Archive Network (CTAN)}.
You should thus periodically update things in case the maintainers of packages find a bug and fix that bug or in case they add new features to the package.%
\footnote{%
One thing that is also great about \LaTeX, in stark contradistinction to Word, is its backward compatibility.
That is, even if package authors introduce new features, they will make sure that any document you previously typeset using their package will be something that you can still typeset using the new updated version of their package.
If package authors do decide to break backwards compatibility, they will usually create a new package with an entirely new name, which effectively maintains backward compatibility because the old package will always be available for use.
On the other hand, with Word, you're lucky if you can open a file from last year's version of Word with this year's version of Word, much less have the formatting look even remotely the same.
With \LaTeX, you could typeset a \LaTeX{} file written in 1739 and the output you get would be identical to the output you got in 1739.%
}

In addition to periodically updating the packages, you will also want to periodically update the entire distribution.
Just like with packages, new features are developed or bugfixes are sometimes made to the engines themselves and other binaries that are the core of a \TeX{} distribution.

For TeX Live, there is a new distribution that is released every year.
The current one is TeX Live 2015.

When the new distribution is about to be released, the old one is ``frozen''.
Once it is frozen, you will no longer be able to update packages, so you will want to install the newest version of TeX Live for any new features or bugfixes to the engines and other binaries as well as for the ability to continue to periodically update packages.

\subsection{Local files}
\label{subsec:local-files}

One thing you will presumably also want to do at some point is set up a directory for local files that you want to be accessible to all of your \File{.tex} files, regardless of where that \File{.tex} file is actually stored on your machine.

The most obvious use case for such a directory is for the purposes of maintaining a single master bibliography file on your computer that can be used for citations in all of your \File{.tex} files (see \S\ref{subsec:bibliographies}).

Where and how to set up this directory depends on your distribution, TeX Live or \hologo{MiKTeX}.
What is common to both cases, however, is that the directory must conform to the standard \TeX{} Directory Structure (TDS) hierarchy.
A minimal example of a directory structure that conforms to this standard is given in Figure~\ref{fig:TDS}.%
\footnote{%
\label{fn:TDS}
There are even more folders in a maximal TDS directory, but the ones depicted in Figure~\ref{fig:TDS} are probably enough for most use cases.
If you're interested in reading more about TDS, you can do so at \url{https://www.tug.org/tds/tds.pdf}.%
}

\begin{figure}[htbp]
	\centering
	\scalebox{0.65}{
		\begin{forest}
			for tree={
				parent anchor=south,
				child anchor=north,
				node options={
					inner sep=11pt
				},
				l sep=25pt,
				s sep=40pt,
				delay={
					content=\myfolder{#1}
				}
			}
			[texmf
				[bibtex
					[bib]
					[bst]
				]
				[doc]
				[fonts, l sep+=60pt 
					[afm]
					[map]
					[misc]
					[pk]
					[source]
					[tfm]
					[type1]
				]
				[generic]
				[scripts]
				[source]
				[tex
					[context]
					[generic]
					[latex
						[biblatex
							[bbx]
							[cbx]
						]
					]
					[plain]
					[xelatex]
					[xetex]
				]
			]
		\end{forest}
	}
	\caption{A minimal directory that conforms to the TDS standard}
	\label{fig:TDS}
\end{figure}

It is necessary to conform to this standard so that the engine you use to compile your \File{.tex} file can find certain types of files.
For example, if you maintain a single master \File{.bib} file, it should be placed in the folder \Directory{texmf/bibtex/bib}.
If you put it in any other folder, the engine you use to compile your document will not find it because it is only programmed to look for bibliography files inside the \Directory{texmf/bibtex/bib} folder.

Note that you should \emph{only} put stuff in this directory that you want to be available to all of your \File{.tex} files, such as a master bibliography file, a custom package or style file that is not part of \href{http://ctan.org/}{CTAN}, \ETC.
This directory is \emph{not} for your \File{.tex} files.
If you wish to learn more, see fn.~\ref{fn:TDS}.

In what follows, I describe how to set up a local TDS-compliant directory for both TeX Live and \hologo{MiKTeX}.
For further discussion, see \href{http://tex.stackexchange.com/q/1137/32888}{this question and its answers on TeX.SX}.

\subsubsection{TeX Live}
\label{subsubsec:local-files:tex-live}

In TeX Live, engines are set up to look in certain places for files that your \File{.tex} file might depend on.
TeX Live specifically provides two places for users to put their own files, such as style files or bibliography files.
These two places are identified by their variable names, TEXMFHOME and TEXMFLOCAL.

TEXMFHOME and TEXMFLOCAL have the same semantics; that is to say, they are both places where users can put their own files that are not part of TeX Live.
However, TEXMFLOCAL will be overwritten every time you install a new version of TeX Live.
For this reason, it is probably best to keep all of your local files in TEXMFHOME.%
\footnote{%
For discussion, see \url{http://www.tex.ac.uk/FAQ-what-TDS.html}.%
}

TEXMFHOME usually refers to the path \Directory{\textasciitilde/Library/texmf} on Mac, the path \Directory{\textasciitilde/texmf} on Linux, and the path \Directory{C:\textbackslash Users\textbackslash <user name>\textbackslash texmf} on Windows.%
\footnote{%
If you're still on Windows XP, it should be \Directory{C:\textbackslash Documents and Settings\textbackslash <user name>\textbackslash texmf} instead of \Directory{C:\textbackslash Users\textbackslash <user name>\textbackslash texmf}.%
}

If you are unsure what the value of TEXMFHOME is, you can check it by going to the command line and running \mintinline{sh}|kpsewhich -var-value=TEXMFHOME|.
If, for example, you're on a Mac and haven't changed the default setting, this should return the following directory path: \Directory{/Users/<user name>/Library/texmf}.%
\footnote{%
The `\Directory{\textasciitilde}' is used as shorthand for a user's home directory.
That is to say, `\Directory{\textasciitilde/Library/texmf}' is the same as `\Directory{/Users/<user name>/Library/texmf}' on a Mac.%
}

Even though the variable TEXMFHOME has a value, the folder might not exist.
You need to create it.
There are two options for doing this.
You can create the folder in that exact location, or you can create the folder in a different location and make a symbolic link (symlink) at the location of TEXMFHOME that points to where the folder is actually located on your computer.

I would highly recommend the second option.
If you do the second option, you could keep your texmf folder in the cloud with \href{https://dropbox.com/}{Dropbox}, for example, which would allow you to have a backup of the folder as well as allow you to sync the folder across multiple machines.%
\footnote{%
Actually, what I would really recommend is keeping your local texmf folder under version control using, for example, \mintinline{sh}|git|, and keeping it on \href{https://github.com/}{GitHub} or \href{https://bitbucket.org/}{Bitbucket}.
However, explaining how to use a version control system is beyond the scope of this document.
If you know how to use one, I assume you can extrapolate from the setup instructions given in \S\ref{subsubsubsec:symlink-texmf-folder-into-TEXMFHOME}.%
}
\S\ref{subsubsubsec:create-a-texmf-folder-at-TEXMFHOME} explains the first option, and \S\ref{subsubsubsec:symlink-texmf-folder-into-TEXMFHOME} explains the second option.

\subsubsubsection{Create a texmf folder at TEXMFHOME}
\label{subsubsubsec:create-a-texmf-folder-at-TEXMFHOME}

Note that this method is discouraged.
Instead, I suggest creating the folder somewhere else and making a symlink to it at TEXMFHOME.
See \S\ref{subsubsubsec:symlink-texmf-folder-into-TEXMFHOME}.

\paragraph{OSX/Linux}

If you're on a Mac or Linux, you can open a terminal and copy and paste the commands that are shown in Listing~\ref{lst:OSX-Linux-make-TEXMFHOME} and hit ENTER.%
\footnote{%
Note that this will only work after TeX Live has been installed, because the command line tool \mintinline{sh}|kpsewhich| is part of TeX Live.%
}

\begin{listing}[htbp]
	\centering
	\begin{minted}{sh}
mkdir -p $(kpsewhich -var-value=TEXMFHOME)/{doc,generic,scripts,source} && \
mkdir -p $(kpsewhich -var-value=TEXMFHOME)/bibtex/{bib,bst} && \
mkdir -p $(kpsewhich -var-value=TEXMFHOME)/fonts/{afm,map,misc,pk,source,tfm,type1} && \
mkdir -p $(kpsewhich -var-value=TEXMFHOME)/tex/{context,generic,latex,plain,xelatex,xetex} && \
mkdir -p $(kpsewhich -var-value=TEXMFHOME)/tex/latex/biblatex/{bbx,cbx}
	\end{minted}
	\caption{Make a minimal TDS-compliant directory at TEXMFHOME on OSX or Linux}
	\label{lst:OSX-Linux-make-TEXMFHOME}
\end{listing}

\paragraph{Windows}

If you're on Windows, you can copy and paste the commands shown in Listing~\ref{lst:Windows-make-TEXMFHOME} into the Command Prompt and hit enter.
(Note that this is currently untested, since I do not have a Windows machine.
If this works for you, please let me know so that I can remove this disclaimer.)

\begin{listing}[htbp]
	\centering
	\begin{minted}{bat}
FOR /F "delims=" %i IN ('kpsewhich -var-value=TEXMFHOME') DO ^
FOR %d IN (doc, generic, scripts, source) DO mkdir /S %i\%d & ^
FOR %d IN (bib, bst) DO mkdir /S %i\bibtex\%d & ^
FOR %d IN (afm, map, misc, pk, source, tfm, type1) DO mkdir /S %i\fonts\%d & ^
FOR %d IN (context, generic, latex, plain, xelatex, xetex) DO mkdir /S %i\tex\%d & ^
FOR %d IN (bbx, cbx) DO mkdir /S %i\tex\latex\biblatex\%d
	\end{minted}
	\caption{Make a minimal TDS-compliant directory at TEXMFHOME on Windows}
	\label{lst:Windows-make-TEXMFHOME}
\end{listing}

\subsubsubsection{Symlink texmf folder into TEXMFHOME}
\label{subsubsubsec:symlink-texmf-folder-into-TEXMFHOME}

An alternative to creating the texmf folder in the precise location that the variable TEXMFHOME points to is to instead create the folder in an alternative location, and then create a symlink at the value of TEXMFHOME that points to the texmf folder.

I would recommend this method because it allows you to create the texmf folder inside of your Dropbox folder, for example, which makes a backup of the texmf folder in the cloud and also allows you to sync your texmf folder across multiple machines.

In what follows, I give instructions for how to do this with Dropbox.
If you wish to make the folder somewhere other than inside of your Dropbox folder, just replace the relevant bits of the directory paths in the commands that are given below.

\paragraph{OSX/Linux}

If you're on a Mac or Linux, the default location of your Dropbox folder should be \Directory{\textasciitilde/Dropbox}.
Therefore, you can open a terminal and copy and paste the commands shown in Listing~\ref{lst:OSX-Linux-make-TEXMFHOME-in-Dropbox} and hit ENTER.

\begin{listing}[htbp]
	\centering
	\begin{minted}{sh}
mkdir -p ~/Dropbox/texmf && \
cd ~/Dropbox/texmf && \
mkdir -p {doc,generic,scripts,source} && \
mkdir -p bibtex/{bib,bst} && \
mkdir -p fonts/{afm,map,misc,pk,source,tfm,type1} && \
mkdir -p tex/{context,generic,latex,plain,xelatex,xetex} && \
mkdir -p tex/latex/biblatex/{bbx,cbx} && \
ln -s ~/Dropbox/texmf $(kpsewhich -var-value=TEXMFHOME)
	\end{minted}
	\caption{Make a TDS-compliant directory in Dropbox and symlink it into TEXMFHOME on OSX or Linux}
	\label{lst:OSX-Linux-make-TEXMFHOME-in-Dropbox}
\end{listing}

\paragraph{Windows}

If you're on Windows Vista or up, you can copy and paste the commands shown in Listing~\ref{lst:Windows-make-TEXMFHOME-in-Dropbox} into the Command Prompt and hit ENTER.
If you're on Windows XP, why are you still running Windows XP?
(Note that this is currently untested, since I do not have a Windows machine.
If this works for you, please let me know so that I can remove this disclaimer.)

\begin{listing}[htbp]
	\centering
	\begin{minted}{bat}
mkdir %HOMEPATH%\Dropbox\texmf & ^
chdir %HOMEPATH%\Dropbox\texmf & ^
FOR %d IN (doc, generic, scripts, source) DO mkdir /S %d & ^
FOR %d IN (bib, bst) DO mkdir /S bibtex\%d & ^
FOR %d IN (afm, map, misc, pk, source, tfm, type1) DO mkdir /S fonts\%d & ^
FOR %d IN (context, generic, latex, plain, xelatex, xetex) DO mkdir /S tex\%d & ^
FOR %d IN (bbx, cbx) DO mkdir /S tex\latex\biblatex\%d & ^
FOR /F "delims=" %i IN ('kpsewhich -var-value=TEXMFHOME') DO mklink /J %i %HOMEPATH%\Dropbox\texmf
	\end{minted}
	\caption{Make a TDS-compliant directory in Dropbox and symlink it into TEXMFHOME on Windows}
	\label{lst:Windows-make-TEXMFHOME-in-Dropbox}
\end{listing}

\subsubsection{\hologo{MiKTeX}}
\label{subsubsec:local-files:miktex}

\hologo{MiKTeX} is different from TeX Live in that it allows users to select directories to be used for storing local files through a graphical user interface, rather than having an environment variable that maps to such a directory.

In order to get things set up on \hologo{MiKTeX}, you will want to first set up a TDS-compliant directory somewhere on your computer.
\hologo{MiKTeX} recommends making the directory at \Directory{C:\textbackslash Local TeX Files}.

However, since it doesn't really matter, I would instead recommend creating the texmf folder inside of a Dropbox folder so that you can have a backup of your texmf folder in the cloud and also sync it across machines if you have multiple machines.

To do this, you can copy and paste the following commands shown in Listing~\ref{lst:Windows-make-local-texmf-MiKTeX} into the Windows Command Prompt and hit ENTER.%
\footnote{%
Note that in this case the folder called \Directory{Local TeX Files} is the same as the \Directory{texmf} folder depicted in Figure~\ref{fig:TDS}.
You should \emph{not} put a folder called \Directory{texmf} inside of \Directory{Local TeX Files}.
Instead, treat \Directory{Local TeX Files} as the \Directory{texmf} folder.
The commands given in Listing~\ref{lst:Windows-make-local-texmf-MiKTeX} will do this automagically for you.%
}
This will only work on Windows Vista and up.
(Note that this is currently untested, since I do not have a Windows machine.
If this works for you, please let me know so that I can remove this disclaimer.)

\begin{listing}[htbp]
	\centering
	\begin{minted}{bat}
mkdir "%HOMEPATH%\Dropbox\Local TeX Files" & ^
chdir "%HOMEPATH%\Dropbox\Local TeX Files" & ^
FOR %d IN (doc, generic, scripts, source) DO mkdir /S %d & ^
FOR %d IN (bib, bst) DO mkdir /S bibtex\%d & ^
FOR %d IN (afm, map, misc, pk, source, tfm, type1) DO mkdir /S fonts\%d & ^
FOR %d IN (context, generic, latex, plain, xelatex, xetex) DO mkdir /S tex\%d & ^
FOR %d IN (bbx, cbx) DO mkdir /S tex\latex\biblatex\%d
	\end{minted}
	\caption{Make a TDS-compliant directory in Dropbox for \hologo{MiKTeX} on Windows}
	\label{lst:Windows-make-local-texmf-MiKTeX}
\end{listing}

After doing this, click Start → Programs → \hologo{MiKTeX} 2.9 → Maintenance → Settings to open the \hologo{MiKTeX} Options window.
Do \emph{not} open Settings (Admin), just open Settings.%
\footnote{%
For a discussion of the differences between administrative and user mode in \hologo{MiKTeX}, see this \href{http://tex.stackexchange.com/q/67712/32888}{question and its answers on TeX.SX}.%
}
Next, click on the Roots tab.
Click Add in order to add a local texmf folder.

Navigate to \Directory{C:\textbackslash Users\textbackslash <user name>\textbackslash Dropbox\textbackslash Local TeX Files} and click OK.

Next, click Apply.

Then, click on the General tab.
Click on Refresh FNDB.
Then click OK.

See also \href{http://docs.miktex.org/manual/localadditions.html#id584820}{here} and \href{http://docs.miktex.org/manual/configuring.html#fndbupdate}{here} for instructions with screenshots.

\subsubsection{Using local files}
\label{subsubsec:using-local-files}

As mentioned above, this local texmf folder that you just created is only for local files that you want to be available for use in all of your \File{.tex} documents.
For example, this texmf folder is where you would put local style or class files.

The most important use case for the local texmf folder, however, is for maintaining a single master bibliography file that can be used for citations in all of your documents.
This will be discussed in detail in \S\ref{subsec:bibliographies}.

To reiterate, do \emph{not} put any of your \File{.tex} files in this texmf folder.

% !TEX encoding = UTF-8 Unicode
% !TEX root = ../latex-workshop-for-linguists.tex

\section{General \LaTeX{} stuff}
\label{sec:general-latex-stuff}

\subsection{\LaTeX{} philosophy}
\label{subsec:latex-philosophy}

This subsection is very similar to \S3 of Alan Munn's \TitleText{A Beginner's Guide to \LaTeX{} (on the Mac)}.

\LaTeX{} was designed with the intent of separating content from formatting.
This is quite different from a what-you-see-is-what-you-get (WYSIWYG) editor like Word, where you see the output of your content formatted as you go along.

Something that goes hand in hand with separating content from formatting is that formatting should be given a semantics.
What does this mean?
(Bahhhh duhhhh chhhh!)
This means that if you want \LaTeX{} to format things that are similar in nature in the same way, then you should give them the same semantic meaning.

Alan gives the example of section headings.
The proper way to make a section heading in \LaTeX{} is to write it like \mintinline{latex}|\section{Section title}|.
The command \mintinline{latex}|\section{}| gives the content ``Section title'' the semantics of being a \Semantics{section}.
Then, if you want to change anything about how your sections are formatted, you can change this in the preamble of your document.
This contrasts with how many folks use Word where they would change this for each individual section heading.%
\footnote{%
Note that Word also allows for semantic markup despite the fact that most people do not use it.
If you cannot convince your Word-using friends and family to switch to \LaTeX, you should at least try to get them to use semantic markup in Word if they don't already do so.
\ASCIIEmoji{:)}%
}
For example, if you want the number preceding all of your section headings to be blue, you can change this at the beginning of the document, like in Listing~\ref{lst:blue-sections}.

\begin{listing}[htbp]
	\centering
	\begin{minted}{latex}
\documentclass{article}

\usepackage{color} % this package provides the command \color{}
\renewcommand\thesection{\color{blue}\arabic{section}}

\begin{document}

\section{Introduction}

Blah blah.

\section{Experiment}

Blah blah.

\section{Conclusion}

Blah blah.

\end{document}
	\end{minted}
	\caption{Example of semantic markup in \LaTeX{} for section headings}
	\label{lst:blue-sections}
\end{listing}

\subsection{Titles}
\label{subsec:titles}

To typeset a title, an author, and a date in a paper using the basic \Package{article} class, you can do what is shown in Listing~\ref{lst:basic-title}.

\begin{listing}[htbp]
	\centering
	\begin{minted}{latex}
\documentclass{article}

\title{Super awesome title}
\author{Best Author}
\date{July 22, 2015} % if you want today's date, replace July 22, 2015 with \today

\begin{document}

\maketitle

\section{Introduction}

Blah blah.

\end{document}
	\end{minted}
	\caption{A basic example of how to typeset the title of a paper using the \Package{article} class}
	\label{lst:basic-title}
\end{listing}

\subsection{Quotes and dashes}
\label{subsec:quotes-and-dashes}

One idiosyncrasy of \LaTeX{} that you will have to get used to is how to typeset quotes and dashes.

To typeset double open quotes, write \mintinline{latex}|``|.

To typeset double close quotes, write \mintinline{latex}|''|.

To typeset a single open quote, write \mintinline{latex}|`|.

To typeset a single close quote, write \mintinline{latex}|'|.

To typeset an en-dash, write \mintinline{latex}|--|.

To typeset an em-dash, write \mintinline{latex}|---|.

Note, however, that if you process your file with an engine that plays nicely with UTF-8 encoded documents (see \S\ref{subsec:fontspec-and-unicode}), you can enter these characters directly into your editor.

Doing this has the advantage of making your documents more readable, but note that it \emph{only} works if you use UTF-8 encoding and a compatible engine.
Also, it is good to know about the old way of typesetting these ligatures using \LaTeX{} since you will probably see many instances of this on the internet.

\subsection{Formatting text}
\label{subsec:formatting-text}

To typeset something in bold, use \mintinline{latex}|\textbf{}|.

To typeset something in italics, use \mintinline{latex}|\textit{}|.

To typeset something in small caps, use \mintinline{latex}|\textsc{}|.

To typeset something in a mono-spaced font, use \mintinline{latex}|\texttt{}|.

To underline something, don't.%
\footnote{%
Underlining is really frowned upon in the typography community.
I also personally do not like it.
However, if you have a really, really, really, really, really (really) good reason, then I suppose you can use \mintinline{latex}|\uline| from the \Package{ulem} package.
When you load, \Package{ulem} be sure to pass it the optional argument \mintinline{latex}|normalem|.%
}

One important thing to note about all of the foregoing commands is that using them directly in your document is \emph{not} good practice.
They are not semantic commands.
Let's consider an example.
Being a linguist, you will probably want to typeset certain glossed features in small caps, like \Nom, for instance.
You might think to write \mintinline{latex}|\textsc{nom}|.

This is \emph{bad practice}.
Instead, you should give glossed feature abbreviations like this a semantics, since you will presumably want to typeset them all the same way.
To do this, you could declare \mintinline{latex}|\newcommand*{\Fts}[1]{\textsc{#1}}| in your preamble.
Then, you would be able to write \mintinline{latex}|\Fts{nom}| instead of \mintinline{latex}|\textsc{nom}|.

A further example can be seen if you look at the source code for this handout.
You will notice that I typeset all the names of packages in a mono-spaced font.
Rather than writing, for example, \mintinline{latex}|\texttt{forest}|, I have written \mintinline{latex}|\Package{forest}|.
In my preamble, I defined the \mintinline{latex}|\Package{}| command in the following way: \mintinline{latex}|\newcommand*{\Package}[1]{\texttt{#1}}|.
If I ever wanted to change how the name of every package is typeset in this document, I would only need to change it once in my preamble.

\subsection{Footnotes}
\label{subsec:footnotes}

To typeset footnotes, use \mintinline{latex}|\footnote{}|.

\subsection{Special characters}
\label{subsec:special-characters}

There are several characters that are treated as special characters in \LaTeX.
These are \mintinline{text}|# $ % & ~ _ ^ \ { }|.

If you ever want to print any of these characters in the output, you need to escape them with `\texttt{\textbackslash}'.%
\footnote{%
One exception to this is the escape character itself, `\texttt{\textbackslash}', because the sequence `\mintinline{text}|\\|' has a special meaning in \LaTeX, used for line breaks in tables.
If you wish to render the character `\texttt{\textbackslash}' you can use the command \mintinline{latex}|\textbackslash|.
Two further exceptions are \mintinline{text}|^| and \mintinline{text}|~|.
Preceding these two characters with `\texttt{\textbackslash}' is used for appending diacritics in \LaTeX{} (though see \S\ref{subsec:fontspec-and-unicode} for a better way of doing this).
If you wish to print these characters, you will need to do \mintinline{latex}|\^{}| and \mintinline{latex}|\~{}|, respectively (though see \href{http://tex.stackexchange.com/q/312/32888}{this post on TeX.SX} for suggestions of better ways to typeset a tilde).
\label{fn:diacritics}%
}

The character `\mintinline{text}|#|' is used for passing arguments to macros.

The character `\mintinline{text}|$|' is used for entering math mode (see \S\ref{subsec:math-mode}).

The character `\mintinline{text}|%|' is used for writing comments in the source document (cf.~Listing~\ref{lst:blue-sections}).

The character `\mintinline{text}|&|' is used for separating columns in a table (see \S\ref{subsec:tables}).

The character `\mintinline{text}|~|' is a non-breaking space.

The character `\mintinline{text}|_|' is used for subscripts in math mode.

The character `\mintinline{text}|^|' is used for superscripts in math mode.

The character `\mintinline{text}|\|' is the escape character.

The characters `\mintinline{text}|{ }|' are used for delimiting the arguments to commands.

\subsection{Math mode}
\label{subsec:math-mode}

One thing worth knowing about \LaTeX{} is that it has a distinct mode for typesetting math, creatively called math mode.
There is inline math, triggered by \mintinline{latex}|$...$|, and display math, triggered by \mintinline{latex}|\[...\]|.

For linguists, math mode is something that is mostly useful for typesetting semantics.%
\footnote{%
This is a bit of an overstatement.
There are other use cases.
Typesetting feature bundles in morphosyntax is one such use case, for example.
However, for the purposes of an introductory workshop, you should really just know what math mode is.%
}
For example, \mintinline{latex}|$\lambda x$| will produce $\lambda x$.

\subsection{Tables}
\label{subsec:tables}

Tables are admittedly a bit of a pain in the ass in \LaTeX.
Typesetting them takes a while to get used to.
Let's see an example of a basic table, such as the one in Listing~\ref{lst:basic-table}.

\begin{listing}[htbp]
	\centering
	\begin{minted}{latex}
\documentclass{article}
\begin{document}

\begin{tabular}{lcr}
    Left-aligned column & Center-aligned column & Right-aligned column \\ \hline
    56\%                & 75\%                  & 34\%                 \\
\end{tabular}

\end{document}
	\end{minted}
	\caption{A basic table in \LaTeX}
	\label{lst:basic-table}
\end{listing}

\begin{minipage}{\textwidth}

The code in Listing~\ref{lst:basic-table} will produce the following table.

\begin{center}
	\begin{tabular}{lcr}
		Left-aligned column & Center-aligned column & Right-aligned column \\ \hline
		56\%                & 75\%                  & 34\%                 \\
	\end{tabular}
\end{center}

\end{minipage}

One useful package for making aesthetically pleasing tables is the package called \Package{booktabs}.
It provides commands called \mintinline{latex}|\toprule|, \mintinline{latex}|\bottomrule|, and \mintinline{latex}|\midrule| for nicer horizontal rules in a table.

\begin{minipage}{\textwidth}

Consider Listing~\ref{lst:booktabs-table}, which produces the following output.

\begin{center}
	\begin{tabular}{lll}
		\toprule
		         & Passive sentences & Active sentences \\ \midrule
		Adults   & 99\%              & 98\%             \\
		Children & 56\%              & 87\%             \\
		\bottomrule
	\end{tabular}
\end{center}

\end{minipage}

\begin{listing}[htbp]
	\centering
	\begin{minted}{latex}
\documentclass{article}
\usepackage{booktabs}
\begin{document}

\begin{tabular}{lll}
    \toprule
             & Passive sentences & Active sentences \\ \midrule
    Adults   & 99\%              & 98\%             \\
    Children & 56\%              & 87\%             \\
    \bottomrule
\end{tabular}

\end{document}
	\end{minted}
	\caption{A table in \LaTeX{} using the package \Package{booktabs}}
	\label{lst:booktabs-table}
\end{listing}

\subsection{Images}
\label{subsec:images}

One thing you will often want to do is include images in a document.
This is what the package \Package{graphicx} is for.
Let's look at the example in Listing~\ref{lst:image-example}.

\begin{listing}[htbp]
	\centering
	\begin{minted}{latex}
\documentclass{article}
\usepackage{graphicx}
\graphicspath{ {figure/} }
\begin{document}

\includegraphics[width=.8\textwidth]{super-important-graph}

\end{document}
	\end{minted}
	\caption{An example of including an image in a document}
	\label{lst:image-example}
\end{listing}

Notice that in the preamble of the document, we declared \mintinline{latex}|\graphicspath{ {figure/} }|.
This means that the package \Package{graphicx} will look in the folder called \texttt{figure} for images.%
\footnote{%
You don't need to do this.
If you prefer, you can just put the image file in the same folder as the \File{.tex} file.
The working directory (\IE{} the folder that the \File{.tex} file is in) is a place that the engine will always search when typesetting a document.
So if you're struggling with setting up a local \Directory{texmf} folder as described in \S\ref{subsec:local-files}, you can always just put files in the same folder as your \File{.tex} file for the time being.
But anyway, a reason you might want to have a separate folder dedicated for figures is to avoid clutter.
It's really up to you.%
}
So, in order to get this \File{.tex} file to compile, you would want to save it in a folder; then, in that same folder, you would want to create a new folder called \Directory{figure}.
Inside that folder, you should put the file \File{super-important-graph.pdf}.%
\footnote{%
The package \Package{graphicx} doesn't always play nicely with spaces and underscores in file names, so you should avoid using those things in the names of your image files.%
}
Notice that the file extension is omitted in the call to \mintinline{latex}|\includegraphics{}|.
This is a best practice because it allows \Package{graphicx} to try a bunch of different file extensions.

Notice also that we passed an optional argument to \mintinline{latex}|\includegraphics{}| in the form of a \textit{key val list}.
The key \KEY{width} can take a value that specifies what the width of the image that appears in the typeset document should be.
You could give it a value of \mintinline{latex}|6in| if you wanted, but it is often much more sensible to specify the width in terms of a dynamically defined value.
In this case, the super important graph will always occupy 80\% of the space allocated to the text, even if we change the margins of the document.

\subsection{Captioning and numbering}
\label{subsec:captioning-and-numbering}

Typesetting a table and including an image is great and all, but we want to be able to caption and number them.

\subsubsection{Floats}
\label{subsubsec:floats}

One common way to do this is to use floats.
In addition to automagically numbering tables and figures, floats also allow us to provide a caption.
One thing to know about floats is that \LaTeX{} has a special way of handling how they are typeset.
Suffice it to say, it's rather complicated.%
\footnote{%
You can read more about it \href{http://tex.stackexchange.com/q/39017/32888}{here} if you're interested.%
}
All you need to know is that \LaTeX{} has a special algorithm for placing floats in the best possible spot, according to general typographical standards.

These places are usually one of four places: right where they are written in the source document, the top of a page, the bottom of a page, or on their own separate page.
These four places correspond to four optional arguments that you can pass to a float environment, \mintinline{text}|htbp|, respectively.

It is generally best practice to pass all four options to a float, at least initially.
Only when you finish writing the document should you fiddle with the placement of floats if you think \LaTeX's algorithm has not done a good job.
However, while you're writing a document, leave all four options and let \LaTeX{} decide where floats should be placed.

If you would prefer to increase the likelihood that the float will show up in exactly the location that it is specified in the source \File{.tex} file, you can place a \mintinline{text}|!| after the \mintinline{text}|h|.

The command \mintinline{latex}|\caption{}| allows you to give a caption to the table or figure.
\LaTeX{} will automagically number the tables and figures in the correct order, so you don't have to worry about that.
Semantic markup FTW!

Take a look at the two examples in Listing~\ref{lst:float-examples}.
Try typesetting this yourself and see what the result is.

\begin{listing}[htbp]
	\centering
	\begin{minted}{latex}
\documentclass{article}
\usepackage{graphicx}
\usepackage{booktabs}
\begin{document}

This is a table that is a float.

\begin{table}[htbp]
    \centering
    \begin{tabular}{lll}
        \toprule
                 & Passive sentences & Active sentences \\ \midrule
        Adults   & 99\%              & 98\%             \\
        Children & 56\%              & 87\%             \\
        \bottomrule
    \end{tabular}
    \caption{Adult performance compared to child performance}
\end{table}

It might not actually show up in between these two sentences.

This is a figure that is a float.

\begin{figure}[htbp]
    \centering
    \includegraphics[width=.8\textwidth]{example-image-a}
    \caption{Super scientificy graphy thingy}
\end{figure}

It might not actually show up in between these two sentences.

\end{document}
	\end{minted}
	\caption{Examples of floats in \LaTeX}
	\label{lst:float-examples}
\end{listing}

\subsubsection{Non-float options}
\label{subsubsec:non-float-options}

It is worth mentioning some non-float options for tables and images.
One of the reasons that it is worth mentioning these options is because there is a common misconception that tables must go inside \mintinline{latex}|table| environments and images must go inside \mintinline{latex}|figure| environments.
This is not true.

When you're writing a paper, it is best to use floats because \LaTeX{} will use its algorithm to place the floats in the best environment.
However, it is not always appropriate to use floats.

One example of when you probably don't want to use floats is when you're making a handout.
In a handout, you usually want the image or the table to show up exactly where you place the code in the source document.

Nonetheless, in this case you might still want to be able to number and caption the figure or table.
The package \Package{capt-of} allows you to do this.
An example is given in Listing~\ref{lst:capt-of-example}.
Note that you will want to put things inside of an enclosing group, such as \mintinline{latex}|\begin{center}...\end{center}|.

\begin{listing}[htbp]
	\centering
	\begin{minted}{latex}
\documentclass{article}
\usepackage{graphicx}
\usepackage{booktabs}
\usepackage{capt-of}
\begin{document}

This is a table that is not a float.

\begin{center}
    \begin{tabular}{lll}
        \toprule
                 & Passive sentences & Active sentences \\ \midrule
        Adults   & 99\%              & 98\%             \\
        Children & 56\%              & 87\%             \\
        \bottomrule
    \end{tabular}
    \captionof{table}{Adult performance compared to child performance}
\end{center}

It will show up in between these two sentences no matter what.

This is a figure that is not a float.

\begin{center}
    \includegraphics[width=.8\textwidth]{example-image-a}
    \captionof{figure}{Super scientificy graphy thingy}
\end{center}

It will show up in between these two sentences no matter what.

\end{document}
	\end{minted}
	\caption{Examples of tables and images as non-floats}
	\label{lst:capt-of-example}
\end{listing}

Another case where you might want to not put a table inside of a float is when you're making a table for a particular morphological paradigm.
In linguistics, we usually number such tables just like we number other examples.
In \S\ref{subsec:examples}, we will see how to make numbered examples.
You can put a \mintinline{latex}|tabular| environment directly inside such an example.

\subsection{Cross referencing}
\label{subsec:cross-referencing}

So automagically numbered tables and images are great and all, but how do I refer to those things in my document?
One thing that is great about \LaTeX{} is that you can give things \mintinline{latex}|\label|s and \mintinline{latex}|\ref|er to them automagically as well.

Consider Listing~\ref{lst:cross-referencing}.

\begin{listing}[htbp]
	\centering
	\begin{minted}{latex}
\documentclass{article}
\usepackage{graphicx}
\begin{document}

As can be seen in Figure~\ref{fig:important-graph}, the results clearly show that I'm right.

\begin{figure}[htbp]
    \centering
    \includegraphics[width=.8\textwidth]{example-image-a}
    \caption{Super scientificy graphy thingy}
    \label{fig:important-graph}
\end{figure}

\end{document}
	\end{minted}
	\caption{An example of referencing a figure in \LaTeX}
	\label{lst:cross-referencing}
\end{listing}

There are a few things to say about cross-referencing.
First, and of particularly important note is the fact that the \mintinline{latex}|\label| comes after the \mintinline{latex}|\caption|.
If you try putting the \mintinline{latex}|\label| first, you will get the wrong number, because the command \mintinline{latex}|\caption| is what gives the float its number.

Second, you will notice that I've put a non-breaking space between \mintinline{text}|Figure| and \mintinline{latex}|\ref|.
This is generally a good practice because it prevents the number from being separated from the description of what it is.%
\footnote{%
If you really like automagic, you might want to check out the \href{http://ctan.org/pkg/cleveref}{\Package{cleveref}} package.%
}

Third, you will also notice that I've given the label a prefix of \mintinline{text}|fig:|.
This isn't strictly necessary, but it is good practice.
Imagine that you had a table and a figure.
The table contains specific values, and the figure is a graph of those values.
It's the same data, so you presumably want to give them similar names.
If you use prefixes like this, you could do \mintinline{latex}|\label{tab:super-important-results}| for the table and \mintinline{latex}|\label{fig:super-important-results}| for the graph.%
\footnote{%
Spaces are not allowed in the names of \mintinline{latex}|\label|s.%
}

Fourth, and most importantly, is that you must compile your document twice in order for this to work.
\LaTeX{} does its automagic by first processing the file and automagically generating all of the table and figure numbers for each float.
Remember that you never manually gave each float a number, so \LaTeX{} has to figure this out.
On the second compilation, it inserts the automagically generated numbers into the places where you \mintinline{latex}|\ref|erenced them.
If something went wrong, you will see question marks instead of a number.
This most likely means you either only compiled your document once or you have tried to refer to something that you never actually labeled.
The most common example of this latter reason is just a simple misspelling of the label that you gave to whatever it is that you're trying to reference.

\subsection{Those annoying files}
\label{subsec:those-annoying-files}

One thing you will quickly notice when you typeset a \File{.tex} file is that a lot of extra files are generated.
People tend to initially find this annoying, but it is all of these extra files that allow \LaTeX{} to do its magic.
For example, the auxiliary file (\File{.aux}) is integral for cross referencing.
Without it, cross referencing just would not work.

To avoid clutter, it's often a good idea to make a new, self-contained folder for each document that you typeset.

One other thing worth noting about all of these extra files is that they sometimes lead to compilation errors.
If you introduced an error in your document and you tried typesetting it, it's possible that the extra files got messed up.
So, if you tried typesetting your document, received a compilation error, figured out what caused the error, and you're like 110\% sure that you fixed the problem in your \File{.tex} file, but you're still getting a compilation error, try deleting all of these extra files and compiling the document again.

\subsection{Bibliographies}
\label{subsec:bibliographies}

Getting a bibliography to work with \LaTeX{} is often one of the big hurdles of learning \LaTeX, but once you've figured it out, it's really, really, really frikken nice.

In order for this to work correctly, you will first need to set up a local \Directory{texmf} folder.
How you do this depends on which \TeX{} distribution you have.
There are explicit instructions in \S\ref{subsec:local-files}.
If you have not already set up a local \Directory{texmf} folder on your machine, go there and follow the instructions.

\subsubsection{Overview}
\label{subsubsec:overview}

There are several components that are relevant to setting up your machine in order to be able to use a single master bibliography file (\File{.bib}) for all of your references in all of your \File{.tex} documents.

\begin{description}

	\item[The \File{.bib} file]{%
		You need to create a bibliographic database that contains all of the information for all of the references that you wish to cite in your \File{.tex} documents.
		This file is just a plain text file, but there are graphical user interfaces that can be used to edit the file, such as \href{http://jabref.sourceforge.net/}{JabRef} or \href{http://bibdesk.sourceforge.net/}{BibDesk}.%
		\footnote{%
		BibDesk is only available on a Mac.
		It is installed by default when you install {Mac\TeX}, so it is probably already on your computer.%
		}
		How to manage this file and other relevant aspects will be discussed in detail in \S\ref{subsubsec:the-bib-file}.%
	}
	
	\item[The citation package]{%
		Strictly speaking, you do not need to use a citation package.
		The base \hologo{LaTeX} format already provides the command \mintinline{latex}|\cite{...}|.
		However, citation packages provide much more flexibility and many more features than what is offered by the base \hologo{LaTeX} format.
		Citation packages are also great for easily changing the style that your references are printed in.
		They also often provide more specific commands for inline citations and parenthetical citations.
		Some examples of citation packages include \Package{cite}, \Package{apacite}, \Package{natbib}, and \Package{biblatex}.
		For the purposes of linguistics at least, I would only recommend \Package{natbib} or \Package{biblatex}.
		Both of these packages will be discussed in more detail in \S\ref{subsubsec:the-citation-package}.%
	}
	
	\item[The backend processor]{%
		In addition to using a citation package, you will also need to use a program for processing your \File{.aux} file.
		Since you only start with your \File{.tex} file, using a \File{.bib} file for references involves a series of compilation steps.
		You must first compile your \File{.tex} file to produce an \File{.aux} file.
		Then you must process your \File{.aux} file with one of the two backend processors: \hologo{BibTeX} or Biber.
		Doing this goes through and matches up the things that were cited in your \File{.tex} file with entries in your \File{.bib} file.
		Then you need to compile your \File{.tex} document again in order to get the references to show up.
		The differences between the two backend processors and the compilation steps that are necessary will be discussed in more detail in \S\ref{subsubsec:the-backend-processor}.%
	}

\end{description}

\subsubsubsection{A terminological note}
\label{subsubsubsec:a-terminological-note}

Before proceeding any further, it's worth touching briefly upon a bit of history to hopefully avoid some terminological confusion.
However, what is perhaps more likely, is that this brief excursion won't make any sense.
If that's the case, don't worry.
That's not your fault.
The history of \TeX{} and the evolution of relevant terminology is complicated.

Hopefully the terminology won't cause too much confusion, but I would nonetheless like to make note of it.
If it doesn't make sense now, perhaps it will in 6 months.

For a while, the only option for processing bibliography files was \hologo{BibTeX}.
\hologo{BibTeX} was developed in the ancient days when memory and hard drive space were expensive, character encodings were limited, and dinosaurs still roamed the earth.

Biblatex is a recent and modern alternative to \hologo{BibTeX}.
People generally use Biblatex to refer to the combination of the citation package \Package{biblatex} and the backend processor Biber.
I will adopt this practice of using Biblatex to refer to both of these things in combination, whereas \Package{biblatex} will only be used to refer to the citation package.

What makes this practice particularly confusing, is that many people use Biblatex in contradistinction to \hologo{BibTeX}.
The more accurate contrast, however, would be to contrast Biber with \hologo{BibTeX}.

For better or worse, however, this terminological parlance is quite common among \TeX{} users and is probably here to stay.
One thing that is actually useful about this seemingly confusing terminology, however, is that there are differences between how you write a \File{.bib} file that is intended to be used with Biblatex and how you write a \File{.bib} file that is intended to be used with \hologo{BibTeX}.
In other words, it can actually be useful to distinguish Biblatex and \hologo{BibTeX} if you want to know how a \File{.bib} file was prepared, for example.
Biblatex supports a wider range of entry types and a wider range of data fields and is thus much more flexible and versatile than \hologo{BibTeX}.%
\footnote{%
You will be introduced to entry types in \S\ref{subsubsubsec:entry-types} and data fields in \S\ref{subsubsubsec:data-fields}.%
}
Biblatex also plays much nicer with accented Latin characters and non-Latin alphabets than does \hologo{BibTeX}.

A full discussion of these differences is beyond the scope of this document.
Some of the differences will be briefly discussed below, but for further discussion, the reader is referred to the following resources.

\begin{itemize}

	\item{The question \href{http://tex.stackexchange.com/q/25701/32888}{bibtex vs.~biber and biblatex vs.~natbib} and its answers on TeX.SX}
	
	\item{The question \href{http://tex.stackexchange.com/q/5091/32888}{What to do to switch to biblatex?} and its answers on TeX.SX}
	
	\item{\S\S1--3 of the \href{http://texdoc.net/texmf-dist/doc/latex/biblatex/biblatex.pdf}{\Package{biblatex} documentation}}

\end{itemize}

\subsubsection{The \File{.bib} file}
\label{subsubsec:the-bib-file}

In order to get started, you need to create a \File{.bib} file.
This is just a plain text file, but your life will probably be made much easier if you use a graphical user interface to edit the file.

\href{http://jabref.sourceforge.net/}{JabRef} is a crossplatform option.
Follow the link to download and install it.

If you have a Mac, you can use \href{http://bibdesk.sourceforge.net/}{BibDesk}.
It should already be installed if you installed {Mac\TeX}.

Another thing that is nice about these graphical user interfaces is that they allow you to pair \File{.pdf} files to the entries in your database.
If you pair a \File{.pdf} with an entry, then you can open JabRef or BibDesk and click on the link to the \File{.pdf} file in order to open it without having to find it on your computer.

Before creating a database, you will want to make sure that you save the \File{.bib} file with UTF-8 encoding.
For some reason, the default character encoding in JabRef (on Windows at least) is Cp1252.
This will just give you a lot of headaches.
Before you create your \File{.bib} file, go into the JabRef preferences and change the default character encoding to UTF-8.

You should do the same with BibDesk.
Before creating your \File{.bib} file, open BibDesk, go to preferences and set the default character encoding to UTF-8.

If you have trouble finding these options in either BibDesk or JabRef, \href{https://xkcd.com/627/}{Google it}. \ASCIIEmoji{:p}

Once you have changed the default encoding, create a new bibliography file (\File{.bib}).
It is \emph{very important} that you save it in the correct location.
It \emph{must} go inside your local \Directory{texmf} folder; specifically, it \emph{must} go inside \Directory{texmf/bibtex/bib}.
If you do not have a local \Directory{texmf} folder set up, go read and follow the instructions in \S\ref{subsec:local-files}.

Moreover, you will also save yourself a lot of pain if you do not use spaces in the name of the file.
I would recommend something simple like \File{master.bib} or \File{linguistics.bib}.

Now that you have your \File{.bib} file setup, you can start adding entries to it.

\subsubsubsection{Entry types}
\label{subsubsubsec:entry-types}

There are various entry types, and the available entry types differ slightly depending on whether you use \Package{biblatex} or not.%
\footnote{%
\label{fn:entry-types-biblatex-vs-bibtex}
Covering this in any detail is really beyond the scope of this document.
For more information, please see the resources mentioned in \S\ref{subsubsubsec:a-terminological-note}.
For the most part, however, you do not need to worry about it.
A \File{.bib} file that was prepared for use with \hologo{BibTeX} will work just fine with Biblatex.
You just will not be able to use some of the more flexible and versatile features of Biblatex (which you probably won't need anyway, except for complicated use cases).%
}
In general, you will probably mostly only need to use the entry types \EntryType{@article}, \EntryType{@book}, \EntryType{@incollection}, \EntryType{@inproceedings}, \EntryType{@misc}, and \EntryType{@unpublished}.

\begin{description}

	\item[\EntryType{@article}]{%
		This entry type should be used for journal articles.%
	}
	
	\item[\EntryType{@book}]{%
		This entry type should be used for complete books.%
	}
	
	\item[\EntryType{@incollection}]{%
		This entry type should be used for chapters in edited volumes.
		Do \emph{not} use the \EntryType{@inbook} entry type for this.%
	}
	
	\item[\EntryType{@inproceedings}]{%
		This entry type should be used for stuff that is published in conference proceedings.%
	}
	
	\item[\EntryType{@misc}]{%
		This entry type should be used for miscellaneous stuff, such as talks or posters presented at conferences (that aren't published in proceedings).
		You can add something to the \DataField{howpublished} field that indicates where it was presented and whether it was a talk or a poster.
		See \S\ref{subsubsubsec:data-fields} for more details.%
	}
	
	\item[\EntryType{@unpublished}]{%
		This entry type can be used for unpublished manuscripts.%
	}

\end{description}

\subsubsubsection{Data fields}
\label{subsubsubsec:data-fields}

Inside of each entry, there are certain data fields that you need to fill out.
The available data fields depend on which entry type you are using.%
\footnote{%
The available data fields also depend on whether you are preparing your \File{.bib} file for use with Biblatex or \hologo{BibTeX} (cf.~fn.~\ref{fn:entry-types-biblatex-vs-bibtex}).
A full discussion of the many more data fields that Biblatex provides is much beyond the scope of this document.
You largely do not need to worry about it unless you would like to use some of Biblatex's more advanced features.
The data fields discussed here are for a \hologo{BibTeX} \File{.bib} file, but they will all work just fine if you use Biblatex instead of \hologo{BibTeX}.%
}
In general, the bare minimum that you need to fill in are the data fields \DataField{author}, \DataField{title}, and \DataField{year}.

For each of the entry types mentioned above, here are the recommended data fields that I would minimally suggest filling in.

\begin{multicols}{3}
\begin{itemize}

	\item{\EntryType{@article}
		
		\begin{itemize}
		
			\item{\DataField{author}}
			
			\item{\DataField{title}}
			
			\item{\DataField{journal}}
			
			\item{\DataField{year}}
			
			\item{\DataField{volume}}
			
			\item{\DataField{number}}
			
			\item{\DataField{pages}}
			
			\item{\DataField{doi}}
		
		\end{itemize}
		
	}
	
	\item{\EntryType{@book}
	
		\begin{itemize}
		
			\item{\DataField{author}}
			
			\item{\DataField{title}}
			
			\item{\DataField{publisher}}
			
			\item{\DataField{year}}
			
			\item{\DataField{address}}
		
		\end{itemize}
	
	}
	
	\item{\EntryType{@incollection}
	
		\begin{itemize}
		
			\item{\DataField{author}}
			
			\item{\DataField{title}}
			
			\item{\DataField{booktitle}}
			
			\item{\DataField{publisher}}
			
			\item{\DataField{year}}
			
			\item{\DataField{editor}}
			
			\item{\DataField{pages}}
			
			\item{\DataField{address}}
		
		\end{itemize}
	
	}
	
	\item{\EntryType{@inproceedings}
	
		\begin{itemize}
		
			\item{\DataField{author}}
			
			\item{\DataField{title}}
			
			\item{\DataField{booktitle}}
			
			\item{\DataField{publisher}}
			
			\item{\DataField{year}}
			
			\item{\DataField{pages}}
			
			\item{\DataField{address}}
		
		\end{itemize}
	
	}
	
	\item{\EntryType{@misc}
	
		\begin{itemize}

			\item{\DataField{author}}

			\item{\DataField{title}}
	
			\item{\DataField{howpublished}}

			\item{\DataField{year}}

		\end{itemize}
	
	}
	
	\item{\EntryType{@unpublished}
	
		\begin{itemize}
		
			\item{\DataField{author}}
			
			\item{\DataField{note}}
			
			\item{\DataField{title}}
			
			\item{\DataField{year}}
		
		\end{itemize}

	}

\end{itemize}
\end{multicols}

\paragraph{Keywords}

Another useful data field is \DataField{keywords}.
This can be used with every entry type, and I would highly recommend filling it in for every entry.

You can make up the keywords that you use.
Doing so allows you to easily sort and search through your \File{.bib} file.

For example, you could put keywords in all of your entries for different subdisciplines, like syntax, or semantics, or phonology, \ETC.
If you wanted to look through your \File{.bib} file for a certain syntax paper that you cannot remember, for example, then you could just look at all of the entries that contain syntax in the \DataField{keywords} data field.

\paragraph{On names}

The method for entering names into the \DataField{author} and \DataField{editor} data fields is a bit idiosyncratic, but it allows names to be typeset in a very fine-grained manner, so it is well worth it.

There are four components to a name.

\begin{enumerate}[label={(\roman*)}]

	\item{First name (this includes any and all middle names)}
	
	\item{The ``von'' part (examples include ``von'', ``van'', ``de'', ``de la'', \ETC)}
	
	\item{The last name (does \emph{not} include the ``von'' part)}
	
	\item{The ``Jr'' part (examples include ``Jr.'', ``III'', \ETC)}

\end{enumerate}

There are three possible ways to enter names into a \File{.bib} database.

\begin{enumerate}[label={(\roman*)}]

	\item{``First von Last''}
	
	\item{``von Last, First''}
	
	\item{``von Last, Jr, First''}

\end{enumerate}

I would highly recommend always using either the second or third option.
The first option does not work if an author has multiple \emph{last} names and no ``von'' part in their name.

Finally, the last thing to know about names is that multiple authors or editors should be separated with \BibliographyData{and}.

So, for example, if something you wish to cite has three authors, you should write the following in the \DataField{author} data field: \BibliographyData{Matthewson, Lisa and von Fintel, Kai and Smith, Jr., Mary}.

\paragraph{On casing}

Depending on the style that you use for your bibliography, you might see your titles being typeset in \href{https://en.wiktionary.org/wiki/sentence_case}{sentence casing} rather than \href{https://en.wiktionary.org/wiki/title_case}{title casing}.
This is what's called for by the \href{http://celxj.org/downloads/UnifiedStyleSheet.pdf}{Unified Stylesheet for Linguistics Journals}, for example.

Nonetheless, there might be certain things that ought to remain capitalized.
In this case, you should surround these things with braces in the data field.

For example, if you are adding an \EntryType{@incollection} type whose \DataField{title} is \TitleText{On the absence of certain quantifiers in Mohawk}, you should enter it into the \DataField{title} field as \BibliographyData{On the Absence of Certain Quantifiers in \{Mohawk\}}.
Note that the braces should surround the entire word, \emph{not} just the letter M.
If the braces do not surround the entire word, it will mess up the kerning.

Note furthermore that the data field was written using title casing, but the output you see will be sentence casing, except for the word \xv{Mohawk}, because we surrounded it with braces.
I would recommend always entering titles into the data fields using title casing.
It is much easier to automagically convert title casing to sentence casing than it is to automagically convert sentence casing to title casing.
Thus, you will save yourself much pain if you need to switch to a style that uses title casing.

\paragraph{On cite keys}

In addition to filling out all of the data fields, you will also need to give each entry a unique cite key.
This is what you will use to cite the entry in one of your \File{.tex} documents.

You can come up with your own algorithm for determining cite keys, but it is best to use some sort of algorithm rather than making up arbitrary cite keys as you go along.
Here are two suggestions.

\begin{enumerate}[label={(\roman*)}]

	\item{\BibliographyData{lastnameYYYY}}
	
	\item{\BibliographyData{lastnameYYYY:informativewordfromtitle}}

\end{enumerate}

The first one is very basic.
You could just use the last name of the first author and the four-digit year of the publication for the cite key.
You will of course run into trouble when an author has multiple publications in the same year.
In this case, you could do something like \BibliographyData{lastnameYYYYa} and \BibliographyData{lastnameYYYYb} to disambiguate the two publications.

However, it might be preferable to do something like the second option instead.
If you have to change the cite key after you've already added it and used it, then you would need to go back and change how you cited it in your \File{.tex} file.
If you use the second suggested algorithm instead, it's very unlikely that you will run into a situation where you might have duplicate cite keys for distinct entries.

\subsubsubsection{Lingbib}
\label{subsubsubsec:lingbib}

However, instead of going through all the trouble of maintaining your own bibliography file \ldots{} why not use and contribute to \href{https://github.com/lingbib/lingbib}{Lingbib}!?

This is a (shameless plug for a) project that Kenneth Hanson and I are currently working on developing.
We think it's rather wasteful that people individually create bibliography files and maintain them on their own.
Instead, we could all use and contribute to one central bibliography file for the entire field.

This has at least the following two advantages. First, it saves everyone time, and, second, it ensures greater accuracy in the \File{.bib} file since more eyes will be looking at it.

The project isn't quite ready for use yet, but I will update this \File{.pdf} when it is ready to use.
We would greatly appreciate it if you spread the word about this project, started using it, and perhaps even contributed! \ASCIIEmoji{:)}

\subsubsection{The citation package}
\label{subsubsec:the-citation-package}

The two main options for citation packages that are recommended are \Package{natbib} and \Package{biblatex}.
There are others, but they are nowhere near as versatile as these two.

\subsubsubsection{\Package{natbib}}
\label{subsubsubsec:natbib}

\Package{natbib} is only compatible with using \hologo{BibTeX} as the backend processor for your \File{.aux} and \File{.bib} files (see \S\ref{subsubsec:the-backend-processor}).
Since \hologo{BibTeX} was developed a long time ago, it does not play all that nicely with accented Latin characters and non-Latin alphabets.

I would highly recommend using \Package{biblatex} and Biber instead of \Package{natbib} and \hologo{BibTeX}.
However, it is worth noting that academic journals that accept \LaTeX{} submissions often require bibliographies that use \Package{natbib} and \hologo{BibTeX}.%
\footnote{%
A notable exception to this is \href{http://semprag.org/}{\Journal{Semantics \& Pragmatics}}, which accepts submissions that use \Package{biblatex} and Biber.%
}

\paragraph{Citation commands}

There are two main commands for citations with \Package{natbib} that you ought to know about.
You can use \mintinline{latex}|\citet[]{...}| for inline citations and \mintinline{latex}|\citep[][]{...}| for parenthetical citations.

The first optional argument of \mintinline{latex}|\citet| can be used to specify page numbers.

The first optional argument of \mintinline{latex}|\citep| can be used to specify a prenote, and the second optional argument can be used to specify page numbers.

There is also a handy \href{http://merkel.zoneo.net/Latex/natbib.php}{\Package{natbib} reference sheet} that describes more of the citation commands.

\paragraph{Styles}

\Package{natbib} and \hologo{BibTeX} use \File{.bst} files to specify different bibliography styles.
\File{.bst} files are awful.

Thankfully, Bridget Samuels has gone through the trouble of creating one for us that conforms to the \href{http://celxj.org/downloads/UnifiedStyleSheet.pdf}{Unified Stylesheet for Linguistics Journals}.
If you decide to use \Package{natbib} and \hologo{BibTeX}, you should download the file \href{http://celxj.org/downloads/unified.bst}{\File{unified.bst}} file and place it in your local \Directory{texmf} directory.
Specifically, it needs to go into the folder \Directory{texmf/bibtex/bst} (see \S\ref{subsec:local-files} if you've somehow made it this far without setting up a local \Directory{texmf} folder).

There are many other styles that you can use out of the box with either a TeX Live/{Mac\TeX} or \hologo{MiKTeX} distribution, such as \Package{alpha} or \Package{apalike}.
But, presumably, you will want to use the \Package{unified} style that Bridget Samuels wrote since you're (presumably) a linguist.

\paragraph{Complete minimal working example}

Listing~\ref{lst:natbib-example} gives a complete minimal working example of how to use \Package{natbib} and \hologo{BibTeX}.
In order to figure out how to compile it correctly, read \S\ref{subsubsubsec:bibtex}.

\begin{listing}[htbp]
	\centering
	\begin{minted}{latex}
\documentclass{article}

\begin{filecontents}{\jobname.bib}
@book{chomsky1995:MP,
    Address = {Cambridge, MA},
    Author = {Chomsky, Noam},
    Publisher = {The MIT Press},
    Title = {The Minimalist Program},
    Year = {1995}}
\end{filecontents}

\usepackage{natbib}

\begin{document}

In contemporary Minimalist syntax, S-Structure has been eliminated \citep[see][]{chomsky1995:MP}.

\bibliography{\jobname}
% the following will only work if unified.bst is in your texmf folder
\bibliographystyle{unified}
\end{document}
	\end{minted}
	\caption{Complete minimal working example showing how to use \Package{natbib} and \hologo{BibTeX}}
	\label{lst:natbib-example}
\end{listing}

Before moving on to \Package{biblatex}, there are a few things to note.

First, the bibliography file and the style file are specified at the end of the document, just before \mintinline{latex}|\end{document}|.
In both cases, be sure to omit the file extensions (\IE{} don't add \File{.bib} or \File{.bst}).

Second, the \File{.bib} file is bundled into this example so that you can copy and paste it into an editor and compile it directly.
But this is \emph{not} how you should write your \File{.tex} files.
You should instead use a single master bibliography file as discussed in \S\ref{subsubsec:the-bib-file}.
In this case, you would replace \mintinline{latex}|\bibliography{\jobname}| with \mintinline{latex}|\bibliography{master}|, if you named your \File{.bib} file \File{master.bib}, for example.
Moreover, you can completely omit everything from \mintinline{latex}|\begin{filecontents}| to \mintinline{latex}|\end{filecontents}|.

\subsubsubsection{\Package{biblatex}}
\label{subsubsubsec:biblatex}

Unlike \Package{natbib}, \Package{biblatex} can be used with either backend processor: Biber or \hologo{BibTeX}.
If you wish to use the full range of features that \Package{biblatex} provides, you will need to use Biber.

As mentioned above, there is good reason to use \Package{biblatex} and Biber over \Package{natbib} and \hologo{BibTeX}.
Most notably, \Package{biblatex} and Biber work a lot more straightforwardly with accented characters and non-Latin alphabets.

However, in addition to not being an acceptable format for submission to many academic journals, the available implementations of the \href{http://celxj.org/downloads/UnifiedStyleSheet.pdf}{Unified Stylesheet for Linguistics Journals} in \Package{biblatex} are currently lacking.
More on this below.

\paragraph{Citation commands}



\paragraph{Styles}



\paragraph{Complete minimal working example}



\subsubsection{The backend processor}
\label{subsubsec:the-backend-processor}

\subsubsubsection{\hologo{BibTeX}}
\label{subsubsubsec:bibtex}



\subsubsubsection{Biber}
\label{subsubsubsec:biber}



\subsection{Paragraphs}
\label{subsec:paragraphs}

Let's end this section with paragraphs.
Let's do this for two reasons.
First, because semantic markup is awesome.
And second, because a very common \emph{really bad practice} of \LaTeX{} beginners is to insert line breaks all over the place.

Recall from \S\ref{subsec:latex-philosophy} that \LaTeX{} is all about semantic markup.
This goes for paragraphs, too.
People who are used to Word are used to pressing ENTER on the keyboard once in order to separate paragraphs.
Perhaps unsurprisingly then, many new \LaTeX{} users will often do stuff like what is shown in Listing~\ref{lst:bad-practice-with-line-breaks}.

\begin{listing}[htbp]
	\centering
	\begin{minted}{latex}
\documentclass{article}
\begin{document}
This is my first awesome paragraph.\\
This is my second paragraph, which is infinitely less awesome because of the line break.
\end{document}
	\end{minted}
	\caption{Really bad practice for separating paragraphs}
	\label{lst:bad-practice-with-line-breaks}
\end{listing}

The command \mintinline{latex}|\\| does a line break, but it does \emph{not} introduce a new paragraph.
In other words, the (first part of the) second sentence is false.
In \LaTeX{}'s eyes, the sentence ``This is my first awesome paragraph.'' is in the \emph{same paragraph} as the sentence ``This is my second paragraph, which is infinitely less awesome because of the line break.''.

Instead, one should use a command for paragraphs so that we can manipulate the semantics of paragraphs in the preamble of the document in the same way that we manipulated the semantics of sections.
The command for separating one paragraph from another paragraph is \mintinline{latex}|\par|.
This would get really annoying to type in between all of your paragraphs, so, luckily, \LaTeX{} treats an empty line as equivalent to \mintinline{latex}|\par|.

Listing~\ref{lst:good-practice-for-paragraphs} exemplifies \emph{good practice} for typesetting paragraphs.
This good practice allows us to manipulate the semantics of paragraphs in the preamble, so we can typeset them as we like, without having to modify each individual paragraph.

There's generally no need to modify the default semantics for paragraphs.
However, in Listing~\ref{lst:good-practice-for-paragraphs}, I've given the same semantics that are used to typeset the paragraphs in this document, which are intended to be much more handout-y and much less essay-y.
Try typesetting Listing~\ref{lst:good-practice-for-paragraphs} yourself and see what happens.

\begin{listing}[htbp]
	\centering
	\begin{minted}{latex}
\documentclass{article}
\setlength{\parindent}{0em}
\setlength{\parskip}{1ex}
\begin{document}

This is the first paragraph.
Wasn't that a great topic sentence?

Next paragraph please.
Paragraph number two is the best.
\par
The third paragraph will rule them all.
Sorry about the Lord of the Rings reference.

This concludes my five paragraph essay. As you can see, my conclusion definitely follows.

Yes I can count.
Jeeze.

\end{document}
	\end{minted}
	\caption{Good practice for typesetting paragraphs}
	\label{lst:good-practice-for-paragraphs}
\end{listing}

Lastly, one thing you might notice is that I sometimes put sentences on their own line.
\LaTeX{} treats these sentences as being in the same paragraph because there is no blank line or \mintinline{latex}|\par| between them.
This isn't strictly necessary; you're more than welcome to put all of the sentences in a paragraph on one line, like I did in paragraph `five'.

However, there are two main reasons why it might be nice to put each sentence on its own line.
First, having shorter lines might be easier to read, depending on how your editor is set up.
Second, if you keep your \File{.tex} file under version control, it makes for cleaner diffs.%
\footnote{%
Explaining what exactly a version control system (VCS) is, is quite beyond the scope of this workshop.
Basically, it's a way to keep a history of all of the changes that have been made to a document.
If you delve any further into learning \LaTeX{} or learning to program, you will probably also want to learn a VCS at some point.
The most popular one is \mintinline{sh}|git|.
Also, hopefully you now understand this \href{https://xkcd.com/1285/}{joke}.%
}

% !TEX encoding = UTF-8 Unicode
% !TEX root = ../latex-workshop-for-linguists.tex

\section{Useful stuff for linguists}
\label{sec:useful-stuff-for-linguists}

\subsection{\Package{fontspec} and Unicode}
\label{subsec:fontspec-and-unicode}

In fn.~\ref{fn:diacritics}, I mentioned a better way of typesetting diacritics in \LaTeX.
This is where the \Package{fontspec} package comes in.
Historically, using \LaTeX{} with fonts hasn't really been a thing.
This has changed recently with the advent of two other engines---\hologo{XeLaTeX} and \hologo{LuaLaTeX}---and the \Package{fontspec} package.

If you process a document with one of these two engines, you can use the \Package{fontspec} package to specify which font you want to use.
You can use any font that is installed on your computer.

In \S6 of \TitleText{A Beginner's Guide to \LaTeX{} (on the Mac)}, Alan gives the example of declaring a new font family to use for phonetic fonts.
A complete example of what Alan suggests is given in Listing~\ref{lst:phonetic-font}.%
\footnote{%
See also \href{http://tex.stackexchange.com/q/25249/32888}{this question and its answers on TeX.SX}.
Note that in order for the example in Listing~\ref{lst:phonetic-font} to actually compile, you will need to have the font Doulos SIL installed on your machine.
Covering how to install a font on your machine is beyond the scope of this workshop, since it is very specific to the type of operating system that you are running.
However, you should be able to search for instructions online and pretty easily figure out how to do it.%
}

\begin{listing}[htbp]
	\centering
	\begin{minted}{latex}
\documentclass{article}
\usepackage{fontspec}
\setmainfont[Ligatures=TeX]{Times New Roman}
\newfontfamily\phonetic[]{Doulos SIL}
\usepackage{textglos} % good semantic markup for inline examples, like \xv{...}, \xm{...}, etc.
\begin{document}

The English word \xv{cat} is underlyingly {\phonetic\xm{kæt}}.

\end{document}
	\end{minted}
	\caption{Example of using a distinct font for phonetics}
	\label{lst:phonetic-font}
\end{listing}

In addition to using a separate font for phonetic stuff, you can also use one font for the entire document if the font you are using has glyphs for all of the characters that you need.
I actually really like the Computer Modern font that is the default font in \TeX.
There is a version of the Computer Modern font that you can \href{http://sourceforge.net/projects/cm-unicode/}{download} and install on your machine which has glyphs for a huge range of the Unicode characters.

If you download and install this font on your machine, then you can do something like in Listing~\ref{lst:one-font}, rather than having to use a separate font for special (phonetic) characters.

\begin{listing}[htbp]
	\centering
	\begin{minted}{latex}
\documentclass{article}
\usepackage{fontspec}
\setmainfont[Ligatures=TeX]{CMU Serif Roman}
\usepackage{textglos} % good semantic markup for inline examples, like \xv{...}, \xm{...}, etc.
\begin{document}

The English word \xv{cat} is underlyingly \xm{kæt}.
Also, look at the cool stuff that I can do in the same font: ášçëû!

\end{document}
	\end{minted}
	\caption{Using one font that has a lot of Unicode glyphs}
	\label{lst:one-font}
\end{listing}

Processing either Listing~\ref{lst:phonetic-font}~or~\ref{lst:one-font} with either \hologo{LuaLaTeX} or \hologo{XeLaTeX} will produce a PDF with the correct glyphs (as long as you have the requisite fonts installed on your computer).
One further thing that is \emph{very important} for this to work correctly is to make sure that your \File{.tex} file has the correct character encoding.
It is best to make sure that all of your \File{.tex} files are saved with UTF-8 encoding.
A good editor should allow to see and change the character encoding of the file.
Since this depends on the editor, it is beyond the scope of the workshop to explain it in any more detail, but you should be able to search online and figure it out.

\subsection{Examples}
\label{subsec:examples}

There are two main packages that I would recommend for typesetting linguistic examples, \Package{gb4e} and \Package{ExPex}.%
\footnote{%
There is also the package called \Package{linguex}.
I don't know much about it.
I've always avoided it because, as Alan points out in \TitleText{A Beginner's Guide to \LaTeX{} (on the Mac)}, its markup is not all that semantic.%
}
The \Package{gb4e} package works well in most cases.
For more complicated use cases, you might want to learn \Package{ExPex}.
However, this workshop will only focus on \Package{gb4e}.

\subsubsection{Basic linguistic example}
\label{subsubsec:basic-linguistic-example}

Listing~\ref{lst:basic-linguistic-example} gives an example of how to typeset some basic examples.%
\footnote{%
Note that it is not strictly necessary to put each example in its own \mintinline{latex}|exe| environment, but it is a good practice to do so, for at least two reasons.
First, the markup is more semantic because \mintinline{latex}|exe| is singular; it refers to \emph{an} example, not a series of examples.
Second, and more practically, it makes moving examples around a lot easier, either within the document or from one document to another.%
}
Try typesetting these examples yourself and see what the result is.
Notice in particular that you can give the examples \mintinline{latex}|\label|s and \mintinline{latex}|\ref|er to them inline just like with captioned things.

\begin{listing}[htbp]
	\centering
	\begin{minted}{latex}
\documentclass{article}

\usepackage{textglos} % good semantic markup for inline examples, like \xv{...}, \xm{...}, etc.

%\usepackage{fixltx2e} % only needed if you have TeX Live < 2015
\newcommand*{\IND}[1]{\textsubscript{#1}}

\usepackage{gb4e}
\noautomath % you should always declare this after loading gb4e

\begin{document}
(\ref{ex:questionable-English}) is marginally acceptable.

\begin{exe}
    \ex[?]{His\IND{i} mother loves every\IND{i} boy no matter what.}
    \label{ex:questionable-English}
\end{exe}
\begin{exe}
    \ex[]{Strong crossover
        \begin{xlist}
            \ex[*]{He\IND{i} loves everyone\IND{i}}
            \ex[*]{She\IND{i} thinks everyone\IND{i} is smart}
            \label{ex:everyone-is-smart}
        \end{xlist}
    }
    \label{ex:strong-crossover}
\end{exe}

The examples in (\ref{ex:strong-crossover}) exemplify the phenomenon of strong crossover.
For example, in (\ref{ex:everyone-is-smart}), \xv{she} c-commands \xv{everyone}.
However, pronouns cannot c-command their binders.

\end{document}
	\end{minted}
	\caption{Typesetting basic linguistics examples}
	\label{lst:basic-linguistic-example}
\end{listing}

\subsubsection{Glossing examples}
\label{subsubsec:glossing-examples}

With \Package{gb4e}, you can also gloss examples.
An example is shown in Listing~\ref{lst:glossed-example}.%
\footnote{%
I'm a bit loathe to recommend this because the package currently has a bug, but it's a really great package.
For an even better way of typesetting common linguistic gloss abbreviations than what you see in Listing~\ref{lst:glossed-example}, check out the \href{http://ctan.org/pkg/leipzig}{\Package{leipzig}} package.
If you use \Package{leipzig} \emph{without} the \Package{glossaries} package, you shouldn't run into any trouble.
However, if you use \Package{leipzig} in conjunction with the \Package{glossaries} package---which is a really great thing to do because it can then automagically generate a list of all glosses that you've used in your document---you will run into problems.
There is a really hacky workaround \href{http://tex.stackexchange.com/a/204615/32888}{here}, but it's a bug that should ultimately be fixed in the \Package{leipzig} package.
I've tried contacting the maintainer of the package, but I haven't gotten a response.
I plan to try contacting her again soon, so hopefully the bug will be fixed at some point.%
}

\begin{listing}[htbp]
	\centering
	\begin{minted}{latex}
\documentclass{article}
\newcommand*{\Fts}[1]{\textsc{#1}}
\usepackage{gb4e}
\noautomath
\begin{document}
\begin{exe}
    \ex[]{\gll Der Apfel würde gegessen.\\
    The.\Fts{m}.\Fts{sg}.\Fts{nom} apple was eaten\\
    \trans `The apple was eaten'
    }
\end{exe}
\end{document}
	\end{minted}
	\caption{A glossed example with \Package{gb4e}}
	\label{lst:glossed-example}
\end{listing}

Try typesetting (\ref{ex:try-glossed-example}) yourself, giving it a \mintinline{latex}|\label|, and \mintinline{latex}|\ref|erring to it in the text of your document. 

\begin{exe}
	%
	\ex[]{\gll Hasan geu-peu-reubah aneuk miet nyan\\
	Hasan \Third\Pol-\Caus-fall child small \Dem\\
	\trans `Hasan caused the child to fall'
	}
	\label{ex:try-glossed-example}
	%
\end{exe}

\subsection{Typesetting trees}
\label{subsec:typesetting-trees}

There are two main packages that are useful for typesetting linguistics trees, \Package{tikz-qtree} and \Package{forest}.
Both are built on top of the package, \Package{tikz}.
I would recommend using \Package{forest}.%
\footnote{%
For further discussion, see \href{http://tex.stackexchange.com/q/5447/32888}{this question and its answers on TeX.SX}.%
}
The syntax for both of these is almost the same, and \Package{forest} is a lot more powerful.
However, you don't need to know its internals to do the basic stuff.

Let's start with a very basic example, given in Listing~\ref{lst:basic-forest-example}, which will produce what you see in (\ref{ex:basic-forest-example}).

Try typesetting this yourself.

\begin{exe}
	\ex[]{
		\begin{forest}
		[VP
			[V]
			[DP]
		]
		\end{forest}
	}
	\label{ex:basic-forest-example}
\end{exe}

\begin{listing}[htbp]
	\centering
	\begin{minted}{latex}
\documentclass{article}
\usepackage{forest}
\begin{document}

\begin{forest}
[VP
    [V]
    [DP]
]
\end{forest}

\end{document}
	\end{minted}
	\caption{A very basic example with \Package{forest}}
	\label{lst:basic-forest-example}
\end{listing}

There are a few things to note about this basic example.

First, line breaks are not necessary.
You could have produced the same output by writing \mintinline{latex}|[VP [V] [DP] ]|.
Nonetheless, spacing things across lines is generally a good practice because it makes your code much more readable, which also in turn makes it much easier to debug if you are getting errors when trying to compile your file.

Second, you will see that this tree does not occur inside of a numbered example.
This is something you will almost always want to do, as it is a standard in the field.
Given what you learned in \S\ref{subsec:examples}, you should be able to imagine how to do this.
And we will see an explicit example of how to do this below.

Third, you will notice that the branches do not actually connect at the bottom of the ``VP'' node.
Having the branches connect at the bottom of their parent node is the standard style in syntax.
We will see how to draw a tree with this style below as well.

With that in mind, let's take a look at a slightly more complicated example, given in Listing~\ref{lst:typesetting-trees}, which will address some of these issues.
The code in Listing~\ref{lst:typesetting-trees} will produce what you see in (\ref{ex:typesetting-trees}).

\begin{exe}
    \ex[]{
        \begin{forest} qtree
        [TP
            [DP
                [D\\he, name=DP2]
            ]
            [T$'$
                [T\\PRES]
                [\emph{v}P, s sep+=20pt
                    [DP, name=DP1]
                    [\emph{v}$'$
                        [\emph{v}
                            [V\\loves, name=V2]
                            [\emph{v}]
                        ]
                        [VP, s sep+=20pt
                            [V, name=V1]
                            [DP, s sep+=30pt
                                [D\\ø]
                                [NP
                                    [{Egyptian cotton shirts}, triangle]
                                ]
                            ]
                        ]
                    ]
                ]
            ]
        ]
        \draw[->] (V1) [in=-90, out=-90, looseness=1.5] to (V2);
        \draw[->,dashed] (DP1) [in=-90, out=-90, looseness=1.5] to (DP2);
        \end{forest}
    }
    \label{ex:typesetting-trees}
\end{exe}

\begin{listing}[htbp]
	\centering
	\begin{minted}{latex}
\documentclass{article}
\usepackage{forest}
\forestset{
    qtree/.style={ % define a style that imitates the qtree package
        for tree={
            parent anchor=south,
            child anchor=north,
            align=center,
            inner sep=1pt
        }
    },
    .style={ % declare styles to apply to all forest environments
        qtree,
        baseline
    }
}
\usepackage{gb4e}
\noautomath
\begin{document}
\begin{exe}
    \ex[]{
        \begin{forest}
        [TP
            [DP
                [D\\he, name=DP2]
            ]
            [T$'$
                [T\\PRES]
                [\emph{v}P, s sep+=20pt
                    [DP, name=DP1]
                    [\emph{v}$'$
                        [\emph{v}
                            [V\\loves, name=V2]
                            [\emph{v}]
                        ]
                        [VP, s sep+=20pt
                            [V, name=V1]
                            [DP, s sep+=30pt
                                [D\\ø]
                                [NP
                                    [{Egyptian cotton shirts}, triangle]
                                ]
                            ]
                        ]
                    ]
                ]
            ]
        ]
        \draw[->] (V1) [in=-90, out=-90, looseness=1.5] to (V2);
        \draw[->,dashed] (DP1) [in=-90, out=-90, looseness=1.5] to (DP2);
        \end{forest}
    }
\end{exe}
\end{document}
	\end{minted}
	\caption{An example of typesetting a syntax tree using \Package{forest}}
	\label{lst:typesetting-trees}
\end{listing}

A few comments about what you see in Listing~\ref{lst:typesetting-trees} are in order.

First, \Package{forest} is designed to typeset trees as compactly as possible.
Sometimes when you want to show movement, you will therefore need to increase the distance between siblings so that arrows don't overlap with something else.
This is what \mintinline{latex}|s sep+=| allows you to do.
You can manually set the sibling separation with just \mintinline{latex}|s sep=|, whereas \mintinline{latex}|s sep+=| allows you to increase the default sibling separation value by a certain amount.

If you want to draw an arrow from one leaf in a tree to another leaf, you need to give them names that you can pass to the \mintinline{latex}|\draw| command.

If you want to have spaces in a node in a tree, you will need to surround the content in braces, like what I did with \mintinline{latex}|{Egyptian cotton shirts}| in Listing~\ref{lst:typesetting-trees}.

Finally, you will most likely want all of your trees to be typeset using both \mintinline{latex}|baseline| and \mintinline{latex}|qtree|.
By declaring \mintinline{latex}|\forestset{.style={qtree,baseline}}| in your preamble, you tell \Package{forest} to typeset all of your trees like this.%
\footnote{%
The method for declaring default styles is going to change in a future version of \Package{forest}.
I will update this handout when it does change.
}

The first option, \mintinline{latex}|baseline|, ensures that the baseline of the typeset object is the top of the tree, rather than the bottom of the tree.
This means that ``TP'' will be typeset as aligned with the example number (if you put it inside of an example environment) rather than ``Egyptian cotton shirts''.

The second option, \mintinline{latex}|qtree|, is a style that mimics the \Package{qtree} package style for trees.%
\footnote{%
Note that this style is not defined by \Package{forest}.
We must define it ourselves.%
}
This is the style that is standard for typesetting trees.
Specifically, it ensures that the branches of the tree connect at the bottom of each node.
Try typesetting a tree without using the \mintinline{latex}|qtree| option to see what happens.

\subsection{Typesetting OT tableaux}
\label{subsec:typesetting-OT-tableaux}

There are two options that I would recommend for typesetting OT tableaux.
One package is called \href{http://sanders.phonologist.org/OTtablx/}{\Package{OTtablx}}, which is in beta and not yet on CTAN.
Thus, in order to use it, you need to put the \File{.sty} file in either the same directory as your \File{.tex} file or put it in a local folder that your \TeX{} distribution can see (\IE{} \Directory{texmf/tex/latex}; cf.~\S\ref{subsec:local-files}).%
\footnote{%
Note that if you do decide to try \Package{OTtablx} at some point, it must first be compiled to DVI format and then PDF format because it relies on the package \Package{pstricks}.
If you compile with \hologo{XeLaTeX}, things should work fine, but it will not work with \hologo{pdfLaTeX}.
See \href{http://tex.stackexchange.com/q/8413/32888}{this question and its answers on TeX.SX} for discussion.%
}

To avoid this complication for the sake of an introductory workshop, we will instead use the package called \Package{ot-tableau}, which is available on CTAN (\IE{} it is part of any good \TeX{} distribution, such as TeX Live or \hologo{MiKTeX}).
An example of how to typeset a tableau is given in Listing~\ref{lst:example-tableau}.
This code will produce the result in (\ref{ex:example-tableau}).%
\footnote{%
Disclaimer: I'm not a phonologist.%
}

\begin{exe}
    %
    \ex[]{
        \begin{tableau}{c|c|c}
            \inp{/bad/}          \const{*Voiced-Coda}  \const*{\Constr{Ident-IO}(voice)}  \const*{*[+voi,--son]}
            \cand{bad}           \vio{*!}              \vio{}                             \vio{**}
            \cand[\Optimal]{bat} \vio{}                \vio{*}                            \vio{*}
            \cand{pat}           \vio{}                \vio{**!}                          \vio{}
            \cand{pad}           \vio{*!}              \vio{*}                            \vio{*}
        \end{tableau}
    }
    \label{ex:example-tableau}
    %
\end{exe}

\begin{listing}[htbp]
	\centering
	\begin{minted}{latex}
\documentclass{article}
\usepackage[
    shadedcells,
    notipa
]{ot-tableau}
\usepackage{gb4e}
\noautomath

\newcommand*{\Constr}[1]{\textsc{#1}}

\begin{document}

\begin{exe}
\ex[]{
\begin{tableau}{c|c|c}
\inp{/bad/}          \const{*Voiced-Coda} \const*{\Constr{Ident-IO}(voice)} \const*{*[+voi,--son]}
\cand{bad}           \vio{*!}             \vio{}                            \vio{**}
\cand[\Optimal]{bat} \vio{}               \vio{*}                           \vio{*}
\cand{pat}           \vio{}               \vio{**!}                         \vio{}
\cand{pad}           \vio{*!}             \vio{*}                           \vio{*}
\end{tableau}
}
\end{exe}

\end{document}
	\end{minted}
	\caption{An example of an OT tableau}
	\label{lst:example-tableau}
\end{listing}

% !TEX encoding = UTF-8 Unicode
% !TEX root = ../latex-workshop-for-linguists.tex

\section{Getting help}
\label{sec:getting-help}

As already mentioned, there is \href{http://tex.stackexchange.com/}{TeX.SX}.
Don't forget to try searching the site before you ask your question.
Chances are that somebody has already asked it.
But we are generally pretty friendly and nice on TeX.SX, so don't hesitate to ask if you can't find an answer!

If you do decide to ask a question on TeX.SX, in most cases you should provide a \href{http://meta.tex.stackexchange.com/q/228/32888}{Minimal (non-)Working Example (MWE)}.
The process of creating an MWE is often a good way to debug any problems you run into, and, in many cases, you might end up fixing the problem yourself in the course of creating an MWE.

There's also the \href{https://en.wikibooks.org/wiki/LaTeX}{\LaTeX{} Wikibook}, which is generally pretty good.

If you're struggling with a particular package, try reading the documentation.
Unlike most open-source software projects, \LaTeX{} packages generally have \emph{really good} documentation.
You can find package documentation in a few places.

First, it's always on \href{http://ctan.org/}{CTAN}.

CTAN isn't always the easiest to navigate, so two folks---Stefan Kottwitz and Paulo Cereda---have set up \href{http://texdoc.net/}{TeXDoc Online}, which allows you to easily search for package documentation by the name of the package.

TeXDoc Online is effectively just an online version of the command line tool, \mintinline{sh}{texdoc}, which is part of both TeX Live and \hologo{MiKTeX}.
You can just open a terminal and type \mintinline{sh}{texdoc <package name>}.

And, of course, there are also some good books for learning \LaTeX.
There's a free one called the \href{http://texdoc.net/texmf-dist/doc/latex/lshort-english/lshort.pdf}{\emph{The Not So Short Introduction to \LaTeXe}}.%
\footnote{%
If you have a \TeX{} distribution installed, you can find this book on your system by typing \mintinline{sh}{texdoc lshort} at a terminal.%
}
There's also \href{http://www.amazon.com/LaTeX-Companion-Techniques-Computer-Typesetting/dp/0201362996}{\emph{The \LaTeX{} Companion}}.

% !TEX encoding = UTF-8 Unicode
% !TEX root = ../latex-workshop-for-linguists.tex

\section{Acknowledgments}
\label{sec:acknowledgments}

Many thanks to both Alan Munn and Kenneth Hanson for reading over early drafts of this handout and providing useful feedback and suggestions.
Thanks also to Joseph Wright for help with sorting out the problems with the previously untested and broken commands in Listings~\ref{lst:Windows-make-TEXMFHOME},~\ref{lst:Windows-make-TEXMFHOME-in-Dropbox},~and~\ref{lst:Windows-make-local-texmf-MiKTeX}.
Moreover, I also owe many thanks to TeX.SX for being a great resource, for helping me to learn a lot about \LaTeX, and, most importantly, for being a great community consisting of a bunch of friendly folks. \resizebox*{!}{1em}{\rubberduck}

\end{document}