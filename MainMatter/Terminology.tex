% !TEX encoding = UTF-8 Unicode
% !TEX root = ../latex-workshop-for-linguists.tex

\section{Terminology}
\label{sec:terminology}

This section is largely a brief recap of \S1 of Alan Munn's \TitleText{A Beginner's Guide to \LaTeX{} (on the Mac)}.%
\footnote{%
A \File{.pdf} of this is available at \url{https://www.msu.edu/~amunn/latex/nano-companion.pdf}.
If you have trouble accessing it, try refreshing the page a few times.
The web servers at Michigan State University can sometimes be sort of shitty.%
}
But you should really just read that whole PDF (and ignore the parts specific to Mac if you're not on a Mac).
It is both short and useful.

\begin{description}

	\item[\TeX{} distribution]{%
		Contains all of the programs and packages that will be used to process and compile your \File{.tex} file.
		There are two main distributions: TeX Live and \hologo{MiKTeX}.
		There is also a distribution called {Mac\TeX}, which is a wrapper around TeX Live that does some stuff to make it work nicely on a Mac.
		\hologo{MiKTeX} is for Windows only, and it is not based on TeX Live.%
		\footnote{%
		For discussion of the differences between \hologo{MiKTeX} and TeX Live, see \url{http://tex.stackexchange.com/q/20036/32888}.
		If you're on Linux, do \emph{not} install TeX Live via your package manager!!!
		You should instead install a ``vanilla'' version of TeX Live. See \url{http://tex.stackexchange.com/q/1092/32888}.
		See also \S\ref{subsec:installing-a-tex-distro} of this handout.%
		}
		In the computer world, it is common to shorten ``distribution'' to ``distro''---\IE{} \TeX{} distro.%
	}
	
	\item[Engines]{%
		There are a few different engines that are standardly used to process a \File{.tex} file and turn it into a PDF, including \hologo{pdfLaTeX}, \hologo{XeLaTeX}, and \hologo{LuaLaTeX}.%
	}
	
	\item[Editor]{%
		The application that is used to write the \File{.tex} file.
		See \href{http://tex.stackexchange.com/q/339/32888}{here} for a long list of editors to choose from.%
	}
	
	\item[Previewer]{%
		An application for viewing the output of compiling the \File{.tex} file with an engine.
		Many editors integrate a previewer into the editor.%
	}
	
	\item[Compiling]{%
		The act of processing a \File{.tex} file with an engine to produce (most likely) a PDF.
		Can sometimes loosely be used interchangeably with ``typesetting''.%
	}
	
	\item[Preamble]{%
		Refers to the part of the document between \mintinline{latex}|\documentclass| and \mintinline{latex}|\begin{document}|.
		It is where you can load packages and define new commands, among other things.
		See Figure~\ref{fig:document-structure}.
		
		\begin{figure}[htbp]
			\centering
			\begin{tikzpicture}
				[
					baseline,
					left brace/.style={
						semithick,
						decorate,
						decoration={
							brace,
							raise=2mm,
							amplitude=3pt,
							mirror
						}
					}
				]
				\matrix(M)[
					matrix of nodes,
					nodes={
						align=left,
						text width=3.5cm
					}
				]{
					\mintinline{latex}|\documentclass{...}|\\
					$\vdots$\\
					\mintinline{latex}|\begin{document}|\\
					$\vdots$\\
					\mintinline{latex}|\end{document}|\\
				};
				\draw (M-1-1.north west) rectangle (M-5-1.south east);
				\draw [left brace] (M-1-1.south west) -- (M-3-1.north west) node [midway, left=1em] {preamble};
				\draw [left brace] (M-3-1.south west) -- (M-5-1.north west) node [midway, left=1em] {content};
			\end{tikzpicture}
			\caption{Schematic structure of a \LaTeX{} document}
			\label{fig:document-structure}
		\end{figure}
	%
	}
	
	\item[TeX.SX]{%
		Throughout this document, you will probably see numerous references to TeX.SX.
		This is short for \href{http://tex.stackexchange.com/}{TeX Stack Exchange}.
		If you're not familiar with the \href{http://stackexchange.com/sites#traffic}{family of Stack Exchange websites}, you should really check them out.
		Each site is a Q\&A website for a specific topic, but the sites are intended to be repositories of knowledge in addition to Q\&A sites, so they aren't like your normal web forum.
		Most sites go with the suffix of .SE, but the folks that use TeX Stack Exchange are a bit idiosyncratic and generally prefer the suffix .SX.%
	}

\end{description}