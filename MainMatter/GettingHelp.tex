% !TEX encoding = UTF-8 Unicode
% !TEX root = ../latex-workshop-for-linguists.tex

\section{Getting help}
\label{sec:getting-help}

As already mentioned, there is \href{http://tex.stackexchange.com/}{TeX.SX}.
Don't forget to try searching the site before you ask your question.
Chances are that somebody has already asked it.
But we are generally pretty friendly and nice on TeX.SX, so don't hesitate to ask if you can't find an answer!

If you do decide to ask a question on TeX.SX, in most cases you should provide a \href{http://meta.tex.stackexchange.com/q/228/32888}{Minimal (non-)Working Example (MWE)}.
The process of creating an MWE is often a good way to debug any problems you run into, and, in many cases, you might end up fixing the problem yourself in the course of creating an MWE.

There's also the \href{https://en.wikibooks.org/wiki/LaTeX}{\LaTeX{} Wikibook}, which is generally pretty good.

If you're struggling with a particular package, try reading the documentation.
Unlike most open-source software projects, \LaTeX{} packages generally have \emph{really good} documentation.
You can find package documentation in a few places.

First, it's always on \href{http://ctan.org/}{CTAN}.

CTAN isn't always the easiest to navigate, so two folks---Stefan Kottwitz and Paulo Cereda---have set up \href{http://texdoc.net/}{TeXDoc Online}, which allows you to easily search for package documentation by the name of the package.

TeXDoc Online is effectively just an online version of the command line tool, \mintinline{sh}{texdoc}, which is part of both TeX Live and \hologo{MiKTeX}.
You can just open a terminal and type \mintinline{sh}{texdoc <package name>}.

And, of course, there are also some good books for learning \LaTeX.
There's a free one called the \href{http://texdoc.net/texmf-dist/doc/latex/lshort-english/lshort.pdf}{\emph{The Not So Short Introduction to \LaTeXe}}.%
\footnote{%
If you have a \TeX{} distribution installed, you can find this book on your system by typing \mintinline{sh}{texdoc lshort} at a terminal.%
}
There's also \href{http://www.amazon.com/LaTeX-Companion-Techniques-Computer-Typesetting/dp/0201362996}{\emph{The \LaTeX{} Companion}}.