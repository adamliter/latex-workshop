% !TEX encoding = UTF-8 Unicode
% !TEX root = ../latex-workshop-for-linguists.tex

\section{Things to learn on your own}
\label{sec:things-to-learn-on-your-own}

This handout is already getting much too long.
Moreover, one of the best ways to get comfortable with \LaTeX{} (or anything for that matter) is to try learning some stuff on your own.

Here are some packages or tools (and some brief descriptions thereof) that I would highly recommend exploring on your own at some point.

\begin{description}

	\item[\Package{latexmk}]{%
		\href{https://www.ctan.org/pkg/latexmk}{\Package{latexmk}} is a tool that tries to automagically guess what compilation steps you need to do in order to successfully compile your document.
		It will then try to do all of these compilation steps for you.
		If you get tired of going through all of the compilation steps necessary for doing a bibliography in \LaTeX{} you might try compiling your document with \Package{latexmk}.%
	}
	
	\item[\Package{arara}]{%
		An alternative to \Package{latexmk} that I would highly recommend is \href{https://www.ctan.org/pkg/arara}{\Package{arara}}.
		Rather than trying to automagically guess the compilation steps necessary, \Package{arara} allows you to specify the compilation steps that are necessary at the top of your document using a special syntax.
		Of course, this requires knowing which compilation steps are necessary, so it's not quite as straightforward as \Package{latexmk}.
		However, it allows for much more fine-grained control of the compilation process that might be preferable when you start using more advanced tools.%
	}
	
	\item[\Package{datatool}]{%
		The \href{https://www.ctan.org/pkg/datatool}{\Package{datatool}} package allows you to loop over information stored in a \File{.csv} file and incorporate it into your \File{.tex} document.%
	}
	
	\item[Different document classes]{%
		Try exploring some different document classes.
		In this handout, we've only used the basic \Package{article} document class.
		If you want to use \LaTeX{} for presentations, consider taking a look at the \href{https://ctan.org/pkg/beamer}{\Package{beamer}} document class.
		Another document class that is useful for typesetting longer documents, such as a monograph, is the \href{https://ctan.org/pkg/memoir}{\Package{memoir}} document class.%
	}
	
	\item[\Package{knitr}]{%
		If you use R for statistics and processing data, I would \emph{highly} recommend looking at \href{http://yihui.name/knitr/}{\Package{knitr}}.
		This is a package that will allow you to write R code and \LaTeX{} markup in the same source document.
		If you do this, then you can automagically generate things like p-values, for example.%
	}

\end{description}