% !TEX encoding = UTF-8 Unicode
% !TEX root = ../latex-workshop-for-linguists.tex

\section{General \LaTeX{} stuff}
\label{sec:general-latex-stuff}

\subsection{\LaTeX{} philosophy}
\label{subsec:latex-philosophy}

This subsection is very similar to \S3 of Alan Munn's \TitleText{A Beginner's Guide to \LaTeX{} (on the Mac)}.

\LaTeX{} was designed with the intent of separating content from formatting.
This is quite different from a what-you-see-is-what-you-get (WYSIWYG) editor like Word, where you see the output of your content formatted as you go along.

Something that goes hand in hand with separating content from formatting is that formatting should be given a semantics.
What does this mean?
(Bahhhh duhhhh chhhh!)
This means that if you want \LaTeX{} to format things that are similar in nature in the same way, then you should give them the same semantic meaning.

Alan gives the example of section headings.
The proper way to make a section heading in \LaTeX{} is to write it like \mintinline{latex}|\section{Section title}|.
The command \mintinline{latex}|\section{}| gives the content ``Section title'' the semantics of being a \Semantics{section}.
Then, if you want to change anything about how your sections are formatted, you can change this in the preamble of your document.
This contrasts with how many folks use Word where they would change this for each individual section heading.%
\footnote{%
Note that Word also allows for semantic markup despite the fact that most people do not use it.
If you cannot convince your Word-using friends and family to switch to \LaTeX, you should at least try to get them to use semantic markup in Word if they don't already do so.
\texttt{:)}%
}
For example, if you want the number preceding all of your section headings to be blue, you can change this at the beginning of the document, like in Listing~\ref{lst:blue-sections}.

\begin{listing}[htbp]
	\centering
	\begin{minted}{latex}
\documentclass{article}

\usepackage{color} % this package provides the command \color{}
\renewcommand\thesection{\color{blue}\arabic{section}}

\begin{document}

\section{Introduction}

Blah blah.

\section{Experiment}

Blah blah.

\section{Conclusion}

Blah blah.

\end{document}
	\end{minted}
	\caption{Example of semantic markup in \LaTeX{} for section headings}
	\label{lst:blue-sections}
\end{listing}

\subsection{Titles}
\label{subsec:titles}

To typeset a title, an author, and a date in a paper using the basic \Package{article} class, you can do what is shown in Listing~\ref{lst:basic-title}.

\begin{listing}[htbp]
	\centering
	\begin{minted}{latex}
\documentclass{article}

\title{Super awesome title}
\author{Best Author}
\date{July 22, 2015} % if you want today's date, replace July 22, 2015 with \today

\begin{document}

\maketitle

\section{Introduction}

Blah blah.

\end{document}
	\end{minted}
	\caption{A basic example of how to typeset the title of a paper using the \Package{article} class}
	\label{lst:basic-title}
\end{listing}

\subsection{Quotes and dashes}
\label{subsec:quotes-and-dashes}

One idiosyncrasy of \LaTeX{} that you will have to get used to is how to typeset quotes and dashes.

To typeset double open quotes, write \mintinline{latex}|``|.

To typeset double close quotes, write \mintinline{latex}|''|.

To typeset a single open quote, write \mintinline{latex}|`|.

To typeset a single close quote, write \mintinline{latex}|'|.

To typeset an en-dash, write \mintinline{latex}|--|.

To typeset an em-dash, write \mintinline{latex}|---|.

Note, however, that if you process your file with an engine that plays nicely with UTF-8 encoded documents (see \S\ref{subsec:fontspec-and-unicode}), you can enter these characters directly into your editor.

Doing this has the advantage of making your documents more readable, but note that it \emph{only} works if you use UTF-8 encoding and a compatible engine.
Also, it is good to know about the old way of typesetting these ligatures using \LaTeX{} since you will probably see many instances of this on the internet.

\subsection{Formatting text}
\label{subsec:formatting-text}

To typeset something in bold, use \mintinline{latex}|\textbf{}|.

To typeset something in italics, use \mintinline{latex}|\textit{}|.

To typeset something in small caps, use \mintinline{latex}|\textsc{}|.

To typeset something in a mono-spaced font, use \mintinline{latex}|\texttt{}|.

To underline something, don't.%
\footnote{%
Underlining is really frowned upon in the typography community.
I also personally do not like it.
However, if you have a really, really, really, really, really (really) good reason, then I suppose you can use \mintinline{latex}|\uline| from the \Package{ulem} package.
When you load, \Package{ulem} be sure to pass it the optional argument \mintinline{latex}|normalem|.%
}

One important thing to note about all of the foregoing commands is that using them directly in your document is \emph{not} good practice.
They are not semantic commands.
Let's consider an example.
Being a linguist, you will probably want to typeset certain glossed features in small caps, like \Nom, for instance.
You might think to write \mintinline{latex}|\textsc{nom}|.

This is \emph{bad practice}.
Instead, you should give glossed feature abbreviations like this a semantics, since you will presumably want to typeset them all the same way.
To do this, you could declare \mintinline{latex}|\newcommand*{\Fts}[1]{\textsc{#1}}| in your preamble.
Then, you would be able to write \mintinline{latex}|\Fts{nom}| instead of \mintinline{latex}|\textsc{nom}|.

A further example can be seen if you look at the source code for this handout.
You will notice that I typeset all the names of packages in a mono-spaced font.
Rather than writing, for example, \mintinline{latex}|\texttt{forest}|, I have written \mintinline{latex}|\Package{forest}|.
In my preamble, I defined the \mintinline{latex}|\Package{}| command in the following way: \mintinline{latex}|\newcommand*{\Package}[1]{\texttt{#1}}|.
If I ever wanted to change how the name of every package is typeset in this document, I would only need to change it once in my preamble.

\subsection{Footnotes}
\label{subsec:footnotes}

To typeset footnotes, use \mintinline{latex}|\footnote{}|.

\subsection{Special characters}
\label{subsec:special-characters}

There are several characters that are treated as special characters in \LaTeX.
These are \mintinline{text}|# $ % & ~ _ ^ \ { }|.

If you ever want to print any of these characters in the output, you need to escape them with `\texttt{\textbackslash}'.%
\footnote{%
One exception to this is the escape character itself, `\texttt{\textbackslash}', because the sequence `\mintinline{text}|\\|' has a special meaning in \LaTeX, used for line breaks in tables.
If you wish to render the character `\texttt{\textbackslash}' you can use the command \mintinline{latex}|\textbackslash|.
Two further exceptions are \mintinline{text}|^| and \mintinline{text}|~|.
Preceding these two characters with `\texttt{\textbackslash}' is used for appending diacritics in \LaTeX{} (though see \S\ref{subsec:fontspec-and-unicode} for a better way of doing this).
If you wish to print these characters, you will need to do \mintinline{latex}|\^{}| and \mintinline{latex}|\~{}|, respectively (though see \href{http://tex.stackexchange.com/q/312/32888}{this post on TeX.SX} for suggestions of better ways to typeset a tilde).
\label{fn:diacritics}%
}

The character `\mintinline{text}|#|' is used for passing arguments to macros.

The character `\mintinline{text}|$|' is used for entering math mode (see \S\ref{subsec:math-mode}).

The character `\mintinline{text}|%|' is used for writing comments in the source document (cf.~Listing~\ref{lst:blue-sections}).

The character `\mintinline{text}|&|' is used for separating columns in a table (see \S\ref{subsec:tables}).

The character `\mintinline{text}|~|' is a non-breaking space.

The character `\mintinline{text}|_|' is used for subscripts in math mode.

The character `\mintinline{text}|^|' is used for superscripts in math mode.

The character `\mintinline{text}|\|' is the escape character.

The characters `\mintinline{text}|{ }|' are used for delimiting the arguments to commands.

\subsection{Math mode}
\label{subsec:math-mode}

One thing worth knowing about \LaTeX{} is that it has a distinct mode for typesetting math, creatively called math mode.
There is inline math, triggered by \mintinline{latex}|$...$|, and display math, triggered by \mintinline{latex}|\[...\]|.

For linguists, math mode is something that is mostly useful for typesetting semantics.%
\footnote{%
This is a bit of an overstatement.
There are other use cases.
Typesetting feature bundles in morphosyntax is one such use case, for example.
However, for the purposes of an introductory workshop, you should really just know what math mode is.%
}
For example, \mintinline{latex}|$\lambda x$| will produce $\lambda x$.

\subsection{Tables}
\label{subsec:tables}

Tables are admittedly a bit of a pain in the ass in \LaTeX.
Typesetting them takes a while to get used to.
Let's see an example of a basic table, such as the one in Listing~\ref{lst:basic-table}.

\begin{listing}[htbp]
	\centering
	\begin{minted}{latex}
\documentclass{article}
\begin{document}

\begin{tabular}{lcr}
    Left-aligned column & Center-aligned column & Right-aligned column \\ \hline
    56\%                & 75\%                  & 34\%                 \\
\end{tabular}

\end{document}
	\end{minted}
	\caption{A basic table in \LaTeX}
	\label{lst:basic-table}
\end{listing}

\begin{minipage}{\textwidth}

The code in Listing~\ref{lst:basic-table} will produce the following table.

\begin{center}
	\begin{tabular}{lcr}
		Left-aligned column & Center-aligned column & Right-aligned column \\ \hline
		56\%                & 75\%                  & 34\%                 \\
	\end{tabular}
\end{center}

\end{minipage}

One useful package for making aesthetically pleasing tables is the package called \Package{booktabs}.
It provides commands called \mintinline{latex}|\toprule|, \mintinline{latex}|\bottomrule|, and \mintinline{latex}|\midrule| for nicer horizontal rules in a table.

\begin{minipage}{\textwidth}

Consider Listing~\ref{lst:booktabs-table}, which produces the following output.

\begin{center}
	\begin{tabular}{lll}
		\toprule
		         & Passive sentences & Active sentences \\ \midrule
		Adults   & 99\%              & 98\%             \\
		Children & 56\%              & 87\%             \\
		\bottomrule
	\end{tabular}
\end{center}

\end{minipage}

\begin{listing}[htbp]
	\centering
	\begin{minted}{latex}
\documentclass{article}
\usepackage{booktabs}
\begin{document}

\begin{tabular}{lll}
    \toprule
             & Passive sentences & Active sentences \\ \midrule
    Adults   & 99\%              & 98\%             \\
    Children & 56\%              & 87\%             \\
    \bottomrule
\end{tabular}

\end{document}
	\end{minted}
	\caption{A table in \LaTeX{} using the package \Package{booktabs}}
	\label{lst:booktabs-table}
\end{listing}

\subsection{Images}
\label{subsec:images}

One thing you will often want to do is include images in a document.
This is what the package \Package{graphicx} is for.
Let's look at the example in Listing~\ref{lst:image-example}.

\begin{listing}[htbp]
	\centering
	\begin{minted}{latex}
\documentclass{article}
\usepackage{graphicx}
\graphicspath{ {figure/} }
\begin{document}

\includegraphics[width=.8\textwidth]{super-important-graph}

\end{document}
	\end{minted}
	\caption{An example of including an image in a document}
	\label{lst:image-example}
\end{listing}

Notice that in the preamble of the document, we declared \mintinline{latex}|\graphicspath{ {figure/} }|.
This means that the package \Package{graphicx} will look in the folder called \texttt{figure} for images.%
\footnote{%
You don't need to do this.
If you prefer, you can just put the image file in the same folder as the \File{.tex} file.
The working directory (\IE{} the folder that the \File{.tex} file is in) is a place that the engine will always search when typesetting a document.
So if you're struggling with setting up a local \Directory{texmf} folder as described in \S\ref{subsec:local-files}, you can always just put files in the same folder as your \File{.tex} file for the time being.
But anyway, a reason you might want to have a separate folder dedicated for figures is to avoid clutter.
It's really up to you.%
}
So, in order to get this \File{.tex} file to compile, you would want to save it in a folder; then, in that same folder, you would want to create a new folder called \Directory{figure}.
Inside that folder, you should put the file \File{super-important-graph.pdf}.%
\footnote{%
The package \Package{graphicx} doesn't always play nicely with spaces and underscores in file names, so you should avoid using those things in the names of your image files.%
}
Notice that the file extension is omitted in the call to \mintinline{latex}|\includegraphics{}|.
This is a best practice because it allows \Package{graphicx} to try a bunch of different file extensions.

Notice also that we passed an optional argument to \mintinline{latex}|\includegraphics{}| in the form of a \textit{key val list}.
The key \KEY{width} can take a value that specifies what the width of the image that appears in the typeset document should be.
You could give it a value of \mintinline{latex}|6in| if you wanted, but it is often much more sensible to specify the width in terms of a dynamically defined value.
In this case, the super important graph will always occupy 80\% of the space allocated to the text, even if we change the margins of the document.

\subsection{Captioning and numbering}
\label{subsec:captioning-and-numbering}

Typesetting a table and including an image is great and all, but we want to be able to caption and number them.

\subsubsection{Floats}
\label{subsubsec:floats}

One common way to do this is to use floats.
In addition to automagically numbering tables and figures, floats also allow us to provide a caption.
One thing to know about floats is that \LaTeX{} has a special way of handling how they are typeset.
Suffice it to say, it's rather complicated.%
\footnote{%
You can read more about it \href{http://tex.stackexchange.com/q/39017/32888}{here} if you're interested.%
}
All you need to know is that \LaTeX{} has a special algorithm for placing floats in the best possible spot, according to general typographical standards.

These places are usually one of four places: right where they are written in the source document, the top of a page, the bottom of a page, or on their own separate page.
These four places correspond to four optional arguments that you can pass to a float environment, \mintinline{text}|htbp|, respectively.

It is generally best practice to pass all four options to a float, at least initially.
Only when you finish writing the document should you fiddle with the placement of floats if you think \LaTeX's algorithm has not done a good job.
However, while you're writing a document, leave all four options and let \LaTeX{} decide where floats should be placed.

If you would prefer to increase the likelihood that the float will show up in exactly the location that it is specified in the source \File{.tex} file, you can place a \mintinline{text}|!| after the \mintinline{text}|h|.

The command \mintinline{latex}|\caption{}| allows you to give a caption to the table or figure.
\LaTeX{} will automagically number the tables and figures in the correct order, so you don't have to worry about that.
Semantic markup FTW!

Take a look at two examples in Listing~\ref{lst:float-examples}.
Try typesetting this yourself and see what the result is.

\begin{listing}[htbp]
	\centering
	\begin{minted}{latex}
\documentclass{article}
\usepackage{graphicx}
\usepackage{booktabs}
\begin{document}

This is a table that is a float.

\begin{table}[htbp]
    \centering
    \begin{tabular}{lll}
        \toprule
                 & Passive sentences & Active sentences \\ \midrule
        Adults   & 99\%              & 98\%             \\
        Children & 56\%              & 87\%             \\
        \bottomrule
    \end{tabular}
    \caption{Adult performance compared to child performance}
\end{table}

It might not actually show up in between these two sentences.

This is a figure that is a float.

\begin{figure}[htbp]
    \centering
    \includegraphics[width=.8\textwidth]{example-image-a}
    \caption{Super scientificy graphy thingy}
\end{figure}

It might not actually show up in between these two sentences.

\end{document}
	\end{minted}
	\caption{Examples of floats in \LaTeX}
	\label{lst:float-examples}
\end{listing}

\subsubsection{Non-float options}
\label{subsubsec:non-float-options}

It is worth mentioning some non-float options for tables and images.
One of the reasons that it is worth mentioning these options is because there is a common misconception that tables must go inside \mintinline{latex}|table| environments and images must go inside \mintinline{latex}|figure| environments.
This is not true.

When you're writing a paper, it is best to use floats because \LaTeX{} will use its algorithm to place the floats in the best environment.
However, it is not always appropriate to use floats.

One example of when you probably don't want to use floats is when you're making a handout.
In a handout, you usually want the image or the table to show up exactly where you place the code in the source document.

Nonetheless, in this case you might still want to be able to number and caption the figure or table.
The package \Package{capt-of} allows you to do this.
An example is given in Listing~\ref{lst:capt-of-example}.
Note that you will want to put things inside of an enclosing group, such as \mintinline{latex}|\begin{center}...\end{center}|.

\begin{listing}[htbp]
	\centering
	\begin{minted}{latex}
\documentclass{article}
\usepackage{graphicx}
\usepackage{booktabs}
\usepackage{capt-of}
\begin{document}

This is a table that is not a float.

\begin{center}
    \begin{tabular}{lll}
        \toprule
                 & Passive sentences & Active sentences \\ \midrule
        Adults   & 99\%              & 98\%             \\
        Children & 56\%              & 87\%             \\
        \bottomrule
    \end{tabular}
    \captionof{table}{Adult performance compared to child performance}
\end{center}

It will show up in between these two sentences no matter what.

This is a figure that is not a float.

\begin{center}
    \includegraphics[width=.8\textwidth]{example-image-a}
    \captionof{figure}{Super scientificy graphy thingy}
\end{center}

It will show up in between these two sentences no matter what.

\end{document}
	\end{minted}
	\caption{Examples of tables and images as non-floats}
	\label{lst:capt-of-example}
\end{listing}

Another case where you might want to not put a table inside of a float is when you're making a table for a particular morphological paradigm.
In linguistics, we usually number such tables just like we number other examples.
In \S\ref{subsec:examples}, we will see how to make numbered examples.
You can put a \mintinline{latex}|tabular| environment directly inside such an example.

\subsection{Cross referencing}
\label{subsec:cross-referencing}

So automagically numbered tables and images are great and all, but how do I refer to those things in my document?
One thing that is great about \LaTeX{} is that you can give things \mintinline{latex}|\label|s and \mintinline{latex}|\ref|er to them automagically as well.

Consider Listing~\ref{lst:cross-referencing}.

\begin{listing}[htbp]
	\centering
	\begin{minted}{latex}
\documentclass{article}
\usepackage{graphicx}
\begin{document}

As can be seen in Figure~\ref{fig:important-graph}, the results clearly show that I'm right.

\begin{figure}[htbp]
    \centering
    \includegraphics[width=.8\textwidth]{example-image-a}
    \caption{Super scientificy graphy thingy}
    \label{fig:important-graph}
\end{figure}

\end{document}
	\end{minted}
	\caption{An example of referencing a figure in \LaTeX}
	\label{lst:cross-referencing}
\end{listing}

There are a few things to say about cross-referencing.
First, and of particularly important note is the fact that the \mintinline{latex}|\label| comes after the \mintinline{latex}|\caption|.
If you try putting the \mintinline{latex}|\label| first, you will get the wrong number, because the command \mintinline{latex}|\caption| is what gives the float its number.

Second, you will notice that I've put a non-breaking space between \mintinline{text}|Figure| and \mintinline{latex}|\ref|.
This is generally a good practice because it prevents the number from being separated from the description of what it is.%
\footnote{%
If you really like automagic, you might want to check out the \href{http://ctan.org/pkg/cleveref}{\Package{cleveref}} package.%
}

Third, you will also notice that I've given the label a prefix of \mintinline{text}|fig:|.
This isn't strictly necessary, but it is good practice.
Imagine that you had a table and a figure.
The table contains specific values, and the figure is a graph of those values.
It's the same data, so you presumably want to give them similar names.
If you use prefixes like this, you could do \mintinline{latex}|\label{tab:super-important-results}| for the table and \mintinline{latex}|\label{fig:super-important-results}| for the graph.%
\footnote{%
Spaces are not allowed in the names of \mintinline{latex}|\label|s.%
}

Fourth, and most importantly, is that you must compile your document twice in order for this to work.
\LaTeX{} does its automagic by first processing the file and automagically generating all of the table and figure numbers for each float.
Remember that you never manually gave each float a number, so \LaTeX{} has to figure this out.
On the second compilation, it inserts the automagically generated numbers into the places where you \mintinline{latex}|\ref|erenced them.
If something went wrong, you will see question marks instead of a number.
This most likely means you either only compiled your document once or you have tried to refer to something that you never actually labeled.
The most common example of this latter reason is just a simple misspelling of the label that you gave to whatever it is that you're trying to reference.

\subsection{Those annoying files}
\label{subsec:those-annoying-files}

One thing you will quickly notice when you typeset a \File{.tex} file is that a lot of extra files are generated.
People tend to initially find this annoying, but it is all of these extra files that allow \LaTeX{} to do its magic.
For example, the auxiliary file (\File{.aux}) is integral for cross referencing.
Without it, cross referencing just would not work.

To avoid clutter, it's often a good idea to make a new, self-contained folder for each document that you typeset.

One other thing worth noting about all of these extra files is that they sometimes lead to compilation errors.
If you introduced an error in your document and you tried typesetting it, it's possible that the extra files got messed up.
So, if you tried typesetting your document, received a compilation error, figured out what caused the error, and you're like 110\% sure that you fixed the problem in your \File{.tex} file, but you're still getting a compilation error, try deleting all of these extra files and compiling the document again.

\subsection{Bibliographies}
\label{subsec:bibliographies}

Getting a bibliography to work with \LaTeX{} is often one of the big hurdles of learning \LaTeX, but once you've figured it out, it's really, really, really frikken nice.

\paragraph{Promissory note}

Unfortunately, getting this set up properly requires going into how to install a \TeX{} distribution on your computer and how to properly set up a local TDS-compliant directory.
If folks are interested, we could do a follow up workshop next Wednesday that goes into how to do this.

\subsection{Paragraphs}
\label{subsec:paragraphs}

Let's end this section with paragraphs.
Let's do this for two reasons.
First, because semantic markup is awesome.
And second, because a very common \emph{really bad practice} of \LaTeX{} beginners is to insert line breaks all over the place.

Recall from \S\ref{subsec:latex-philosophy} that \LaTeX{} is all about semantic markup.
This goes for paragraphs, too.
People who are used to Word are used to pressing ENTER on the keyboard once in order to separate paragraphs.
Perhaps unsurprisingly then, many new \LaTeX{} users will often do stuff like what is shown in Listing~\ref{lst:bad-practice-with-line-breaks}.

\begin{listing}[htbp]
	\centering
	\begin{minted}{latex}
\documentclass{article}
\begin{document}
This is my first awesome paragraph.\\
This is my second paragraph, which is infinitely less awesome because of the line break.
\end{document}
	\end{minted}
	\caption{Really bad practice for separating paragraphs}
	\label{lst:bad-practice-with-line-breaks}
\end{listing}

The command \mintinline{latex}|\\| does a line break, but it does \emph{not} introduce a new paragraph.
In other words, the (first part of the) second sentence is false.
In \LaTeX{}'s eyes, the sentence ``This is my first awesome paragraph.'' is in the \emph{same paragraph} as the sentence ``This is my second paragraph, which is infinitely less awesome because of the line break.''.

Instead, one should use a command for paragraphs so that we can manipulate the semantics of paragraphs in the preamble of the document in the same way that we manipulated the semantics of sections.
The command for separating one paragraph from another paragraph is \mintinline{latex}|\par|.
This would get really annoying to type in between all of your paragraphs, so, luckily, \LaTeX{} treats an empty line as equivalent to \mintinline{latex}|\par|.

Listing~\ref{lst:good-practice-for-paragraphs} exemplifies \emph{good practice} for typesetting paragraphs.
This good practice allows us to manipulate the semantics of paragraphs in the preamble, so we can typeset them as we like, without having to modify each individual paragraph.

There's generally no need to modify the default semantics for paragraphs.
However, in Listing~\ref{lst:good-practice-for-paragraphs}, I've given the same semantics that are used to typeset the paragraphs in this document, which are intended to be much more handout-y and much less essay-y.
Try typesetting Listing~\ref{lst:good-practice-for-paragraphs} yourself and see what happens.

\begin{listing}[htbp]
	\centering
	\begin{minted}{latex}
\documentclass{article}
\setlength{\parindent}{0em}
\setlength{\parskip}{1ex}
\begin{document}

This is the first paragraph.
Wasn't that a great topic sentence?

Next paragraph please.
Paragraph number two is the best.
\par
The third paragraph will rule them all.
Sorry about the Lord of the Rings reference.

This concludes my five paragraph essay. As you can see, my conclusion definitely follows.

Yes I can count.
Jeeze.

\end{document}
	\end{minted}
	\caption{Good practice for typesetting paragraphs}
	\label{lst:good-practice-for-paragraphs}
\end{listing}

Lastly, one thing you might notice is that I sometimes put sentences on their own line.
\LaTeX{} treats these sentences as being in the same paragraph because there is no blank line or \mintinline{latex}|\par| between them.
This isn't strictly necessary; you're more than welcome to put all of the sentences in a paragraph on one line, like I did in paragraph `five'.

However, there are two main reasons why it might be nice to put each sentence on its own line.
First, having shorter lines might be easier to read, depending on how your editor is set up.
Second, if you keep your \File{.tex} file under version control, it makes for cleaner diffs.%
\footnote{%
Explaining what exactly a version control system (VCS) is, is quite beyond the scope of this workshop.
Basically, it's a way to keep a history of all of the changes that have been made to a document.
If you delve any further into learning \LaTeX{} or learning to program, you will probably also want to learn a VCS at some point.
The most popular one is \mintinline{sh}|git|.
Also, hopefully you now understand this \href{https://xkcd.com/1285/}{joke}.%
}