% !TEX encoding = UTF-8 Unicode
% !TEX root = ../latex-workshop-for-linguists.tex

\section{General \LaTeX{} stuff}
\label{sec:general-latex-stuff}

\subsection{\LaTeX{} philosophy}
\label{subsec:latex-philosophy}

This subsection is very similar to \S3 of Alan Munn's \TitleText{A Beginner's Guide to \LaTeX{} (on the Mac)}.

\LaTeX{} was designed with the intent of separating content from formatting.
This is quite different from a what-you-see-is-what-you-get (WYSIWYG) editor like Word, where you see the output of your content formatted as you go along.

Something that goes hand in hand with separating content from formatting is that formatting should be given a semantics.
What does this mean?
(Bahhhh duhhhh chhhh!)
This means that if you want \LaTeX{} to format things that are similar in nature in the same way, then you should give them the same semantic meaning.

Alan gives the example of section headings.
The proper way to make a section heading in \LaTeX{} is to write it like \mintinline{latex}|\section{Section title}|.
The command \mintinline{latex}|\section{}| gives the content ``Section title'' the semantics of being a \Semantics{section}.
Then, if you want to change anything about how your sections are formatted, you can change this in the preamble of your document.
This contrasts with how many folks use Word where they would change this for each individual section heading.%
\footnote{%
Note that Word also allows for semantic markup despite the fact that most people do not use it.
If you cannot convince your Word-using friends and family to switch to \LaTeX, you should at least try to get them to use semantic markup in Word if they don't already do so.
\ASCIIEmoji{:)}%
}
For example, if you want the number preceding all of your section headings to be blue, you can change this at the beginning of the document, like in Listing~\ref{lst:blue-sections}.

\begin{listing}[htbp]
	\centering
	\begin{minted}{latex}
\documentclass{article}

\usepackage{color} % this package provides the command \color{}
\renewcommand\thesection{\color{blue}\arabic{section}}

\begin{document}

\section{Introduction}

Blah blah.

\section{Experiment}

Blah blah.

\section{Conclusion}

Blah blah.

\end{document}
	\end{minted}
	\caption{Example of semantic markup in \LaTeX{} for section headings}
	\label{lst:blue-sections}
\end{listing}

\subsection{Titles}
\label{subsec:titles}

To typeset a title, an author, and a date in a paper using the basic \Package{article} class, you can do what is shown in Listing~\ref{lst:basic-title}.

\begin{listing}[htbp]
	\centering
	\begin{minted}{latex}
\documentclass{article}

\title{Super awesome title}
\author{Best Author}
\date{July 22, 2015} % if you want today's date, replace July 22, 2015 with \today

\begin{document}

\maketitle

\section{Introduction}

Blah blah.

\end{document}
	\end{minted}
	\caption{A basic example of how to typeset the title of a paper using the \Package{article} class}
	\label{lst:basic-title}
\end{listing}

\subsection{Quotes and dashes}
\label{subsec:quotes-and-dashes}

One idiosyncrasy of \LaTeX{} that you will have to get used to is how to typeset quotes and dashes.

To typeset double open quotes, write \mintinline{latex}|``|.

To typeset double close quotes, write \mintinline{latex}|''|.

To typeset a single open quote, write \mintinline{latex}|`|.

To typeset a single close quote, write \mintinline{latex}|'|.

To typeset an en-dash, write \mintinline{latex}|--|.

To typeset an em-dash, write \mintinline{latex}|---|.

Note, however, that if you process your file with an engine that plays nicely with UTF-8 encoded documents (see \S\ref{subsec:fontspec-and-unicode}), you can enter these characters directly into your editor.

Doing this has the advantage of making your documents more readable, but note that it \emph{only} works if you use UTF-8 encoding and a compatible engine.
Also, it is good to know about the old way of typesetting these ligatures using \LaTeX{} since you will probably see many instances of this on the internet.

\subsection{Formatting text}
\label{subsec:formatting-text}

To typeset something in bold, use \mintinline{latex}|\textbf{}|.

To typeset something in italics, use \mintinline{latex}|\textit{}|.

To typeset something in small caps, use \mintinline{latex}|\textsc{}|.

To typeset something in a mono-spaced font, use \mintinline{latex}|\texttt{}|.

To underline something, don't.%
\footnote{%
Underlining is really frowned upon in the typography community.
I also personally do not like it.
However, if you have a really, really, really, really, really (really) good reason, then I suppose you can use \mintinline{latex}|\uline| from the \Package{ulem} package.
When you load, \Package{ulem} be sure to pass it the optional argument \mintinline{latex}|normalem|.%
}

One important thing to note about all of the foregoing commands is that using them directly in your document is \emph{not} good practice.
They are not semantic commands.
Let's consider an example.
Being a linguist, you will probably want to typeset certain glossed features in small caps, like \Nom, for instance.
You might think to write \mintinline{latex}|\textsc{nom}|.

This is \emph{bad practice}.
Instead, you should give glossed feature abbreviations like this a semantics, since you will presumably want to typeset them all the same way.
To do this, you could declare \mintinline{latex}|\newcommand*{\Fts}[1]{\textsc{#1}}| in your preamble.
Then, you would be able to write \mintinline{latex}|\Fts{nom}| instead of \mintinline{latex}|\textsc{nom}|.%
\footnote{%
This is just intended as an example of semantic markup.
I would not actually recommend typesetting glossed features in this manner; instead of defining your own \mintinline{latex}|\Fts| macro, you should use the \Package{leipzig} package.
This package provides commands for many of the common feature abbreviations, all following the \LGR, which is a standard for glosses in the field of linguistics.
In addition to providing glossing macros for all of the glosses in the \LGR, \Package{leipzig} also provides a way to automagically print all of the glosses used in a document and their meanings, via the \Package{glossaries} package.
This package will be covered in more detail in \S\ref{subsubsec:glossing-examples}.
\label{fn:leipzig}%
}

A further example can be seen if you look at the source code for this handout.
You will notice that I typeset all the names of packages in a mono-spaced font.
Rather than writing, for example, \mintinline{latex}|\texttt{forest}|, I have written \mintinline{latex}|\Package{forest}|.
In my preamble, I defined the \mintinline{latex}|\Package{}| command in the following way: \mintinline{latex}|\newcommand*{\Package}[1]{\texttt{#1}}|.
If I ever wanted to change how the name of every package is typeset in this document, I would only need to change it once in my preamble.
This is called semantic markup, and it makes any changes you might want to make to the formatting of a document much easier.

\subsection{Footnotes}
\label{subsec:footnotes}

To typeset footnotes, use \mintinline{latex}|\footnote{}|.

\subsection{Special characters}
\label{subsec:special-characters}

There are several characters that are treated as special characters in \LaTeX.
These are \mintinline{text}|# $ % & ~ _ ^ \ { }|.

If you ever want to print any of these characters in the output, you need to escape them with `\texttt{\textbackslash}'.%
\footnote{%
One exception to this is the escape character itself, `\texttt{\textbackslash}', because the sequence `\mintinline{text}|\\|' has a special meaning in \LaTeX, used for line breaks in tables.
If you wish to render the character `\texttt{\textbackslash}' you can use the command \mintinline{latex}|\textbackslash|.
Two further exceptions are \mintinline{text}|^| and \mintinline{text}|~|.
Preceding these two characters with `\texttt{\textbackslash}' is used for appending diacritics in \LaTeX{} (though see \S\ref{subsec:fontspec-and-unicode} for a better way of doing this).
If you wish to print these characters, you will need to do \mintinline{latex}|\^{}| and \mintinline{latex}|\~{}|, respectively (though see \href{http://tex.stackexchange.com/q/312/32888}{this post on TeX.SX} for suggestions of better ways to typeset a tilde).
\label{fn:diacritics}%
}

The character `\mintinline{text}|#|' is used for passing arguments to macros.

The character `\mintinline{text}|$|' is used for entering math mode (see \S\ref{subsec:math-mode}).

The character `\mintinline{text}|%|' is used for writing comments in the source document (cf.~Listing~\ref{lst:blue-sections}).

The character `\mintinline{text}|&|' is used for separating columns in a table (see \S\ref{subsec:tables}).

The character `\mintinline{text}|~|' is a non-breaking space.

The character `\mintinline{text}|_|' is used for subscripts in math mode.

The character `\mintinline{text}|^|' is used for superscripts in math mode.

The character `\mintinline{text}|\|' is the escape character.

The characters `\mintinline{text}|{ }|' are used for delimiting the arguments to commands.

\subsection{Math mode}
\label{subsec:math-mode}

One thing worth knowing about \LaTeX{} is that it has a distinct mode for typesetting math, creatively called math mode.
There is inline math, triggered by \mintinline{latex}|$...$|, and display math, triggered by \mintinline{latex}|\[...\]|.

For linguists, math mode is something that is mostly useful for typesetting semantics.%
\footnote{%
This is a bit of an overstatement.
There are other use cases.
Typesetting feature bundles in morphosyntax is one such use case, for example.
However, for the purposes of an introductory workshop, you should really just know what math mode is.%
}
For example, \mintinline{latex}|$\lambda x$| will produce $\lambda x$.

\subsection{Tables}
\label{subsec:tables}

Tables are admittedly a bit of a pain in the ass in \LaTeX.
Typesetting them takes a while to get used to.
Let's see an example of a basic table, such as the one in Listing~\ref{lst:basic-table}.

\begin{listing}[htbp]
	\centering
	\begin{minted}{latex}
\documentclass{article}
\begin{document}

\begin{tabular}{lcr}
    Left-aligned column & Center-aligned column & Right-aligned column \\ \hline
    56\%                & 75\%                  & 34\%                 \\
\end{tabular}

\end{document}
	\end{minted}
	\caption{A basic table in \LaTeX}
	\label{lst:basic-table}
\end{listing}

\begin{minipage}{\textwidth}

The code in Listing~\ref{lst:basic-table} will produce the following table.

\begin{center}
	\begin{tabular}{lcr}
		Left-aligned column & Center-aligned column & Right-aligned column \\ \hline
		56\%                & 75\%                  & 34\%                 \\
	\end{tabular}
\end{center}

\end{minipage}

One useful package for making aesthetically pleasing tables is the package called \Package{booktabs}.
It provides commands called \mintinline{latex}|\toprule|, \mintinline{latex}|\bottomrule|, and \mintinline{latex}|\midrule| for nicer horizontal rules in a table.

\begin{minipage}{\textwidth}

Consider Listing~\ref{lst:booktabs-table}, which produces the following output.

\begin{center}
	\begin{tabular}{lll}
		\toprule
		         & Passive sentences & Active sentences \\ \midrule
		Adults   & 99\%              & 98\%             \\
		Children & 56\%              & 87\%             \\
		\bottomrule
	\end{tabular}
\end{center}

\end{minipage}

\begin{listing}[htbp]
	\centering
	\begin{minted}{latex}
\documentclass{article}
\usepackage{booktabs}
\begin{document}

\begin{tabular}{lll}
    \toprule
             & Passive sentences & Active sentences \\ \midrule
    Adults   & 99\%              & 98\%             \\
    Children & 56\%              & 87\%             \\
    \bottomrule
\end{tabular}

\end{document}
	\end{minted}
	\caption{A table in \LaTeX{} using the package \Package{booktabs}}
	\label{lst:booktabs-table}
\end{listing}

\subsection{Images}
\label{subsec:images}

One thing you will often want to do is include images in a document.
This is what the package \Package{graphicx} is for.%
\footnote{%
Notice that the name of the package is \Package{graphicx} with an `x', not \Package{graphics}.
The \Package{graphics} package is different from \Package{graphicx}, and the example in Listing~\ref{lst:image-example} will not work if you use the \Package{graphics} package by accident.%
}
Let's look at the example in Listing~\ref{lst:image-example}.

\begin{listing}[htbp]
	\centering
	\begin{minted}{latex}
\documentclass{article}
\usepackage{graphicx}
\graphicspath{ {figure/} }
\begin{document}

\includegraphics[width=.8\textwidth]{super-important-graph}

\end{document}
	\end{minted}
	\caption{An example of including an image in a document}
	\label{lst:image-example}
\end{listing}

Notice that in the preamble of the document, we declared \mintinline{latex}|\graphicspath{ {figure/} }|.
This means that the package \Package{graphicx} will look in the folder called \texttt{figure} for images.%
\footnote{%
You don't need to do this.
If you prefer, you can just put the image file in the same folder as the \File{.tex} file.
The working directory (\IE{} the folder that the \File{.tex} file is in) is a place that the engine will always search when typesetting a document.
So if you're struggling with setting up a local \Directory{texmf} folder as described in \S\ref{subsec:local-files}, you can always just put files in the same folder as your \File{.tex} file for the time being.
But anyway, a reason you might want to have a separate folder dedicated for figures is to avoid clutter.
It's really up to you.%
}
So, in order to get this \File{.tex} file to compile, you would want to save it in a folder; then, in that same folder, you would want to create a new folder called \Directory{figure}.
Inside that folder, you should put the file \File{super-important-graph.pdf}.%
\footnote{%
The package \Package{graphicx} doesn't always play nicely with spaces and underscores in file names, so you should avoid using those things in the names of your image files.%
}
Notice that the file extension is omitted in the call to \mintinline{latex}|\includegraphics{}|.
This is a best practice because it allows \Package{graphicx} to try a bunch of different file extensions.

Notice also that we passed an optional argument to \mintinline{latex}|\includegraphics{}| in the form of a \textit{key val list}.
The key \KEY{width} can take a value that specifies what the width of the image that appears in the typeset document should be.
You could give it a value of \mintinline{latex}|6in| if you wanted, but it is often much more sensible to specify the width in terms of a dynamically defined value.
In this case, the super important graph will always occupy 80\% of the space allocated to the text, even if we change the margins of the document.

\subsection{Captioning and numbering}
\label{subsec:captioning-and-numbering}

Typesetting a table and including an image is great and all, but we want to be able to caption and number them.

\subsubsection{Floats}
\label{subsubsec:floats}

One common way to do this is to use floats.
In addition to automagically numbering tables and figures, floats also allow us to provide a caption.
One thing to know about floats is that \LaTeX{} has a special way of handling how they are typeset.
Suffice it to say, it's rather complicated.%
\footnote{%
You can read more about it \href{http://tex.stackexchange.com/q/39017/32888}{here} if you're interested.%
}
All you need to know is that \LaTeX{} has a special algorithm for placing floats in the best possible spot, according to general typographical standards.

These places are usually one of four places: right where they are written in the source document, the top of a page, the bottom of a page, or on their own separate page.
These four places correspond to four optional arguments that you can pass to a float environment, \mintinline{text}|htbp|, respectively.

It is generally best practice to pass all four options to a float, at least initially.
Only when you finish writing the document should you fiddle with the placement of floats if you think \LaTeX's algorithm has not done a good job.
However, while you're writing a document, leave all four options and let \LaTeX{} decide where floats should be placed.

If you would prefer to increase the likelihood that the float will show up in exactly the location that it is specified in the source \File{.tex} file, you can place a \mintinline{text}|!| after the \mintinline{text}|h|.

The command \mintinline{latex}|\caption{}| allows you to give a caption to the table or figure.
\LaTeX{} will automagically number the tables and figures in the correct order, so you don't have to worry about that.
Semantic markup FTW!

Take a look at the two examples in Listing~\ref{lst:float-examples}.
Try typesetting this yourself and see what the result is.

\begin{listing}[htbp]
	\centering
	\begin{minted}{latex}
\documentclass{article}
\usepackage{graphicx}
\usepackage{booktabs}
\begin{document}

This is a table that is a float.

\begin{table}[htbp]
    \centering
    \begin{tabular}{lll}
        \toprule
                 & Passive sentences & Active sentences \\ \midrule
        Adults   & 99\%              & 98\%             \\
        Children & 56\%              & 87\%             \\
        \bottomrule
    \end{tabular}
    \caption{Adult performance compared to child performance}
\end{table}

It might not actually show up in between these two sentences.

This is a figure that is a float.

\begin{figure}[htbp]
    \centering
    \includegraphics[width=.8\textwidth]{example-image-a}
    \caption{Super scientificy graphy thingy}
\end{figure}

It might not actually show up in between these two sentences.

\end{document}
	\end{minted}
	\caption{Examples of floats in \LaTeX}
	\label{lst:float-examples}
\end{listing}

\subsubsection{Non-float options}
\label{subsubsec:non-float-options}

It is worth mentioning some non-float options for tables and images.
One of the reasons that it is worth mentioning these options is because there is a common misconception that tables must go inside \mintinline{latex}|table| environments and images must go inside \mintinline{latex}|figure| environments.
This is not true.

When you're writing a paper, it is best to use floats because \LaTeX{} will use its algorithm to place the floats in the best environment.
However, it is not always appropriate to use floats.

One example of when you probably don't want to use floats is when you're making a handout.
In a handout, you usually want the image or the table to show up exactly where you place the code in the source document.

Nonetheless, in this case you might still want to be able to number and caption the figure or table.
The package \Package{capt-of} allows you to do this.
An example is given in Listing~\ref{lst:capt-of-example}.
Note that you will want to put things inside of an enclosing group, such as \mintinline{latex}|\begin{center}...\end{center}|.

\begin{listing}[htbp]
	\centering
	\begin{minted}{latex}
\documentclass{article}
\usepackage{graphicx}
\usepackage{booktabs}
\usepackage{capt-of}
\begin{document}

This is a table that is not a float.

\begin{center}
    \begin{tabular}{lll}
        \toprule
                 & Passive sentences & Active sentences \\ \midrule
        Adults   & 99\%              & 98\%             \\
        Children & 56\%              & 87\%             \\
        \bottomrule
    \end{tabular}
    \captionof{table}{Adult performance compared to child performance}
\end{center}

It will show up in between these two sentences no matter what.

This is a figure that is not a float.

\begin{center}
    \includegraphics[width=.8\textwidth]{example-image-a}
    \captionof{figure}{Super scientificy graphy thingy}
\end{center}

It will show up in between these two sentences no matter what.

\end{document}
	\end{minted}
	\caption{Examples of tables and images as non-floats}
	\label{lst:capt-of-example}
\end{listing}

Another case where you might want to not put a table inside of a float is when you're making a table for a particular morphological paradigm.
In linguistics, we usually number such tables just like we number other examples.
In \S\ref{subsec:examples}, we will see how to make numbered examples.
You can put a \mintinline{latex}|tabular| environment directly inside such an example.

\subsection{Cross referencing}
\label{subsec:cross-referencing}

So automagically numbered tables and images are great and all, but how do I refer to those things in my document?
One thing that is great about \LaTeX{} is that you can give things \mintinline{latex}|\label|s and \mintinline{latex}|\ref|er to them automagically as well.

Consider Listing~\ref{lst:cross-referencing}.

\begin{listing}[htbp]
	\centering
	\begin{minted}{latex}
\documentclass{article}
\usepackage{graphicx}
\begin{document}

As can be seen in Figure~\ref{fig:important-graph}, the results clearly show that I'm right.

\begin{figure}[htbp]
    \centering
    \includegraphics[width=.8\textwidth]{example-image-a}
    \caption{Super scientificy graphy thingy}
    \label{fig:important-graph}
\end{figure}

\end{document}
	\end{minted}
	\caption{An example of referencing a figure in \LaTeX}
	\label{lst:cross-referencing}
\end{listing}

There are a few things to say about cross-referencing.
First, and of particularly important note is the fact that the \mintinline{latex}|\label| comes after the \mintinline{latex}|\caption|.
If you try putting the \mintinline{latex}|\label| first, you will get the wrong number, because the command \mintinline{latex}|\caption| is what gives the float its number.

Second, you will notice that I've put a non-breaking space between \mintinline{text}|Figure| and \mintinline{latex}|\ref|.
This is generally a good practice because it prevents the number from being separated from the description of what it is.%
\footnote{%
If you really like automagic, you might want to check out the \href{http://ctan.org/pkg/cleveref}{\Package{cleveref}} package.%
}

Third, you will also notice that I've given the label a prefix of \mintinline{text}|fig:|.
This isn't strictly necessary, but it is good practice.
Imagine that you had a table and a figure.
The table contains specific values, and the figure is a graph of those values.
It's the same data, so you presumably want to give them similar names.
If you use prefixes like this, you could do \mintinline{latex}|\label{tab:super-important-results}| for the table and \mintinline{latex}|\label{fig:super-important-results}| for the graph.%
\footnote{%
Spaces are technically allowed in the names of \mintinline{latex}|\label|s, but they are highly discouraged.
In certain cases, spaces in \mintinline{latex}|\label| names might cause problems.%
}

Fourth, and most importantly, is that you must compile your document twice in order for this to work.
\LaTeX{} does its automagic by first processing the file and automagically generating all of the table and figure numbers for each float.
Remember that you never manually gave each float a number, so \LaTeX{} has to figure this out.
On the second compilation, it inserts the automagically generated numbers into the places where you \mintinline{latex}|\ref|erenced them.
If something went wrong, you will see question marks instead of a number.
This most likely means you either only compiled your document once or you have tried to refer to something that you never actually labeled.
The most common example of this latter reason is just a simple misspelling of the label that you gave to whatever it is that you're trying to reference.

\subsection{Those annoying files}
\label{subsec:those-annoying-files}

One thing you will quickly notice when you typeset a \File{.tex} file is that a lot of extra files are generated.
People tend to initially find this annoying, but it is all of these extra files that allow \LaTeX{} to do its magic.
For example, the auxiliary file (\File{.aux}) is integral for cross referencing.
Without it, cross referencing just would not work.

To avoid clutter, it's often a good idea to make a new, self-contained folder for each document that you typeset.

One other thing worth noting about all of these extra files is that they sometimes lead to compilation errors.
If you introduced an error in your document and you tried typesetting it, it's possible that the extra files got messed up.
So, if you tried typesetting your document, received a compilation error, figured out what caused the error, and you're like 110\% sure that you fixed the problem in your \File{.tex} file, but you're still getting a compilation error, try deleting all of these extra files and compiling the document again.

\subsection{Bibliographies}
\label{subsec:bibliographies}

Getting a bibliography to work with \LaTeX{} is often one of the big hurdles of learning \LaTeX, but once you've figured it out, it's really, really, really frikken nice.

In order for this to work correctly, you will first need to set up a local \Directory{texmf} folder.
How you do this depends on which \TeX{} distribution you have.
There are explicit instructions in \S\ref{subsec:local-files}.
If you have not already set up a local \Directory{texmf} folder on your machine, go there and follow the instructions.

\subsubsection{Overview}
\label{subsubsec:overview}

There are several components that are relevant to setting up your machine in order to be able to use a single master bibliography file (\File{.bib}) for all of your references in all of your \File{.tex} documents.

\begin{description}

	\item[The \File{.bib} file]{%
		You need to create a bibliographic database that contains all of the information for all of the references that you wish to cite in your \File{.tex} documents.
		This file is just a plain text file, but there are graphical user interfaces that can be used to edit the file, such as \href{http://jabref.sourceforge.net/}{JabRef} or \href{http://bibdesk.sourceforge.net/}{BibDesk}.%
		\footnote{%
		BibDesk is only available on a Mac.
		It is installed by default when you install {Mac\TeX}, so it is probably already on your computer.%
		}
		How to manage this file and other relevant aspects will be discussed in detail in \S\ref{subsubsec:the-bib-file}.%
	}

	\item[The citation package]{%
		Strictly speaking, you do not need to use a citation package.
		The base \hologo{LaTeX} format already provides the command \mintinline{latex}|\cite{...}|.
		However, citation packages provide much more flexibility and many more features than what is offered by the base \hologo{LaTeX} format.
		Citation packages are also great for easily changing the style that your references are printed in.
		They also often provide more specific commands for inline citations and parenthetical citations.
		Some examples of citation packages include \Package{cite}, \Package{apacite}, \Package{natbib}, and \Package{biblatex}.
		For the purposes of linguistics at least, I would only recommend \Package{natbib} or \Package{biblatex}.
		Both of these packages will be discussed in more detail in \S\ref{subsubsec:the-citation-package}.%
	}

	\item[The backend processor]{%
		In addition to using a citation package, you will also need to use a program for getting the relevant citations from your \File{.bib} file.
		Using a \File{.bib} file for references involves a series of compilation steps.
		You must first compile your \File{.tex} file to produce either an \File{.aux} or \File{.bcf} file.
		Then you must process the \File{.aux} or \File{.bcf} file with a backend processor.
		The two backend processors on the market are \hologo{BibTeX} and Biber.
		Processing the \File{.aux} or \File{.bcf} goes through and matches up the things that were cited in your \File{.tex} file with entries in your \File{.bib} file.
		The differences between the two backend processors will be discussed in more detail in \S\ref{subsubsec:the-backend-processor}.
		The necessary compilation steps will actually be discussed in \S\ref{subsubsec:the-citation-package} because it depends on the citation package that you use, not the backend processor.%
	}

\end{description}

\subsubsubsection{A terminological note}
\label{subsubsubsec:a-terminological-note}

Before proceeding any further, it's worth touching briefly upon a bit of history to hopefully avoid some terminological confusion.
However, what is perhaps more likely, is that this brief excursion won't make any sense.
If that's the case, don't worry.
That's not your fault.
The history of \TeX{} and the evolution of relevant terminology is complicated.

Hopefully the terminology won't cause too much confusion, but I would nonetheless like to make note of it.
If it doesn't make sense now, perhaps it will in 6 months.

For a while, the only option for a backend processor was \hologo{BibTeX}.
\hologo{BibTeX} was developed in the ancient days when memory and hard drive space were expensive, character encodings were limited, and dinosaurs still roamed the earth.

Biblatex is a recent and modern alternative to \hologo{BibTeX}.
People sometimes use Biblatex to refer to the combination of the citation package \Package{biblatex} and the backend processor Biber.
I will adopt this practice of using Biblatex to refer to both of these things in combination, whereas \Package{biblatex} will only be used to refer to the citation package.

What makes this practice particularly confusing, is that many people use Biblatex in contradistinction to \hologo{BibTeX}.
The more accurate contrast, however, would be to contrast Biber with \hologo{BibTeX}.

For better or worse, however, this terminological parlance is quite common among \TeX{} users and is probably here to stay.
One thing that is actually useful about this seemingly confusing terminology, however, is that there are differences between how you write a \File{.bib} file that is intended to be used with Biblatex and how you write a \File{.bib} file that is intended to be used with \hologo{BibTeX}.

In other words, it can actually be useful to distinguish Biblatex and \hologo{BibTeX} if you want to know how a \File{.bib} file was prepared, for example.
Biblatex supports a wider range of entry types and a wider range of data fields and is thus much more flexible and versatile than \hologo{BibTeX}.%
\footnote{%
You will be introduced to entry types in \S\ref{subsubsubsec:entry-types} and data fields in \S\ref{subsubsubsec:data-fields}.%
}
Biblatex also plays much nicer with accented Latin characters and non-Latin alphabets than does \hologo{BibTeX}.

A full discussion of these differences is beyond the scope of this document.
Some of the differences will be briefly discussed below, but for further discussion, the reader is referred to the following resources.

\begin{itemize}

	\item{The question \href{http://tex.stackexchange.com/q/25701/32888}{bibtex vs.~biber and biblatex vs.~natbib} and its answers on TeX.SX}

	\item{The question \href{http://tex.stackexchange.com/q/5091/32888}{What to do to switch to biblatex?} and its answers on TeX.SX}

	\item{The question \href{http://tex.stackexchange.com/q/37095/32888}{Compatibility of bibtex and biblatex bibliography files?} and its answers on TeX.SX}

	\item{\S\S1--3 of the \href{http://texdoc.net/texmf-dist/doc/latex/biblatex/biblatex.pdf}{\Package{biblatex} documentation}}

\end{itemize}

\subsubsection{The \File{.bib} file}
\label{subsubsec:the-bib-file}

In order to get started, you need to create a \File{.bib} file.
This is just a plain text file, but your life will probably be made much easier if you use a graphical user interface to edit the file.

\href{http://jabref.sourceforge.net/}{JabRef} is a crossplatform option.
Follow the link to download and install it.

If you have a Mac, you can use \href{http://bibdesk.sourceforge.net/}{BibDesk}.
It should already be installed if you installed {Mac\TeX}.

Another thing that is nice about these graphical user interfaces is that they allow you to pair \File{.pdf} files to the entries in your database.
If you pair a \File{.pdf} with an entry, then you can open JabRef or BibDesk and click on the link to the \File{.pdf} file in order to open it without having to find it on your computer.

Before creating a database, you will want to make sure that you save the \File{.bib} file with UTF-8 encoding.
For some reason, the default character encoding in JabRef (on Windows at least) is Cp1252.
This will just give you a lot of headaches.
Before you create your \File{.bib} file, go into the JabRef preferences and change the default character encoding to UTF-8.

You should do the same with BibDesk.
Before creating your \File{.bib} file, open BibDesk, go to preferences and set the default character encoding to UTF-8.

If you have trouble finding these options in either BibDesk or JabRef, \href{https://xkcd.com/627/}{Google it}. \ASCIIEmoji{:p}

Once you have changed the default encoding, create a new bibliography file (\File{.bib}).
It is \emph{very important} that you save it in the correct location.
It \emph{must} go inside your local \Directory{texmf} folder; specifically, it \emph{must} go inside \Directory{texmf/bibtex/bib}.
If you do not have a local \Directory{texmf} folder set up, go read and follow the instructions in \S\ref{subsec:local-files}.

Moreover, you will also save yourself a lot of pain if you \emph{do not use spaces} in the name of the file.
I would recommend something simple like \File{master.bib} or \File{linguistics.bib}.

Now that you have your \File{.bib} file setup, you can start adding entries to it.

\subsubsubsection{Entry types}
\label{subsubsubsec:entry-types}

There are various entry types, and the available entry types differ slightly depending on whether you use \Package{biblatex} or not.%
\footnote{%
\label{fn:entry-types-biblatex-vs-bibtex}%
Covering this in any detail is really beyond the scope of this document.
For more information, please see the resources mentioned in \S\ref{subsubsubsec:a-terminological-note}.
For the most part, however, you do not need to worry about it.
A \File{.bib} file that was prepared for use with \hologo{BibTeX} will work just fine with Biblatex.
You just will not be able to use some of the more flexible and versatile features of Biblatex (which you probably won't need anyway, except for complicated use cases).%
}
In general, you will probably mostly only need to use the entry types \EntryType{@article}, \EntryType{@book}, \EntryType{@incollection}, \EntryType{@inproceedings}, \EntryType{@misc}, and \EntryType{@unpublished}.

\begin{description}

	\item[\EntryType{@article}]{%
		This entry type should be used for journal articles.%
	}

	\item[\EntryType{@book}]{%
		This entry type should be used for complete books.%
	}

	\item[\EntryType{@incollection}]{%
		This entry type should be used for chapters in edited volumes.
		Do \emph{not} use the \EntryType{@inbook} entry type for this.%
	}

	\item[\EntryType{@inproceedings}]{%
		This entry type should be used for stuff that is published in conference proceedings.%
	}

	\item[\EntryType{@misc}]{%
		This entry type should be used for miscellaneous stuff, such as talks or posters presented at conferences (that aren't published in proceedings).
		You can add something to the \DataField{howpublished} field that indicates where it was presented and whether it was a talk or a poster.
		See \S\ref{subsubsubsec:data-fields} for more details.%
	}

	\item[\EntryType{@unpublished}]{%
		This entry type can be used for unpublished manuscripts.%
	}

\end{description}

\subsubsubsection{Data fields}
\label{subsubsubsec:data-fields}

Inside of each entry, there are certain data fields that you need to fill out.
The available data fields depend on which entry type you are using.%
\footnote{%
The available data fields also depend on whether you are preparing your \File{.bib} file for use with Biblatex or \hologo{BibTeX} (cf.~fn.~\ref{fn:entry-types-biblatex-vs-bibtex}).
A full discussion of the many more data fields that Biblatex provides is much beyond the scope of this document.
You largely do not need to worry about it unless you would like to use some of Biblatex's more advanced features.
The data fields discussed here are for a \hologo{BibTeX} \File{.bib} file, but they will all work just fine if you use Biblatex instead of \hologo{BibTeX}.%
}
In general, the bare minimum that you need to fill in are the data fields \DataField{author}, \DataField{title}, and \DataField{year}.

For each of the entry types mentioned above, here are the recommended data fields that I would minimally suggest filling in.

\begin{multicols}{3}
\begin{itemize}

	\item{\EntryType{@article}

		\begin{itemize}

			\item{\DataField{author}}

			\item{\DataField{title}}

			\item{\DataField{journal}}

			\item{\DataField{year}}

			\item{\DataField{volume}}

			\item{\DataField{number}}

			\item{\DataField{pages}}

			\item{\DataField{doi}}

		\end{itemize}

	}

	\item{\EntryType{@book}

		\begin{itemize}

			\item{\DataField{author}}

			\item{\DataField{title}}

			\item{\DataField{publisher}}

			\item{\DataField{year}}

			\item{\DataField{address}}

		\end{itemize}

	}

	\item{\EntryType{@incollection}

		\begin{itemize}

			\item{\DataField{author}}

			\item{\DataField{title}}

			\item{\DataField{booktitle}}

			\item{\DataField{publisher}}

			\item{\DataField{year}}

			\item{\DataField{editor}}

			\item{\DataField{pages}}

			\item{\DataField{address}}

		\end{itemize}

	}

	\item{\EntryType{@inproceedings}

		\begin{itemize}

			\item{\DataField{author}}

			\item{\DataField{title}}

			\item{\DataField{booktitle}}

			\item{\DataField{publisher}}

			\item{\DataField{year}}

			\item{\DataField{pages}}

			\item{\DataField{address}}

		\end{itemize}

	}

	\item{\EntryType{@misc}

		\begin{itemize}

			\item{\DataField{author}}

			\item{\DataField{title}}

			\item{\DataField{howpublished}}

			\item{\DataField{year}}

		\end{itemize}

	}

	\item{\EntryType{@unpublished}

		\begin{itemize}

			\item{\DataField{author}}

			\item{\DataField{note}}

			\item{\DataField{title}}

			\item{\DataField{year}}

		\end{itemize}

	}

\end{itemize}
\end{multicols}

\paragraph{Keywords}

Another useful data field is \DataField{keywords}.
This can be used with every entry type, and I would highly recommend filling it in for every entry.

You can make up the keywords that you use.
Doing so allows you to easily sort and search through your \File{.bib} file.

For example, you could put keywords in all of your entries for different subdisciplines, like syntax, or semantics, or phonology, \ETC.
If you wanted to look through your \File{.bib} file for a certain syntax paper that you cannot remember, for example, then you could just look at all of the entries that contain syntax in the \DataField{keywords} data field.

\paragraph{On names}

The method for entering names into the \DataField{author} and \DataField{editor} data fields is a bit idiosyncratic, but it allows names to be typeset in a very fine-grained manner, so it is well worth it.

There are four components to a name.

\begin{enumerate}[label={(\roman*)}]

	\item{First name (this includes any and all middle names)}

	\item{The ``von'' part (examples include ``von'', ``van'', ``de'', ``de la'', \ETC)}

	\item{The last name (does \emph{not} include the ``von'' part)}

	\item{The ``Jr'' part (examples include ``Jr.'', ``Sr.'', ``IV'', \ETC)}

\end{enumerate}

There are three possible ways to enter names into a \File{.bib} database.

\begin{enumerate}[label={(\roman*)}]

	\item{``First von Last''}

	\item{``von Last, First''}

	\item{``von Last, Jr, First''}

\end{enumerate}

I would highly recommend always using either the second or third option.
The first option does not work if an author has multiple \emph{last} names and no ``von'' part in their name.

Finally, the last thing to know about names is that multiple authors or editors should be separated with \BibliographyData{and}.

So, for example, if something you wish to cite has three authors, you should write the following in the \DataField{author} data field: \BibliographyData{Matthewson, Lisa and von Fintel, Kai and Smith, Jr., Mary}.

\paragraph{On casing}

Depending on the style that you use for your bibliography, you might see your titles being typeset in \href{https://en.wiktionary.org/wiki/sentence_case}{sentence casing} rather than \href{https://en.wiktionary.org/wiki/title_case}{title casing}.
Sentence casing is what's called for by the \href{http://celxj.org/downloads/UnifiedStyleSheet.pdf}{Unified Stylesheet for Linguistics Journals}, for example.

Nonetheless, there might be certain things that ought to remain capitalized.
In this case, you should surround these things with braces in the data field.

For example, if you are adding an \EntryType{@incollection} entry type whose \DataField{title} is \TitleText{On the absence of certain quantifiers in Mohawk}, you should enter it into the \DataField{title} field as \BibliographyData{On the Absence of Certain Quantifiers in \{Mohawk\}}.
Note that the braces should surround the entire word, \emph{not} just the letter M.
If the braces do not surround the entire word, it will mess up the kerning.

Note furthermore that the data field was written using title casing, but the output you see will be sentence casing, except for the word \xv{Mohawk}, because we surrounded it with braces.
I would recommend always entering titles into the data fields using title casing.
It is much easier to automagically convert title casing to sentence casing than it is to automagically convert sentence casing to title casing.
Thus, you will save yourself much pain if you need to switch to a style that uses title casing.

\paragraph{On cite keys}

In addition to filling out all of the data fields, you will also need to give each entry a unique cite key.
This is what you will use to cite the entry in one of your \File{.tex} documents.

You can come up with your own algorithm for determining cite keys, but it is best to use some sort of algorithm rather than making up arbitrary cite keys as you go along.
Here are two suggestions.

\begin{enumerate}[label={(\roman*)}]

	\item{\BibliographyData{lastnameYYYY}}

	\item{\BibliographyData{lastnameYYYY:informativewordfromtitle}}

\end{enumerate}

The first one is very basic.
You could just use the last name of the first author and the four-digit year of the publication for the cite key.
You will of course run into trouble when an author has multiple publications in the same year.
In this case, you could do something like \BibliographyData{lastnameYYYYa} and \BibliographyData{lastnameYYYYb} to disambiguate the two publications.

However, it might be preferable to do something like the second option instead.
If you have to change the cite key after you've already added it and used it, then you would need to go back and change how you cited it in your \File{.tex} file.
If you use the second suggested algorithm instead, it's very unlikely that you will run into a situation where you might have duplicate cite keys for distinct entries.

\subsubsubsection{Lingbib}
\label{subsubsubsec:lingbib}

However, instead of going through all the trouble of maintaining your own bibliography file \ldots{} why not use and contribute to \href{https://github.com/lingbib/lingbib}{Lingbib}!?

This is a (shameless plug for a) project that Kenneth Hanson and I are currently working on developing.
We think it's rather wasteful that people individually create bibliography files and maintain them on their own.
Instead, we could all use and contribute to one central bibliography file for the entire field.

This has at least the following two advantages. First, it saves everyone time, and, second, it ensures greater accuracy in the \File{.bib} file since more eyes will be looking at it.

The project isn't quite ready for use yet, but I will update this \File{.pdf} when it is ready to use.
We would greatly appreciate it if you spread the word about this project, started using it, and perhaps even contributed! \ASCIIEmoji{:)}

\subsubsection{The citation package}
\label{subsubsec:the-citation-package}

The two main options for citation packages that are recommended are \Package{natbib} and \Package{biblatex}.
There are others, but they are nowhere near as versatile as these two.

\subsubsubsection{\Package{natbib}}
\label{subsubsubsec:natbib}

\Package{natbib} is only compatible with using \hologo{BibTeX} as the backend processor for your \File{.aux} and \File{.bib} files (see \S\ref{subsubsec:the-backend-processor}).
Since \hologo{BibTeX} was developed a long time ago, it does not play all that nicely with accented Latin characters and non-Latin alphabets.

I would highly recommend using \Package{biblatex} and Biber instead of \Package{natbib} and \hologo{BibTeX}.
However, it is worth noting that academic journals that accept \LaTeX{} submissions often require bibliographies that use \Package{natbib} and \hologo{BibTeX}.%
\footnote{%
A notable exception to this is \href{http://semprag.org/}{\Journal{Semantics \& Pragmatics}}, which accepts submissions that use \Package{biblatex} and Biber.%
}

\paragraph{Citation commands}

There are two main commands for citations with \Package{natbib} that you ought to know about.
You can use \mintinline{latex}|\citet[][]{...}| for inline citations and \mintinline{latex}|\citep[][]{...}| for parenthetical citations.

The first optional argument of both commands can be used to specify a prenote, and the second optional argument can be used to specify page numbers.

There is also a handy \href{http://merkel.zoneo.net/Latex/natbib.php}{\Package{natbib} reference sheet} that describes more of the citation commands.

\paragraph{Styles}

\Package{natbib} and \hologo{BibTeX} use \File{.bst} files to specify different bibliography styles.
\File{.bst} files are awful.

Thankfully, Bridget Samuels has gone through the trouble of creating one for us that conforms to the \href{http://celxj.org/downloads/UnifiedStyleSheet.pdf}{Unified Stylesheet for Linguistics Journals}.
If you decide to use \Package{natbib} and \hologo{BibTeX}, you should download the file \href{http://celxj.org/downloads/unified.bst}{\File{unified.bst}} and place it in your local \Directory{texmf} directory.
Specifically, it needs to go into the folder \Directory{texmf/bibtex/bst} (see \S\ref{subsec:local-files} if you've somehow made it this far without setting up a local \Directory{texmf} folder).

There are many other styles that you can use out of the box with either a TeX Live/{Mac\TeX} or \hologo{MiKTeX} distribution, such as \Package{alpha} or \Package{apalike}.
But, presumably, you will want to use the \Package{unified} style that Bridget Samuels wrote since you're (presumably) a linguist.

\paragraph{Compilation steps}

In order for the references to be handled automagically in your document, you need to go through a series of compilation steps.
This series of steps might depend on the citation package that you use and perhaps also the style.

In most cases when using \Package{natbib}, it is necessary to do four compilation steps.

\begin{enumerate}[label={(\roman*)}]

	\item{\mintinline{sh}|latex myfile.tex|}

	\item{\mintinline{sh}|bibtex myfile.aux|}

	\item{\mintinline{sh}|latex myfile.tex|}

	\item{\mintinline{sh}|latex myfile.tex|}

\end{enumerate}

In these four steps, you should treat \Engine{latex} as a placeholder for whatever engine you are using.
For example, if you are using \hologo{pdfLaTeX}, then you should replace \Engine{latex} with \Engine{pdflatex} in these four steps.
Similarly, if you are using \hologo{XeLaTeX}, then you should replace \Engine{latex} with \Engine{xelatex} in these four steps.

These commands can be run from the command line, but, more likely, you will just run these commands from inside of your editor.

Note that the file extensions are optional.
In fact, it's probably better practice to omit them and let the program try to guess which file to process.
I've only included them here to draw your attention to the fact that the \File{.aux} file is what you should process with \hologo{BibTeX}, \emph{not} the \File{.tex} file or the \File{.bib} file.

Again, if you omit the \File{.aux} file extension, \hologo{BibTeX} will do things correctly, so just go ahead and omit it if you do do your build steps on the command line.
More likely, you will just run these commands from within your editor, in which case you won't need to worry about the file extensions.

The first pass of \Engine{latex} will collect all of the cite keys that you are trying to cite inside of your \File{.tex} document and put them in the \File{.aux} file.

The pass of \Engine{bibtex} over the \File{.aux} file goes and gets the citations from your \File{.bib} file that match the cite keys in the \File{.aux} file.

The second pass of \Engine{latex} inserts the citations into the generated output, and the third pass of \Engine{latex} resolves any cross references that have changed.

While these series of compilation steps may seem a bit annoying, it is necessary for \LaTeX{} to do its automagic.%
\footnote{%
See \S\ref{sec:things-to-learn-on-your-own} for some suggestions on how to automate the compilation process.%
}
You will definitely come to appreciate how \LaTeX{} handles bibliographies and references compared to Microsoft Word.

\paragraph{Complete minimal working example}

Listing~\ref{lst:natbib-example} gives a complete minimal working example of how to use \Package{natbib} and \hologo{BibTeX}.

There are a few things to note about this example.

First, the bibliography file and the style file are specified at the end of the document, just before \mintinline{latex}|\end{document}|.%
\footnote{%
Technically, you can put the \mintinline{latex}|\bibliographystyle| command anywhere in your document, but I like to keep it next to the \mintinline{latex}|\bibliography| command, which is what prints the references section and thus must come at the end of the document.%
}
In both cases, be sure to omit the file extensions (\IE{} do \emph{not} add \File{.bib} or \File{.bst}).
You can use multiple bibliography files, but you should only have one call to the \mintinline{latex}|\bibliography| command.
If you need to use multiple \File{.bib} files, you should pass a comma-separated list to the \mintinline{latex}|\bibliography| command, such as \mintinline{latex}|\bibliography{linguistics,philosophy}|.
This command would allow you to use the two bibliography files \File{linguistics.bib} and \File{philosophy.bib}.

Second, the \File{.bib} file is bundled into this example so that you can copy and paste it into an editor and compile it directly.
But this is \emph{not} how you should write your \File{.tex} files.
You should instead use a single master bibliography file as discussed in \S\ref{subsubsec:the-bib-file}.

In this case, you would replace \mintinline{latex}|\bibliography{\jobname}| with \mintinline{latex}|\bibliography{master}|, if, for example, you named your \File{.bib} file \File{master.bib}.
Moreover, you would also then completely delete everything from \mintinline{latex}|\begin{filecontents}| to \mintinline{latex}|\end{filecontents}|.

Try compiling this example using what you just learned above about how to compile a \File{.tex} document that uses \Package{natbib} for citations.

\begin{listing}[htbp]
	\centering
	\begin{minted}{latex}
\documentclass{article}

\begin{filecontents}{\jobname.bib}
@book{chomsky1995:MP,
    Address = {Cambridge, MA},
    Author = {Chomsky, Noam},
    Publisher = {The MIT Press},
    Title = {The Minimalist Program},
    Year = {1995}}
\end{filecontents}

\usepackage{natbib}

\begin{document}

In Minimalist syntax, S-Structure has been eliminated \citep[see][73--124]{chomsky1995:MP}.

\bibliography{\jobname}
% the following will only work if unified.bst is in your local texmf folder
% if you haven't downloaded that file yet, just replace unified with
% apalike and then try compiling this example
\bibliographystyle{unified}
\end{document}
	\end{minted}
	\caption{Complete minimal working example showing how to use \Package{natbib} and \hologo{BibTeX}}
	\label{lst:natbib-example}
\end{listing}

\subsubsubsection{\Package{biblatex}}
\label{subsubsubsec:biblatex}

Unlike \Package{natbib}, \Package{biblatex} can be used with either backend processor: Biber or \hologo{BibTeX}.
If you wish to use the full range of features that \Package{biblatex} provides, you will need to use Biber.

As mentioned above, there is good reason to use \Package{biblatex} and Biber over \Package{natbib} and \hologo{BibTeX}.
Most notably, \Package{biblatex} and Biber work a lot more straightforwardly with accented Latin characters and non-Latin alphabets.

In addition to not being an acceptable format for submission to many academic journals, the only available implementation of the \href{http://celxj.org/downloads/UnifiedStyleSheet.pdf}{Unified Stylesheet for Linguistics Journals} in \Package{biblatex} is currently very early in development.
However, at the time of the most recent update to this \File{.pdf}, the implementation is much better than it was a few months ago (more on this below).

\paragraph{Citation commands}

\Package{biblatex} provides \mintinline{latex}|\textcite[][]{...}| for inline citations and \mintinline{latex}|\autocite[][]{...}| for non-inline citations.
Like with the \Package{natbib} citation commands, the first optional argument can be used for a prenote, and the second optional argument can be used to specify specific page numbers.

One thing that is nice about \Package{biblatex} is that the \mintinline{latex}|\autocite| command is a high-level command for non-inline citations.
In the case of linguistics, at least, you will usually want this to be a parenthetical citation.
However, some styles call for footnotes for non-inline citations instead of parenthetical citations.
If you use \Package{biblatex} and the \mintinline{latex}|\autocite| command, all you have to do is change the style, and this will be handled automagically for you.

To learn more about these citation commands and others that \Package{biblatex} provides, see \S3.7 of the \href{http://texdoc.net/texmf-dist/doc/latex/biblatex/biblatex.pdf}{\Package{biblatex} documentation}.

\paragraph{Styles}

\Package{biblatex} provides many predefined styles that control both the citation style and the bibliography style.
The citation style is controlled by a \File{.cbx} file, and the bibliography style is controlled by a \File{.bbx} file.

The style options are declared globally when the \Package{biblatex} package is loaded.
Assuming that a style provides both a \File{.cbx} and a \File{.bbx} file with the same name, you can declare these at the same time by setting \PackageOption{style} to the name of the style that you wish to use.

If you wish to mix and match citation and bibliography styles, you can instead declare them separately, with \PackageOption{citestyle} and \PackageOption{bibstyle} respectively.
For an example, see Listing~\ref{lst:biblatex-non-complete-example}.

As mentioned above, the one implementation of the \href{http://celxj.org/downloads/UnifiedStyleSheet.pdf}{Unified Stylesheet for Linguistics Journals} for \Package{biblatex} is currently very much in an alpha stage of testing.
This implementation is maintained by \href{http://semprag.org/}{\Journal{Semantics \& Pragmatics}}, and it is available \href{https://github.com/semprag/biblatex-sp-unified}{here}.

At the time of the most recent update to this \File{.pdf} (\today), the implementation is much better than it was a few months ago.
If you would like to use \Package{biblatex} (which I would recommend for the reasons already discussed), I would recommend using \href{https://github.com/semprag/biblatex-sp-unified}{\Journal{S\&P}'s implementation} of the \href{http://celxj.org/downloads/UnifiedStyleSheet.pdf}{Unified Stylesheet for Linguistics Journals}.

Nonetheless, if you don't want to download it and put it in a local \Directory{texmf} folder, then you can use the \PackageOption{authoryear-comp} citation style and the \PackageOption{authoryear} bibliography style that are both provided by \Package{biblatex} out of the box.
The code given in Listing~\ref{lst:biblatex-non-complete-example}, which shows how to set these styles when loading \Package{biblatex}, will produce an approximation of the \href{http://celxj.org/downloads/UnifiedStyleSheet.pdf}{Unified Stylesheet for Linguistics Journals} that is probably good enough for all purposes other than submitting to a journal.%
\footnote{%
And, I mean, bibliography and citation styles are just prescriptivist nonsense anyway.
As long as all of the information is there, the style doesn't call for underlining, and the formatting is consistent, who cares \ldots{} ?%
}

\begin{listing}[htbp]
	\centering
	\begin{minted}{latex}
\documentlcass{article}
\usepackage[
    backend=biber,% you can use bibtex instead of biber if you want
    citestyle=authoryear-comp,
    bibstyle=authoryear
]{biblatex}
% This is needed because the default authoryear style puts "In:"
% before a journal name, which is not standard in linguistics styles.
% So this code gets rid of that.
\renewbibmacro{in:}{%
  \ifentrytype{article}{}{\printtext{\bibstring{in}\intitlepunct}}}

\begin{document}

The preamble of this document shows how to approximate a linguistics style.

\end{document}
	\end{minted}
	\caption{Setting up citation and bibliography styles with \Package{biblatex}}
	\label{lst:biblatex-non-complete-example}
\end{listing}

The code in Listing~\ref{lst:biblatex-non-complete-example} only shows how to set up the style.
For a complete minimal working example, see Listing~\ref{lst:biblatex-example}.

\paragraph{Compilation steps}

Just like with \Package{natbib}, \Package{biblatex} requires a certain series of compilation steps in order for everything to work.
Again, the steps that are necessary might depend on certain things, such as which style you use.%
\footnote{%
For example, I think citation styles that do non-inline citations as footnotes might require a third pass of \Engine{latex}.%
}

In most cases when using \Package{biblatex}, it is sufficient to do the following three compilation steps.

\begin{enumerate}[label={(\roman*)}]

	\item{\mintinline{sh}|latex myfile.tex|}

	\item{\mintinline{sh}|biber myfile.bcf| OR \mintinline{sh}|bibtex myfile.aux|%
	\footnote{%
	\label{fn:backends-with-biblatex}%
	As discussed above and as will be discussed further below in \S\ref{subsubsec:the-backend-processor}, it is possible to use either backend with \Package{biblatex}.
	Because Biber has many advantages over \hologo{BibTeX}, I would always recommend using Biber.
	Nonetheless, if you choose to use \hologo{BibTeX} with \Package{biblatex} for whatever reason, then you will need to run \Engine{bibtex} on the \File{.aux} file instead of running \Engine{biber} on the \File{.bcf} file.%
	}
	}

	\item{\mintinline{sh}|latex myfile.tex|}

\end{enumerate}

Just like we discussed above for \Package{natbib}, you should treat \Engine{latex} as a placeholder for whatever engine you are using.
So, for example, if you are using \hologo{pdfLaTeX}, then you should replace \Engine{latex} with \Engine{pdflatex} in these three steps.
Similarly, if you are using \hologo{XeLaTeX}, then you should replace \Engine{latex} with \Engine{xelatex} in these three steps.

Likewise, these are commands that you could enter into a terminal, but in most cases you will just run them from within your editor.

Were you to run these commands in a terminal, a further similarity to what we discussed above for \Package{natbib} is that you could leave off the file extensions.
In fact, I would encourage you to leave off the file extensions if you do end up doing your build steps inside of a terminal.

I have only shown the file extensions here in order to draw your attention to the fact that Biber processes a \File{.bcf} file rather than the \File{.aux} file.
Rather than having to remember, this, though, you could just leave off the file extension and let the program guess which file to process.

The first pass through \Engine{latex} is necessary to generate the \File{.bcf} file.%
\footnote{%
Or the \File{.aux} file if you're using \hologo{BibTeX} as the backend (cf.~fn.~\ref{fn:backends-with-biblatex}).
}
This serves a similar purpose to the \File{.aux} file in the case of \hologo{BibTeX}.
Specifically, it allows Biber to get all of the citations from your \File{.bib} file that match the cite keys that you used in your \File{.tex} document.

The second pass through \Engine{latex} typesets all of the references.%
\footnote{%
Unlike with \Package{natbib}, you do not need a third pass through \Engine{latex} in most cases.
This is because \Package{biblatex} loads all of the references into the preamble of your document, and so it can determine the cross references before processing the document.
In other words, you do not need a third pass through \Engine{latex} (in most cases) to sort out any cross references that have changed, like you do with \Package{natbib}.%
}

\paragraph{Complete minimal working example}

In Listing~\ref{lst:biblatex-example} you can see a complete minimal working example.
Like with the complete minimal working example above for \Package{natbib}, the \File{.bib} file is bundled into this example only so that you can copy and paste it into an editor and try compiling it yourself.
But this is \emph{not} how you should write your \File{.tex} files.

You should instead use a single master bibliography file as discussed in \S\ref{subsubsec:the-bib-file}.
In this case, you would replace \mintinline{latex}|\addbibresource{\jobname.bib}| with \mintinline{latex}|\addbibresource{master.bib}|, if, for example, you named your \File{.bib} file \File{master.bib}.
Moreover, you would also then completely delete everything from \mintinline{latex}|\begin{filecontents}| to \mintinline{latex}|\end{filecontents}|.

A second thing to note about the code in Listing~\ref{lst:biblatex-example}---particularly in contradistinction to the example for \Package{natbib}---is that the bibliography file is added in the preamble of your document and you \emph{must} specify the file extension (\File{.bib}).

You can have multiple calls to \mintinline{latex}|\addbibresource| in your preamble if you want to use multiple \File{.bib} files.
You cannot pass the names of multiple \File{.bib} files to one call of \mintinline{latex}|\addbibresource|; separate calls are necessary.

Finally, you print your bibliography at the end of your document by writing \mintinline{latex}|\printbibliography|.

Copy and paste the example code in Listing~\ref{lst:biblatex-example} into your editor, and try compiling it using the three compilation steps discussed above.

\begin{listing}[htbp]
	\centering
	\begin{minted}{latex}
\documentclass{article}

\begin{filecontents}{\jobname.bib}
@book{chomsky1995:MP,
    Address = {Cambridge, MA},
    Author = {Chomsky, Noam},
    Publisher = {The MIT Press},
    Title = {The Minimalist Program},
    Year = {1995}}
\end{filecontents}

\usepackage[
    backend=biber,% you can use bibtex instead of biber if you want
    citestyle=authoryear-comp,
    bibstyle=authoryear
]{biblatex}
% This is needed because the default authoryear style puts "In:"
% before a journal name, which is not standard in linguistics styles.
% So this code gets rid of that.
\renewbibmacro{in:}{%
  \ifentrytype{article}{}{\printtext{\bibstring{in}\intitlepunct}}}

\addbibresource{\jobname.bib}

\begin{document}

In Minimalist syntax, S-Structure has been eliminated \autocite[see][73--124]{chomsky1995:MP}.

\printbibliography

\end{document}
	\end{minted}
	\caption{A complete minimal working example with \Package{biblatex}}
	\label{lst:biblatex-example}
\end{listing}

\subsubsection{The backend processor}
\label{subsubsec:the-backend-processor}

As mentioned above in \S\ref{subsubsec:overview}, there are two backend processors that you can use for doing bibliographies with \LaTeX.
The purpose of the backend processor is to serve as an interface between your \File{.tex} document and your bibliography file (\File{.bib}).
That is, they go through your \File{.bib} file and automagically pull out all of the things that you cited in your \File{.tex} document.

The citation package \Package{natbib} is only compatible with the backend processor \hologo{BibTeX}, whereas the citation package \Package{biblatex} is compatible with both \hologo{BibTeX} and Biber.

\subsubsubsection{\hologo{BibTeX}}
\label{subsubsubsec:bibtex}

For a long time, the only backend processor for bibliographies with \LaTeX{} was \hologo{BibTeX}.
Because it was developed quite some time ago, it does not play nicely with accented Latin characters and non-Latin alphabets.
For this reason, I would encourage you to use Biber with \Package{biblatex} instead.

\subsubsubsection{Biber}
\label{subsubsubsec:biber}

Biber is a modern alternative to \hologo{BibTeX}.
In addition to playing nicely with accented Latin characters and non-Latin alphabets, another advantage of Biber is that it is capable of reading and processing many more entry types and data fields than \hologo{BibTeX} is (cf.~\S\ref{subsubsubsec:a-terminological-note} and fn.~\ref{fn:entry-types-biblatex-vs-bibtex}).
This is the basis for many of the advanced and versatile features of the \Package{biblatex} package.

\subsection{Paragraphs}
\label{subsec:paragraphs}

Before moving on to stuff that is specifically useful for linguistics, let's end this general \LaTeX{} section with paragraphs.
Let's do this for two reasons.
First, because semantic markup is awesome.
And second, because a very common \emph{really bad practice} of \LaTeX{} beginners is to insert line breaks all over the place.

Recall from \S\ref{subsec:latex-philosophy} that \LaTeX{} is all about semantic markup.
This goes for paragraphs, too.
People who are used to Word are used to pressing ENTER on the keyboard once in order to separate paragraphs.
Perhaps unsurprisingly then, many new \LaTeX{} users will often do stuff like what is shown in Listing~\ref{lst:bad-practice-with-line-breaks}.

\begin{listing}[htbp]
	\centering
	\begin{minted}{latex}
\documentclass{article}
\begin{document}
This is my first awesome paragraph.\\
This is my second paragraph, which is infinitely less awesome because of the line break.
\end{document}
	\end{minted}
	\caption{Really bad practice for separating paragraphs}
	\label{lst:bad-practice-with-line-breaks}
\end{listing}

The command \mintinline{latex}|\\| does a line break, but it does \emph{not} introduce a new paragraph.
In other words, the (first part of the) second sentence is false.
In \LaTeX{}'s eyes, the sentence ``This is my first awesome paragraph.'' is in the \emph{same paragraph} as the sentence ``This is my second paragraph, which is infinitely less awesome because of the line break.''.

Instead, one should use a command for paragraphs so that we can manipulate the semantics of paragraphs in the preamble of the document in the same way that we manipulated the semantics of sections.
The command for separating one paragraph from another paragraph is \mintinline{latex}|\par|.
This would get really annoying to type in between all of your paragraphs, so, luckily, \LaTeX{} treats an empty line as equivalent to \mintinline{latex}|\par|.

Listing~\ref{lst:good-practice-for-paragraphs} exemplifies \emph{good practice} for typesetting paragraphs.
This good practice allows us to manipulate the semantics of paragraphs in the preamble, so we can typeset them as we like, without having to modify each individual paragraph.

There's generally no need to modify the default semantics for paragraphs.
However, in Listing~\ref{lst:good-practice-for-paragraphs}, I've given semantics that are similar to those used to typeset the paragraphs in this document, which are intended to be much more handout-y and much less essay-y.
Try typesetting Listing~\ref{lst:good-practice-for-paragraphs} yourself and see what happens.%
\footnote{%
Instead of using the two \mintinline{latex}|\setlength| commands in Listing~\ref{lst:good-practice-for-paragraphs}, I would actually just recommend loading the \Package{parskip} package.
It gives you handout-y semantics for paragraphs in a much better and more robust way.
Listing~\ref{lst:good-practice-for-paragraphs} is only given for pedagogical purposes---namely, to illustrate the point of semantic markup for paragraphs, which isn't necessarily clearly illustrated by just loading a package.
Nonetheless, just loading the \Package{parskip} package instead is indeed the better way to go.%
}

\begin{listing}[htbp]
	\centering
	\begin{minted}{latex}
\documentclass{article}
\setlength{\parindent}{0em}
\setlength{\parskip}{1ex}
\begin{document}

This is the first paragraph.
Wasn't that a great topic sentence?

Next paragraph please.
Paragraph number two is the best.
\par
The third paragraph will rule them all.
Sorry about the Lord of the Rings reference.

This concludes my five paragraph essay. As you can see, my conclusion definitely follows.

Yes I can count.
Jeeze.

\end{document}
	\end{minted}
	\caption{Good practice for typesetting paragraphs}
	\label{lst:good-practice-for-paragraphs}
\end{listing}

Lastly, one thing you might notice is that I sometimes put sentences on their own line.
\LaTeX{} treats these sentences as being in the same paragraph because there is no blank line or \mintinline{latex}|\par| between them.
This isn't strictly necessary; you're more than welcome to put all of the sentences in a paragraph on one line, like I did in paragraph `five'.

However, there are two main reasons why it might be nice to put each sentence on its own line.
First, having shorter lines might be easier to read, depending on how your editor is set up.
Second, if you keep your \File{.tex} file under version control, it makes for cleaner diffs.%
\footnote{%
Explaining what exactly a version control system (VCS) is, is quite beyond the scope of this workshop.
Basically, it's a way to keep a history of all of the changes that have been made to a document.
If you delve any further into learning \LaTeX{} or learning to program, you will probably also want to learn a VCS at some point.
The most popular one is \mintinline{sh}|git|.
Also, hopefully you now understand this \href{https://xkcd.com/1285/}{joke}.%
}