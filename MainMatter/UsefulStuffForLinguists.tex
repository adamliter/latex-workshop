% !TEX encoding = UTF-8 Unicode
% !TEX root = ../latex-workshop-for-linguists.tex

\section{Useful stuff for linguists}
\label{sec:useful-stuff-for-linguists}

\subsection{\Package{fontspec} and Unicode}
\label{subsec:fontspec-and-unicode}

In fn.~\ref{fn:diacritics}, I mentioned a better way of typesetting diacritics in \LaTeX.
This is where the \Package{fontspec} package comes in.
Historically, using \LaTeX{} with fonts hasn't really been a thing.
This has changed recently with the advent of two other engines---\hologo{XeLaTeX} and \hologo{LuaLaTeX}---and the \Package{fontspec} package.

If you process a document with one of these two engines, you can use the \Package{fontspec} package to specify which font you want to use.
You can use any font that is installed on your computer.

In \S6 of \TitleText{A Beginner's Guide to \LaTeX{} (on the Mac)}, Alan gives the example of declaring a new font family to use for phonetic fonts.
A complete example of what Alan suggests is given in Listing~\ref{lst:phonetic-font}.%
\footnote{%
See also \href{http://tex.stackexchange.com/q/25249/32888}{this question and its answers on TeX.SX}.
Note that in order for the example in Listing~\ref{lst:phonetic-font} to actually compile, you will need to have the font Doulos SIL installed on your machine.
Covering how to install a font on your machine is beyond the scope of this workshop, since it is very specific to the type of operating system that you are running.
However, you should be able to search for instructions online and pretty easily figure out how to do it.%
}

\begin{listing}[htbp]
	\centering
	\begin{minted}{latex}
\documentclass{article}
\usepackage{fontspec}
\setmainfont[Ligatures=TeX]{Times New Roman}
\newfontfamily\phonetic[]{Doulos SIL}
\usepackage{textglos} % good semantic markup for inline examples, like \xv{...}, \xm{...}, etc.
\begin{document}

The English word \xv{cat} is underlyingly {\phonetic\xm{kæt}}.

\end{document}
	\end{minted}
	\caption{Example of using a distinct font for phonetics}
	\label{lst:phonetic-font}
\end{listing}

In addition to using a separate font for phonetic stuff, you can also use one font for the entire document if the font you are using has glyphs for all of the characters that you need.
I actually really like the Computer Modern font that is the default font in \TeX.
There is a version of the Computer Modern font that you can \href{http://sourceforge.net/projects/cm-unicode/}{download} and install on your machine which has glyphs for a huge range of the Unicode characters.

If you download and install this font on your machine, then you can do something like in Listing~\ref{lst:one-font}, rather than having to use a separate font for special (phonetic) characters.

\begin{listing}[htbp]
	\centering
	\begin{minted}{latex}
\documentclass{article}
\usepackage{fontspec}
\setmainfont[Ligatures=TeX]{CMU Serif Roman}
\usepackage{textglos} % good semantic markup for inline examples, like \xv{...}, \xm{...}, etc.
\begin{document}

The English word \xv{cat} is underlyingly \xm{kæt}.
Also, look at the cool stuff that I can do in the same font: ášçëû!

\end{document}
	\end{minted}
	\caption{Using one font that has a lot of Unicode glyphs}
	\label{lst:one-font}
\end{listing}

Processing either Listing~\ref{lst:phonetic-font}~or~\ref{lst:one-font} with either \hologo{LuaLaTeX} or \hologo{XeLaTeX} will produce a PDF with the correct glyphs (as long as you have the requisite fonts installed on your computer).
One further thing that is \emph{very important} for this to work correctly is to make sure that your \File{.tex} file has the correct character encoding.
It is best to make sure that all of your \File{.tex} files are saved with UTF-8 encoding.
A good editor should allow to see and change the character encoding of the file.
Since this depends on the editor, it is beyond the scope of this document to explain it in any more detail, but you should be able to search online and figure it out.

\subsection{Examples}
\label{subsec:examples}

There are two main packages that I would recommend for typesetting linguistic examples, \Package{gb4e} and \Package{ExPex}.%
\footnote{%
There is also the package called \Package{linguex}.
I don't know much about it.
I've always avoided it because, as Alan points out in \TitleText{A Beginner's Guide to \LaTeX{} (on the Mac)}, its markup is not all that semantic.%
}
The \Package{gb4e} package works well in most cases.
For more complicated use cases, you might want to learn \Package{ExPex}.
However, this workshop will only focus on \Package{gb4e}.

\subsubsection{Basic linguistic example}
\label{subsubsec:basic-linguistic-example}

Listing~\ref{lst:basic-linguistic-example} gives an example of how to typeset some basic examples.%
\footnote{%
Note that it is not strictly necessary to put each example in its own \mintinline{latex}|exe| environment, but it is a good practice to do so, for at least two reasons.
First, the markup is more semantic because \mintinline{latex}|exe| is singular; it refers to \emph{an} example, not a series of examples.
Second, and more practically, it makes moving examples around a lot easier, either within the document or from one document to another.%
}
Try typesetting these examples yourself and see what the result is.
Notice in particular that you can give the examples \mintinline{latex}|\label|s and \mintinline{latex}|\ref|er to them inline just like with captioned things.

\begin{listing}[htbp]
	\centering
	\begin{minted}{latex}
\documentclass{article}

\usepackage{textglos} % good semantic markup for inline examples, like \xv{...}, \xm{...}, etc.

%\usepackage{fixltx2e} % only needed if you have TeX Live < 2015
\newcommand*{\IND}[1]{\textsubscript{#1}}

\usepackage{gb4e}
\noautomath % you should always declare this after loading gb4e

\begin{document}
(\ref{ex:questionable-English}) is marginally acceptable.

\begin{exe}
    \ex[?]{His\IND{i} mother loves every\IND{i} boy no matter what.}
    \label{ex:questionable-English}
\end{exe}
\begin{exe}
    \ex[]{Strong crossover
        \begin{xlist}
            \ex[*]{He\IND{i} loves everyone\IND{i}}
            \ex[*]{She\IND{i} thinks everyone\IND{i} is smart}
            \label{ex:everyone-is-smart}
        \end{xlist}
    }
    \label{ex:strong-crossover}
\end{exe}

The examples in (\ref{ex:strong-crossover}) exemplify the phenomenon of strong crossover.
For example, in (\ref{ex:everyone-is-smart}), \xv{she} c-commands \xv{everyone}.
However, pronouns cannot c-command their binders.

\end{document}
	\end{minted}
	\caption{Typesetting basic linguistics examples}
	\label{lst:basic-linguistic-example}
\end{listing}

\subsubsection{Glossing examples}
\label{subsubsec:glossing-examples}

With \Package{gb4e}, you can also gloss examples.
An example is shown in Listing~\ref{lst:glossed-example}.
The package \Package{leipzig} provides several feature glossing macros, following the glossing standards for the field of linguistics as set forth by the \LGR.
You can find a list of all the macros that \Package{leipzig} provides in the package documentation.
Most of them are pretty straightforward to guess, like \mintinline{latex}|\Acc{}| for accusative and \mintinline{latex}|\Prog{}| for progressive; however, some of them you might need to look up in the package documentation.

\begin{listing}[htbp]
	\centering
	\begin{minted}{latex}
\documentclass{article}
\usepackage{gb4e}
\noautomath
\usepackage{leipzig}
\begin{document}
\begin{exe}
    \ex[]{\gll Der Apfel würde gegessen.\\
    The.\M{}.\Sg{}.\Nom{} apple was eaten\\
    \trans `The apple was eaten'
    }
\end{exe}
\end{document}
	\end{minted}
	\caption{A glossed example with \Package{gb4e}}
	\label{lst:glossed-example}
\end{listing}

While the \LGR{} are relatively comprehensive, you might find yourself in a situation where you need to use a gloss that is not standardized in the \LGR.
If this need arises, \Package{leipzig} provides a way to define your own glossing macros.
One further feature of the \Package{leipzig} package that is really nice is that it provides a way, via the \Package{glossaries} package, to automagically typeset a list of all glosses used in a paper along with their definitions.
This is something that is very commonly printed in a footnote in a paper the first time a gloss is used in that paper.
Listing~\ref{lst:leipzig-example} gives an example of both of these features of the package.

In order to correctly compile Listing~\ref{lst:leipzig-example}, there are some special compilation steps that you will need to go through, just like with bibliographies.
This is because the \Package{glossaries} package needs to make use of some information written to the \File{.aux} file after you run \Engine{latex} for the first time.
The compilation steps that you will need to go through in order to get this to work are the following three steps:

\begin{enumerate}[label={(\roman*)}]

	\item{\mintinline{sh}|latex myfile.tex|}

	\item{\mintinline{sh}|makeglossaries myfile|}

	\item{\mintinline{sh}|latex myfile.tex|}

\end{enumerate}

The second command runs a Perl script that should have been installed when you installed your TeX distribution.
Notably, you should just call the script on the base name of your file; do not specify a file extension when you run the command \Engine{makeglossaries}!
You should also ensure that your file name does not include spaces, which might give the \Engine{makeglossaries} script a hard time.
Just like with bibliographies, it is possible to run these compilation steps from the command line, but it's also possible to configure your editor to run the \Engine{makeglossaries} command.
And again, \Engine{latex} should be treated as a placeholder for whatever engine you are using.
For example, if you are using \hologo{XeLaTeX}, then you should replace \Engine{latex} with \Engine{xelatex} in two of these three steps.

\begin{listing}[htbp]
	\centering
	\begin{minted}{latex}
\documentclass{article}
\usepackage{gb4e}
\noautomath
\usepackage{leipzig}
\makeglossaries
\newleipzig{pol}{pol}{polite}
\begin{document}
As can be seen in (\ref{ex:causative}), \ldots
\footnote{The following abbreviations are used in this paper: \printglosses}

\begin{exe}
    \ex[]{\gll Hasan geu-peu-reubah aneuk miet nyan\\
    Hasan \Third{}\Pol{}-\Caus{}-fall child small \Dem{}\\
    \trans `Hasan caused the child to fall'
    }
    \label{ex:causative}
\end{exe}
\end{document}
	\end{minted}
	\caption{An example with \Package{leipzig} and \Package{glossaries}}
	\label{lst:leipzig-example}
\end{listing}

There are a few further things worth noting about the use of \Package{leipzig} package.
In order for you to be able to automagically print out a list of all glosses used in the paper and their definitions, you must execute \mintinline{latex}|\makeglossaries| in the preamble, after loading the \Package{leipzig} package.
This then allows you to use the command \mintinline{latex}|\printglosses| to print out all glosses used in the document.
Moreover, if you need to define some glosses that are not part of the \LGR, such as a polite feature gloss as shown in the example, you can do so with \mintinline{latex}|\newleipzig|.
New feature glossing macros should be declared after \mintinline{latex}|\makeglossaries|.

\mintinline{latex}|\newleipzig| takes three arguments.
The first argument is a label, which is what will be used to define the macro, except with the first letter being capitalized.
As can be seen in Listing~\ref{lst:leipzig-example}, the first argument to \mintinline{latex}|\newleipzig| is \MacroArg{pol}, which means that the command that will be created for you to use inside the document itself will be \mintinline{latex}|\Pol{}|.
The second argument is the short form of the abbreviation.
This is what will show up in small caps in the gloss itself.
In Listing~\ref{lst:leipzig-example}, the second argument to \mintinline{latex}|\newleipzig| is \MacroArg{pol}, which means that \Pol{} is what will show up in the gloss.
Note that this second argument must be all lowercase.
Finally, the third argument is the definition of the gloss that will show up when you print out the abbreviations and their definitions with \mintinline{latex}|\printglosses|.
In Listing~\ref{lst:leipzig-example}, the third argument to \mintinline{latex}|\newleipzig| is \MacroArg{polite}.

One last thing that is worth noting about the \Package{leipzig} package is that it should be loaded after you load the \Package{hyperref} package, if you are also using that package.
Moreover, \Package{hyperref} should generally be the last package that you load in your preamble, so this means that the last two packages you will probably want to load are \Package{hyperref} and \Package{leipzig}, in that order.

If you haven't already, try typesetting the example in Listing~\ref{lst:leipzig-example} yourself, to make sure that you've got the hang of doing the three different compilation steps.
Note that if your document has a bibliography, you will still need to do the compilation steps for the bibliography as well, so your compilation process might ultimately involve five steps:

\begin{enumerate}[label={(\roman*)}]

	\item{\mintinline{sh}|latex myfile.tex|}

	\item{\mintinline{sh}|biber myfile| OR \mintinline{sh}|bibtex myfile|}

	\item{\mintinline{sh}|makeglossaries myfile|}

	\item{\mintinline{sh}|latex myfile.tex|}

	\item{\mintinline{sh}|latex myfile.tex|}

\end{enumerate}

To make sure you've further got the hang of the \Package{leipzig} package, try defining a new gloss that isn't part of the standardized set from the \LGR.

\subsection{Typesetting trees}
\label{subsec:typesetting-trees}

There are two main packages that are useful for typesetting linguistics trees, \Package{tikz-qtree} and \Package{forest}.
Both are built on top of the package, \Package{tikz}.
I would recommend using \Package{forest}.%
\footnote{%
For further discussion, see \href{http://tex.stackexchange.com/q/5447/32888}{this question and its answers on TeX.SX}.%
}
The syntax for both of these is almost the same, except that \Package{forest} is a bit more forgiving in its syntax for labeling the leafs in a tree.

Let's start with a very basic example, given in Listing~\ref{lst:basic-forest-example}, which will produce what you see in (\ref{ex:basic-forest-example}).

Try typesetting this yourself.

\begin{exe}
	\ex[]{
		\begin{forest} forest default
		[VP
			[V]
			[DP]
		]
		\end{forest}
	}
	\label{ex:basic-forest-example}
\end{exe}

\begin{listing}[htbp]
	\centering
	\begin{minted}{latex}
\documentclass{article}
\usepackage{forest}
\begin{document}

\begin{forest}
[VP
    [V]
    [DP]
]
\end{forest}

\end{document}
	\end{minted}
	\caption{A very basic example with \Package{forest}}
	\label{lst:basic-forest-example}
\end{listing}

There are a few things to note about this basic example.

First, line breaks are not necessary.
You could have produced the same output by writing \mintinline{latex}|[VP [V] [DP] ]|.
Nonetheless, spacing things across lines is generally a good practice because it makes your code much more readable, which in turn makes it easier to debug if you are getting errors when trying to compile your file.

Second, you will see that this tree does not occur inside of a numbered example.
This is something you will almost always want to do, as it is a standard in the field.
Given what you learned in \S\ref{subsec:examples}, you should be able to imagine how to do this.
We will see an explicit example of how to do this below.

Third, you will notice that the branches do not actually connect at the bottom of the ``VP'' node.
Having the branches connect at the bottom of their parent node is the standard style in syntax.
We will see how to draw a tree with this style below as well.

With that in mind, let's take a look at a slightly more complicated example, given in Listing~\ref{lst:typesetting-trees}, which will address some of these issues.
The code in Listing~\ref{lst:typesetting-trees} will produce what you see in (\ref{ex:typesetting-trees}).

\begin{exe}
    \ex[]{
        \begin{forest}
        [TP
            [DP
                [D\\he, name=DP2]
            ]
            [T$'$
                [T\\PRES]
                [\emph{v}P, s sep+=20pt
                    [DP, name=DP1]
                    [\emph{v}$'$
                        [\emph{v}
                            [V\\loves, name=V2]
                            [\emph{v}]
                        ]
                        [VP, s sep+=20pt
                            [V, name=V1]
                            [DP, s sep+=30pt
                                [D\\ø, for c-commanded={red}]
                                [NP
                                    [{Egyptian cotton shirts}, roof]
                                ]
                            ]
                        ]
                    ]
                ]
            ]
        ]
        \draw[->] (V1) [in=-90, out=-90, looseness=0.75] to (V2);
        \draw[->,dashed] (DP1) [in=-90, out=-90, looseness=1.5] to (DP2);
        \end{forest}
    }
    \label{ex:typesetting-trees}
\end{exe}

\begin{listing}[htbp]
	\centering
	\begin{minted}{latex}
\documentclass{article}
\usepackage{forest}
\useforestlibrary*{linguistics}
\usepackage{gb4e}
\noautomath
\begin{document}
\begin{exe}
    \ex[]{
        \begin{forest}
        [TP
            [DP
                [D\\he, name=DP2]
            ]
            [T$'$
                [T\\PRES]
                [\emph{v}P, s sep+=20pt
                    [DP, name=DP1]
                    [\emph{v}$'$
                        [\emph{v}
                            [V\\loves, name=V2]
                            [\emph{v}]
                        ]
                        [VP, s sep+=20pt
                            [V, name=V1]
                            [DP, s sep+=30pt
                                [D\\ø, for c-commanded={red}]
                                [NP
                                    [{Egyptian cotton shirts}, roof]
                                ]
                            ]
                        ]
                    ]
                ]
            ]
        ]
        \draw[->] (V1) [in=-90, out=-90, looseness=0.75] to (V2);
        \draw[->,dashed] (DP1) [in=-90, out=-90, looseness=1.5] to (DP2);
        \end{forest}
    }
\end{exe}
\end{document}
	\end{minted}
	\caption{An example of typesetting a syntax tree using \Package{forest}}
	\label{lst:typesetting-trees}
\end{listing}

A few comments about what you see in Listing~\ref{lst:typesetting-trees} are in order.

First, \Package{forest} is designed to typeset trees as compactly as possible.
Sometimes when you want to show movement, you will therefore need to increase the distance between siblings so that the arrows don't overlap with something else.
This is what \mintinline{latex}|s sep+=| allows you to do.
You can manually set the sibling separation with just \mintinline{latex}|s sep=|, whereas \mintinline{latex}|s sep+=| allows you to increase the default sibling separation value by a certain amount.

If you want to draw an arrow from one leaf in a tree to another leaf, you need to give them names that you can pass to the \mintinline{latex}|\draw| command.

If you want to have spaces in a node in a tree, you will need to surround the content in braces, like what I did with \mintinline{latex}|{Egyptian cotton shirts}| in Listing~\ref{lst:typesetting-trees}.

Finally, you will most likely want all of your trees to be typeset using the default styles of the \Package{linguistics} library for \Package{forest}.
The \Package{linguistics} library was introduced in version 2.0 of the \Package{forest} package. If you have an outdated version of \Package{forest}, I would highly recommend upgrading to the latest version.

By using the \Package{linguistics} library for \Package{forest}, three default styles are applied to all trees: (i) the branches (lines) that connect leafs in a tree are set to meet at the bottom of their parent leaf; (ii) the baseline of the resulting tree is set to the top of the tree, which causes the top of the tree to align with the example number when the tree occurs inside of a numbered example (rather than aligning with the bottom of the tree); and (iii) leafs that span multiple lines are set to have centered horizontal alignment.

If you'd rather that these three styles were not applied to every tree by default, you can load the \Package{linguistics} library with the unstarred variant of the \mintinline{latex}|\useforestlibrary| command (\IE{} \mintinline{latex}|\useforestlibrary{linguistics}|).
Then, to apply the styles to any given tree, you would need to add the three styles to each tree before writing the bracketed structure for the tree, like the following:\\\mintinline{latex}|\begin{forest} sn edges, baseline, for tree={align=center} [VP [V] [DP]]\end{forest}|.

The package \Package{forest} also defines several other handy options, the first two of which are exhibited in Listing~\ref{lst:typesetting-trees}:
\begin{itemize}
	\item{\mintinline{latex}|roof|, for drawing triangles}
	\item{\mintinline{latex}|c-commanded|, for visiting all leafs c-commanded by the current leaf}
	\item{\mintinline{latex}|c-commanders|, for visiting all leafs that c-command the current leaf}
\end{itemize}
These options are intended to be used at a particular leaf and do not apply to the tree as a whole.

There are also two other styles that \Package{forest} defines.
Styles are applied to a tree as a whole and are written before the bracketed structure of the tree.
These two styles are \ForestStyle{nice empty nodes} and \ForestStyle{GP1}.
The style \ForestStyle{nice empty nodes} can be used when you have leafs without labels and you want the branches of child leafs to connect to the branch of the parent leaf.
The style \ForestStyle{GP1} can be used to typeset Government Phonology representations.
If you'd like to see examples of both styles, you can take a look at the \Package{forest} documentation.

\subsection{Typesetting OT tableaux}
\label{subsec:typesetting-OT-tableaux}

There are two options that I would recommend for typesetting OT tableaux.
One package is called \href{http://sanders.phonologist.org/OTtablx/}{\Package{OTtablx}}, which is in beta and not yet on CTAN.
Thus, in order to use it, you need to put the \File{.sty} file in either the same directory as your \File{.tex} file or put it in a local folder that your \TeX{} distribution can see (\IE{} \Directory{texmf/tex/latex}; cf.~\S\ref{subsec:local-files}).%
\footnote{%
Note that if you do decide to try \Package{OTtablx} at some point, it must first be compiled to DVI format and then PDF format because it relies on the package \Package{pstricks}.
If you compile with \hologo{XeLaTeX}, things should work fine, but it will not work with \hologo{pdfLaTeX}.
See \href{http://tex.stackexchange.com/q/8413/32888}{this question and its answers on TeX.SX} for discussion.%
}

To avoid this complication for the sake of an introductory workshop, we will instead use the package called \Package{ot-tableau}, which is available on CTAN (\IE{} it is part of any good \TeX{} distribution, such as TeX Live or \hologo{MiKTeX}).
An example of how to typeset a tableau is given in Listing~\ref{lst:example-tableau}.
This code will produce the result in (\ref{ex:example-tableau}).%
\footnote{%
Disclaimer: I'm not a phonologist.%
}

\begin{exe}
    %
    \ex[]{
        \begin{tableau}{c|c|c}
            \inp{/bad/}          \const{*Voiced-Coda}  \const*{\Constr{Ident-IO}(voice)}  \const*{*[+voi,--son]}
            \cand{bad}           \vio{*!}              \vio{}                             \vio{**}
            \cand[\Optimal]{bat} \vio{}                \vio{*}                            \vio{*}
            \cand{pat}           \vio{}                \vio{**!}                          \vio{}
            \cand{pad}           \vio{*!}              \vio{*}                            \vio{*}
        \end{tableau}
    }
    \label{ex:example-tableau}
    %
\end{exe}

\begin{listing}[htbp]
	\centering
	\begin{minted}{latex}
\documentclass{article}
\usepackage[
    shadedcells,
    notipa
]{ot-tableau}
\usepackage{gb4e}
\noautomath

\newcommand*{\Constr}[1]{\textsc{#1}}

\begin{document}

\begin{exe}
\ex[]{
\begin{tableau}{c|c|c}
\inp{/bad/}          \const{*Voiced-Coda} \const*{\Constr{Ident-IO}(voice)} \const*{*[+voi,--son]}
\cand{bad}           \vio{*!}             \vio{}                            \vio{**}
\cand[\Optimal]{bat} \vio{}               \vio{*}                           \vio{*}
\cand{pat}           \vio{}               \vio{**!}                         \vio{}
\cand{pad}           \vio{*!}             \vio{*}                           \vio{*}
\end{tableau}
}
\end{exe}

\end{document}
	\end{minted}
	\caption{An example of an OT tableau using \Package{ot-tableau}}
	\label{lst:example-tableau}
\end{listing}