% !TEX encoding = UTF-8 Unicode
% !TEX root = ../latex-workshop-for-linguists.tex

\section{Useful stuff for linguists}
\label{sec:useful-stuff-for-linguists}

\subsection{\Package{fontspec} and Unicode}
\label{subsec:fontspec-and-unicode}

In fn.~\ref{fn:diacritics}, I mentioned a better way of typesetting diacritics in \LaTeX.
This is where the \Package{fontspec} package comes in.
Historically, using \LaTeX{} with fonts hasn't really been a thing.
This has changed recently with the advent of two other engines---\hologo{XeLaTeX} and \hologo{LuaLaTeX}---and the \Package{fontspec} package.

If you process a document with one of these two engines, you can use the \Package{fontspec} package to specify which font you want to use.
You can use any font that is installed on your computer.

In \S6 of \TitleText{A Beginner's Guide to \LaTeX{} (on the Mac)}, Alan gives the example of declaring a new font family to use for phonetic fonts.
A complete example of what Alan suggests is given in Listing~\ref{lst:phonetic-font}.%
\footnote{%
See also \href{http://tex.stackexchange.com/q/25249/32888}{this question and its answers on TeX.SX}.
Note that in order for the example in Listing~\ref{lst:phonetic-font} to actually compile, you will need to have the font Doulos SIL installed on your machine.
Covering how to install a font on your machine is beyond the scope of this workshop, since it is very specific to the type of operating system that you are running.
However, you should be able to search for instructions online and pretty easily figure out how to do it.%
}

\begin{listing}[htbp]
	\centering
	\begin{minted}{latex}
\documentclass{article}
\usepackage{fontspec}
\setmainfont[Ligatures=TeX]{Times New Roman}
\newfontfamily\phonetic[]{Doulos SIL}
\usepackage{textglos} % good semantic markup for inline examples, like \xv{...}, \xm{...}, etc.
\begin{document}

The English word \xv{cat} is underlyingly {\phonetic\xm{kæt}}.

\end{document}
	\end{minted}
	\caption{Example of using a distinct font for phonetics}
	\label{lst:phonetic-font}
\end{listing}

In addition to using a separate font for phonetic stuff, you can also use one font for the entire document if the font you are using has glyphs for all of the characters that you need.
I actually really like the Computer Modern font that is the default font in \TeX.
There is a version of the Computer Modern font that you can \href{http://sourceforge.net/projects/cm-unicode/}{download} and install on your machine which has glyphs for a huge range of the Unicode characters.

If you download and install this font on your machine, then you can do something like in Listing~\ref{lst:one-font}, rather than having to use a separate font for special (phonetic) characters.

\begin{listing}[htbp]
	\centering
	\begin{minted}{latex}
\documentclass{article}
\usepackage{fontspec}
\setmainfont[Ligatures=TeX]{CMU Serif Roman}
\usepackage{textglos} % good semantic markup for inline examples, like \xv{...}, \xm{...}, etc.
\begin{document}

The English word \xv{cat} is underlyingly \xm{kæt}.
Also, look at the cool stuff that I can do in the same font: ášçëû!

\end{document}
	\end{minted}
	\caption{Using one font that has a lot of Unicode glyphs}
	\label{lst:one-font}
\end{listing}

Processing either Listing~\ref{lst:phonetic-font}~or~\ref{lst:one-font} with either \hologo{LuaLaTeX} or \hologo{XeLaTeX} will produce a PDF with the correct glyphs (as long as you have the requisite fonts installed on your computer).
One further thing that is \emph{very important} for this to work correctly is to make sure that your \File{.tex} file has the correct character encoding.
It is best to make sure that all of your \File{.tex} files are saved with UTF-8 encoding.
A good editor should allow to see and change the character encoding of the file.
Since this depends on the editor, it is beyond the scope of the workshop to explain it in any more detail, but you should be able to search online and figure it out.

\subsection{Examples}
\label{subsec:examples}

There are two main packages that I would recommend for typesetting linguistic examples, \Package{gb4e} and \Package{ExPex}.%
\footnote{%
There is also the package called \Package{linguex}.
I don't know much about it.
I've always avoided it because, as Alan points out in \TitleText{A Beginner's Guide to \LaTeX{} (on the Mac)}, its markup is not all that semantic.%
}
The \Package{gb4e} package works well in most cases.
For more complicated use cases, you might want to learn \Package{ExPex}.
However, this workshop will only focus on \Package{gb4e}.

\subsubsection{Basic linguistic example}
\label{subsubsec:basic-linguistic-example}

Listing~\ref{lst:basic-linguistic-example} gives an example of how to typeset some basic examples.%
\footnote{%
Note that it is not strictly necessary to put each example in its own \mintinline{latex}|exe| environment, but it is a good practice to do so, for at least two reasons.
First, the markup is more semantic because \mintinline{latex}|exe| is singular; it refers to \emph{an} example, not a series of examples.
Second, and more practically, it makes moving examples around a lot easier, either within the document or from one document to another.%
}
Try typesetting these examples yourself and see what the result is.
Notice in particular that you can give the examples \mintinline{latex}|\label|s and \mintinline{latex}|\ref|er to them inline just like with captioned things.

\begin{listing}[htbp]
	\centering
	\begin{minted}{latex}
\documentclass{article}

\usepackage{textglos} % good semantic markup for inline examples, like \xv{...}, \xm{...}, etc.

%\usepackage{fixltx2e} % only needed if you have TeX Live < 2015
\newcommand*{\IND}[1]{\textsubscript{#1}}

\usepackage{gb4e}
\noautomath % you should always declare this after loading gb4e

\begin{document}
(\ref{ex:questionable-English}) is marginally acceptable.

\begin{exe}
    \ex[?]{His\IND{i} mother loves every\IND{i} boy no matter what.}
    \label{ex:questionable-English}
\end{exe}
\begin{exe}
    \ex[]{Strong crossover
        \begin{xlist}
            \ex[*]{He\IND{i} loves everyone\IND{i}}
            \ex[*]{She\IND{i} thinks everyone\IND{i} is smart}
            \label{ex:everyone-is-smart}
        \end{xlist}
    }
    \label{ex:strong-crossover}
\end{exe}

The examples in (\ref{ex:strong-crossover}) exemplify the phenomenon of strong crossover.
For example, in (\ref{ex:everyone-is-smart}), \xv{she} c-commands \xv{everyone}.
However, pronouns cannot c-command their binders.

\end{document}
	\end{minted}
	\caption{Typesetting basic linguistics examples}
	\label{lst:basic-linguistic-example}
\end{listing}

\subsubsection{Glossing examples}
\label{subsubsec:glossing-examples}

With \Package{gb4e}, you can also gloss examples.
An example is shown in Listing~\ref{lst:glossed-example}.%
\footnote{%
I'm a bit loathe to recommend this because the package currently has a bug, but it's a really great package.
For an even better way of typesetting common linguistic gloss abbreviations than what you see in Listing~\ref{lst:glossed-example}, check out the \href{http://ctan.org/pkg/leipzig}{\Package{leipzig}} package.
If you use \Package{leipzig} \emph{without} the \Package{glossaries} package, you shouldn't run into any trouble.
However, if you use \Package{leipzig} in conjunction with the \Package{glossaries} package---which is a really great thing to do because it can then automagically generate a list of all glosses that you've used in your document---you will run into problems.
There is a really hacky workaround \href{http://tex.stackexchange.com/a/204615/32888}{here}, but it's a bug that should ultimately be fixed in the \Package{leipzig} package.
I've tried contacting the maintainer of the package, but I haven't gotten a response.
I plan to try contacting her again soon, so hopefully the bug will be fixed at some point.%
}

\begin{listing}[htbp]
	\centering
	\begin{minted}{latex}
\documentclass{article}
\newcommand*{\Fts}[1]{\textsc{#1}}
\usepackage{gb4e}
\noautomath
\begin{document}
\begin{exe}
    \ex[]{\gll Der Apfel würde gegessen.\\
    The.\Fts{m}.\Fts{sg}.\Fts{nom} apple was eaten\\
    \trans `The apple was eaten'
    }
\end{exe}
\end{document}
	\end{minted}
	\caption{A glossed example with \Package{gb4e}}
	\label{lst:glossed-example}
\end{listing}

Try typesetting (\ref{ex:try-glossed-example}) yourself, giving it a \mintinline{latex}|\label|, and \mintinline{latex}|\ref|erring to it in the text of your document. 

\begin{exe}
	%
	\ex[]{\gll Hasan geu-peu-reubah aneuk miet nyan\\
	Hasan \Third\Pol-\Caus-fall child small \Dem\\
	\trans `Hasan caused the child to fall'
	}
	\label{ex:try-glossed-example}
	%
\end{exe}

\subsection{Typesetting trees}
\label{subsec:typesetting-trees}

There are two main packages that are useful for typesetting linguistics trees, \Package{tikz-qtree} and \Package{forest}.
Both are built on top of the package, \Package{tikz}.
I would recommend using \Package{forest}.%
\footnote{%
For further discussion, see \href{http://tex.stackexchange.com/q/5447/32888}{this question and its answers on TeX.SX}.%
}
The syntax for both of these is almost the same, and \Package{forest} is a lot more powerful.
However, you don't need to know its internals to do the basic stuff.

Let's start with a very basic example, given in Listing~\ref{lst:basic-forest-example}, which will produce what you see in (\ref{ex:basic-forest-example}).

Try typesetting this yourself.

\begin{exe}
	\ex[]{
		\begin{forest}
		[VP
			[V]
			[DP]
		]
		\end{forest}
	}
	\label{ex:basic-forest-example}
\end{exe}

\begin{listing}[htbp]
	\centering
	\begin{minted}{latex}
\documentclass{article}
\usepackage{forest}
\begin{document}

\begin{forest}
[VP
    [V]
    [DP]
]
\end{forest}

\end{document}
	\end{minted}
	\caption{A very basic example with \Package{forest}}
	\label{lst:basic-forest-example}
\end{listing}

There are a few things to note about this basic example.

First, line breaks are not necessary.
You could have produced the same output by writing \mintinline{latex}|[VP [V] [DP] ]|.
Nonetheless, spacing things across lines is generally a good practice because it makes your code much more readable, which also in turn makes it much easier to debug if you are getting errors when trying to compile your file.

Second, you will see that this tree does not occur inside of a numbered example.
This is something you will almost always want to do, as it is a standard in the field.
Given what you learned in \S\ref{subsec:examples}, you should be able to imagine how to do this.
And we will see an explicit example of how to do this below.

Third, you will notice that the branches do not actually connect at the bottom of the ``VP'' node.
Having the branches connect at the bottom of their parent node is the standard style in syntax.
We will see how to draw a tree with this style below as well.

With that in mind, let's take a look at a slightly more complicated example, given in Listing~\ref{lst:typesetting-trees}, which will address some of these issues.
The code in Listing~\ref{lst:typesetting-trees} will produce what you see in (\ref{ex:typesetting-trees}).

\begin{exe}
    \ex[]{
        \begin{forest} qtree
        [TP
            [DP
                [D\\he, name=DP2]
            ]
            [T$'$
                [T\\PRES]
                [\emph{v}P, s sep+=20pt
                    [DP, name=DP1]
                    [\emph{v}$'$
                        [\emph{v}
                            [V\\loves, name=V2]
                            [\emph{v}]
                        ]
                        [VP, s sep+=20pt
                            [V, name=V1]
                            [DP, s sep+=30pt
                                [D\\ø]
                                [NP
                                    [{Egyptian cotton shirts}, triangle]
                                ]
                            ]
                        ]
                    ]
                ]
            ]
        ]
        \draw[->] (V1) [in=-90, out=-90, looseness=1.5] to (V2);
        \draw[->,dashed] (DP1) [in=-90, out=-90, looseness=1.5] to (DP2);
        \end{forest}
    }
    \label{ex:typesetting-trees}
\end{exe}

\begin{listing}[htbp]
	\centering
	\begin{minted}{latex}
\documentclass{article}
\usepackage{forest}
\forestset{
    qtree/.style={ % define a style that imitates the qtree package
        for tree={
            parent anchor=south,
            child anchor=north,
            align=center,
            inner sep=1pt
        }
    },
    .style={ % declare styles to apply to all forest environments
        qtree,
        baseline
    }
}
\usepackage{gb4e}
\noautomath
\begin{document}
\begin{exe}
    \ex[]{
        \begin{forest}
        [TP
            [DP
                [D\\he, name=DP2]
            ]
            [T$'$
                [T\\PRES]
                [\emph{v}P, s sep+=20pt
                    [DP, name=DP1]
                    [\emph{v}$'$
                        [\emph{v}
                            [V\\loves, name=V2]
                            [\emph{v}]
                        ]
                        [VP, s sep+=20pt
                            [V, name=V1]
                            [DP, s sep+=30pt
                                [D\\ø]
                                [NP
                                    [{Egyptian cotton shirts}, triangle]
                                ]
                            ]
                        ]
                    ]
                ]
            ]
        ]
        \draw[->] (V1) [in=-90, out=-90, looseness=1.5] to (V2);
        \draw[->,dashed] (DP1) [in=-90, out=-90, looseness=1.5] to (DP2);
        \end{forest}
    }
\end{exe}
\end{document}
	\end{minted}
	\caption{An example of typesetting a syntax tree using \Package{forest}}
	\label{lst:typesetting-trees}
\end{listing}

A few comments about what you see in Listing~\ref{lst:typesetting-trees} are in order.

First, \Package{forest} is designed to typeset trees as compactly as possible.
Sometimes when you want to show movement, you will therefore need to increase the distance between siblings so that arrows don't overlap with something else.
This is what \mintinline{latex}|s sep+=| allows you to do.
You can manually set the sibling separation with just \mintinline{latex}|s sep=|, whereas \mintinline{latex}|s sep+=| allows you to increase the default sibling separation value by a certain amount.

If you want to draw an arrow from one leaf in a tree to another leaf, you need to give them names that you can pass to the \mintinline{latex}|\draw| command.

If you want to have spaces in a node in a tree, you will need to surround the content in braces, like what I did with \mintinline{latex}|{Egyptian cotton shirts}| in Listing~\ref{lst:typesetting-trees}.

Finally, you will most likely want all of your trees to be typeset using both \mintinline{latex}|baseline| and \mintinline{latex}|qtree|.
By declaring \mintinline{latex}|\forestset{.style={qtree,baseline}}| in your preamble, you tell \Package{forest} to typeset all of your trees like this.%
\footnote{%
The method for declaring default styles is going to change in a future version of \Package{forest}.
I will update this handout when it does change.
}

The first option, \mintinline{latex}|baseline|, ensures that the baseline of the typeset object is the top of the tree, rather than the bottom of the tree.
This means that ``TP'' will be typeset as aligned with the example number (if you put it inside of an example environment) rather than ``Egyptian cotton shirts''.

The second option, \mintinline{latex}|qtree|, is a style that mimics the \Package{qtree} package style for trees.%
\footnote{%
Note that this style is not defined by \Package{forest}.
We must define it ourselves.%
}
This is the style that is standard for typesetting trees.
Specifically, it ensures that the branches of the tree connect at the bottom of each node.
Try typesetting a tree without using the \mintinline{latex}|qtree| option to see what happens.

\subsection{Typesetting OT tableaux}
\label{subsec:typesetting-OT-tableaux}

There are two options that I would recommend for typesetting OT tableaux.
One package is called \href{http://sanders.phonologist.org/OTtablx/}{\Package{OTtablx}}, which is in beta and not yet on CTAN.
Thus, in order to use it, you need to put the \File{.sty} file in either the same directory as your \File{.tex} file or put it in a local folder that your \TeX{} distribution can see (\IE{} \Directory{texmf/tex/latex}; cf.~\S\ref{subsec:local-files}).%
\footnote{%
Note that if you do decide to try \Package{OTtablx} at some point, it must first be compiled to DVI format and then PDF format because it relies on the package \Package{pstricks}.
If you compile with \hologo{XeLaTeX}, things should work fine, but it will not work with \hologo{pdfLaTeX}.
See \href{http://tex.stackexchange.com/q/8413/32888}{this question and its answers on TeX.SX} for discussion.%
}

To avoid this complication for the sake of an introductory workshop, we will instead use the package called \Package{ot-tableau}, which is available on CTAN (\IE{} it is part of any good \TeX{} distribution, such as TeX Live or \hologo{MiKTeX}).
An example of how to typeset a tableau is given in Listing~\ref{lst:example-tableau}.
This code will produce the result in (\ref{ex:example-tableau}).%
\footnote{%
Disclaimer: I'm not a phonologist.%
}

\begin{exe}
    %
    \ex[]{
        \begin{tableau}{c|c|c}
            \inp{/bad/}          \const{*Voiced-Coda}  \const*{\Constr{Ident-IO}(voice)}  \const*{*[+voi,--son]}
            \cand{bad}           \vio{*!}              \vio{}                             \vio{**}
            \cand[\Optimal]{bat} \vio{}                \vio{*}                            \vio{*}
            \cand{pat}           \vio{}                \vio{**!}                          \vio{}
            \cand{pad}           \vio{*!}              \vio{*}                            \vio{*}
        \end{tableau}
    }
    \label{ex:example-tableau}
    %
\end{exe}

\begin{listing}[htbp]
	\centering
	\begin{minted}{latex}
\documentclass{article}
\usepackage[
    shadedcells,
    notipa
]{ot-tableau}
\usepackage{gb4e}
\noautomath

\newcommand*{\Constr}[1]{\textsc{#1}}

\begin{document}

\begin{exe}
\ex[]{
\begin{tableau}{c|c|c}
\inp{/bad/}          \const{*Voiced-Coda} \const*{\Constr{Ident-IO}(voice)} \const*{*[+voi,--son]}
\cand{bad}           \vio{*!}             \vio{}                            \vio{**}
\cand[\Optimal]{bat} \vio{}               \vio{*}                           \vio{*}
\cand{pat}           \vio{}               \vio{**!}                         \vio{}
\cand{pad}           \vio{*!}             \vio{*}                           \vio{*}
\end{tableau}
}
\end{exe}

\end{document}
	\end{minted}
	\caption{An example of an OT tableau using \Package{ot-tableau}}
	\label{lst:example-tableau}
\end{listing}